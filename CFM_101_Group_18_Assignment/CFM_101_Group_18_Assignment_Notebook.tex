\documentclass[11pt]{article}

    \usepackage[breakable]{tcolorbox}
    \usepackage{parskip} % Stop auto-indenting (to mimic markdown behaviour)
    
    \usepackage{iftex}
    \ifPDFTeX
    	\usepackage[T1]{fontenc}
    	\usepackage{mathpazo}
    \else
    	\usepackage{fontspec}
    \fi

    % Basic figure setup, for now with no caption control since it's done
    % automatically by Pandoc (which extracts ![](path) syntax from Markdown).
    \usepackage{graphicx}
    % Maintain compatibility with old templates. Remove in nbconvert 6.0
    \let\Oldincludegraphics\includegraphics
    % Ensure that by default, figures have no caption (until we provide a
    % proper Figure object with a Caption API and a way to capture that
    % in the conversion process - todo).
    \usepackage{caption}
    \DeclareCaptionFormat{nocaption}{}
    \captionsetup{format=nocaption,aboveskip=0pt,belowskip=0pt}

    \usepackage{float}
    \floatplacement{figure}{H} % forces figures to be placed at the correct location
    \usepackage{xcolor} % Allow colors to be defined
    \usepackage{enumerate} % Needed for markdown enumerations to work
    \usepackage{geometry} % Used to adjust the document margins
    \usepackage{amsmath} % Equations
    \usepackage{amssymb} % Equations
    \usepackage{textcomp} % defines textquotesingle
    % Hack from http://tex.stackexchange.com/a/47451/13684:
    \AtBeginDocument{%
        \def\PYZsq{\textquotesingle}% Upright quotes in Pygmentized code
    }
    \usepackage{upquote} % Upright quotes for verbatim code
    \usepackage{eurosym} % defines \euro
    \usepackage[mathletters]{ucs} % Extended unicode (utf-8) support
    \usepackage{fancyvrb} % verbatim replacement that allows latex
    \usepackage{grffile} % extends the file name processing of package graphics 
                         % to support a larger range
    \makeatletter % fix for old versions of grffile with XeLaTeX
    \@ifpackagelater{grffile}{2019/11/01}
    {
      % Do nothing on new versions
    }
    {
      \def\Gread@@xetex#1{%
        \IfFileExists{"\Gin@base".bb}%
        {\Gread@eps{\Gin@base.bb}}%
        {\Gread@@xetex@aux#1}%
      }
    }
    \makeatother
    \usepackage[Export]{adjustbox} % Used to constrain images to a maximum size
    \adjustboxset{max size={0.9\linewidth}{0.9\paperheight}}

    % The hyperref package gives us a pdf with properly built
    % internal navigation ('pdf bookmarks' for the table of contents,
    % internal cross-reference links, web links for URLs, etc.)
    \usepackage{hyperref}
    % The default LaTeX title has an obnoxious amount of whitespace. By default,
    % titling removes some of it. It also provides customization options.
    \usepackage{titling}
    \usepackage{longtable} % longtable support required by pandoc >1.10
    \usepackage{booktabs}  % table support for pandoc > 1.12.2
    \usepackage[inline]{enumitem} % IRkernel/repr support (it uses the enumerate* environment)
    \usepackage[normalem]{ulem} % ulem is needed to support strikethroughs (\sout)
                                % normalem makes italics be italics, not underlines
    \usepackage{mathrsfs}
    

    
    % Colors for the hyperref package
    \definecolor{urlcolor}{rgb}{0,.145,.698}
    \definecolor{linkcolor}{rgb}{.71,0.21,0.01}
    \definecolor{citecolor}{rgb}{.12,.54,.11}

    % ANSI colors
    \definecolor{ansi-black}{HTML}{3E424D}
    \definecolor{ansi-black-intense}{HTML}{282C36}
    \definecolor{ansi-red}{HTML}{E75C58}
    \definecolor{ansi-red-intense}{HTML}{B22B31}
    \definecolor{ansi-green}{HTML}{00A250}
    \definecolor{ansi-green-intense}{HTML}{007427}
    \definecolor{ansi-yellow}{HTML}{DDB62B}
    \definecolor{ansi-yellow-intense}{HTML}{B27D12}
    \definecolor{ansi-blue}{HTML}{208FFB}
    \definecolor{ansi-blue-intense}{HTML}{0065CA}
    \definecolor{ansi-magenta}{HTML}{D160C4}
    \definecolor{ansi-magenta-intense}{HTML}{A03196}
    \definecolor{ansi-cyan}{HTML}{60C6C8}
    \definecolor{ansi-cyan-intense}{HTML}{258F8F}
    \definecolor{ansi-white}{HTML}{C5C1B4}
    \definecolor{ansi-white-intense}{HTML}{A1A6B2}
    \definecolor{ansi-default-inverse-fg}{HTML}{FFFFFF}
    \definecolor{ansi-default-inverse-bg}{HTML}{000000}

    % common color for the border for error outputs.
    \definecolor{outerrorbackground}{HTML}{FFDFDF}

    % commands and environments needed by pandoc snippets
    % extracted from the output of `pandoc -s`
    \providecommand{\tightlist}{%
      \setlength{\itemsep}{0pt}\setlength{\parskip}{0pt}}
    \DefineVerbatimEnvironment{Highlighting}{Verbatim}{commandchars=\\\{\}}
    % Add ',fontsize=\small' for more characters per line
    \newenvironment{Shaded}{}{}
    \newcommand{\KeywordTok}[1]{\textcolor[rgb]{0.00,0.44,0.13}{\textbf{{#1}}}}
    \newcommand{\DataTypeTok}[1]{\textcolor[rgb]{0.56,0.13,0.00}{{#1}}}
    \newcommand{\DecValTok}[1]{\textcolor[rgb]{0.25,0.63,0.44}{{#1}}}
    \newcommand{\BaseNTok}[1]{\textcolor[rgb]{0.25,0.63,0.44}{{#1}}}
    \newcommand{\FloatTok}[1]{\textcolor[rgb]{0.25,0.63,0.44}{{#1}}}
    \newcommand{\CharTok}[1]{\textcolor[rgb]{0.25,0.44,0.63}{{#1}}}
    \newcommand{\StringTok}[1]{\textcolor[rgb]{0.25,0.44,0.63}{{#1}}}
    \newcommand{\CommentTok}[1]{\textcolor[rgb]{0.38,0.63,0.69}{\textit{{#1}}}}
    \newcommand{\OtherTok}[1]{\textcolor[rgb]{0.00,0.44,0.13}{{#1}}}
    \newcommand{\AlertTok}[1]{\textcolor[rgb]{1.00,0.00,0.00}{\textbf{{#1}}}}
    \newcommand{\FunctionTok}[1]{\textcolor[rgb]{0.02,0.16,0.49}{{#1}}}
    \newcommand{\RegionMarkerTok}[1]{{#1}}
    \newcommand{\ErrorTok}[1]{\textcolor[rgb]{1.00,0.00,0.00}{\textbf{{#1}}}}
    \newcommand{\NormalTok}[1]{{#1}}
    
    % Additional commands for more recent versions of Pandoc
    \newcommand{\ConstantTok}[1]{\textcolor[rgb]{0.53,0.00,0.00}{{#1}}}
    \newcommand{\SpecialCharTok}[1]{\textcolor[rgb]{0.25,0.44,0.63}{{#1}}}
    \newcommand{\VerbatimStringTok}[1]{\textcolor[rgb]{0.25,0.44,0.63}{{#1}}}
    \newcommand{\SpecialStringTok}[1]{\textcolor[rgb]{0.73,0.40,0.53}{{#1}}}
    \newcommand{\ImportTok}[1]{{#1}}
    \newcommand{\DocumentationTok}[1]{\textcolor[rgb]{0.73,0.13,0.13}{\textit{{#1}}}}
    \newcommand{\AnnotationTok}[1]{\textcolor[rgb]{0.38,0.63,0.69}{\textbf{\textit{{#1}}}}}
    \newcommand{\CommentVarTok}[1]{\textcolor[rgb]{0.38,0.63,0.69}{\textbf{\textit{{#1}}}}}
    \newcommand{\VariableTok}[1]{\textcolor[rgb]{0.10,0.09,0.49}{{#1}}}
    \newcommand{\ControlFlowTok}[1]{\textcolor[rgb]{0.00,0.44,0.13}{\textbf{{#1}}}}
    \newcommand{\OperatorTok}[1]{\textcolor[rgb]{0.40,0.40,0.40}{{#1}}}
    \newcommand{\BuiltInTok}[1]{{#1}}
    \newcommand{\ExtensionTok}[1]{{#1}}
    \newcommand{\PreprocessorTok}[1]{\textcolor[rgb]{0.74,0.48,0.00}{{#1}}}
    \newcommand{\AttributeTok}[1]{\textcolor[rgb]{0.49,0.56,0.16}{{#1}}}
    \newcommand{\InformationTok}[1]{\textcolor[rgb]{0.38,0.63,0.69}{\textbf{\textit{{#1}}}}}
    \newcommand{\WarningTok}[1]{\textcolor[rgb]{0.38,0.63,0.69}{\textbf{\textit{{#1}}}}}
    
    
    % Define a nice break command that doesn't care if a line doesn't already
    % exist.
    \def\br{\hspace*{\fill} \\* }
    % Math Jax compatibility definitions
    \def\gt{>}
    \def\lt{<}
    \let\Oldtex\TeX
    \let\Oldlatex\LaTeX
    \renewcommand{\TeX}{\textrm{\Oldtex}}
    \renewcommand{\LaTeX}{\textrm{\Oldlatex}}
    % Document parameters
    % Document title
    \title{CFM\_101\_Group\_18\_Assignment\_Notebook}
    
    
    
    
    
% Pygments definitions
\makeatletter
\def\PY@reset{\let\PY@it=\relax \let\PY@bf=\relax%
    \let\PY@ul=\relax \let\PY@tc=\relax%
    \let\PY@bc=\relax \let\PY@ff=\relax}
\def\PY@tok#1{\csname PY@tok@#1\endcsname}
\def\PY@toks#1+{\ifx\relax#1\empty\else%
    \PY@tok{#1}\expandafter\PY@toks\fi}
\def\PY@do#1{\PY@bc{\PY@tc{\PY@ul{%
    \PY@it{\PY@bf{\PY@ff{#1}}}}}}}
\def\PY#1#2{\PY@reset\PY@toks#1+\relax+\PY@do{#2}}

\@namedef{PY@tok@w}{\def\PY@tc##1{\textcolor[rgb]{0.73,0.73,0.73}{##1}}}
\@namedef{PY@tok@c}{\let\PY@it=\textit\def\PY@tc##1{\textcolor[rgb]{0.25,0.50,0.50}{##1}}}
\@namedef{PY@tok@cp}{\def\PY@tc##1{\textcolor[rgb]{0.74,0.48,0.00}{##1}}}
\@namedef{PY@tok@k}{\let\PY@bf=\textbf\def\PY@tc##1{\textcolor[rgb]{0.00,0.50,0.00}{##1}}}
\@namedef{PY@tok@kp}{\def\PY@tc##1{\textcolor[rgb]{0.00,0.50,0.00}{##1}}}
\@namedef{PY@tok@kt}{\def\PY@tc##1{\textcolor[rgb]{0.69,0.00,0.25}{##1}}}
\@namedef{PY@tok@o}{\def\PY@tc##1{\textcolor[rgb]{0.40,0.40,0.40}{##1}}}
\@namedef{PY@tok@ow}{\let\PY@bf=\textbf\def\PY@tc##1{\textcolor[rgb]{0.67,0.13,1.00}{##1}}}
\@namedef{PY@tok@nb}{\def\PY@tc##1{\textcolor[rgb]{0.00,0.50,0.00}{##1}}}
\@namedef{PY@tok@nf}{\def\PY@tc##1{\textcolor[rgb]{0.00,0.00,1.00}{##1}}}
\@namedef{PY@tok@nc}{\let\PY@bf=\textbf\def\PY@tc##1{\textcolor[rgb]{0.00,0.00,1.00}{##1}}}
\@namedef{PY@tok@nn}{\let\PY@bf=\textbf\def\PY@tc##1{\textcolor[rgb]{0.00,0.00,1.00}{##1}}}
\@namedef{PY@tok@ne}{\let\PY@bf=\textbf\def\PY@tc##1{\textcolor[rgb]{0.82,0.25,0.23}{##1}}}
\@namedef{PY@tok@nv}{\def\PY@tc##1{\textcolor[rgb]{0.10,0.09,0.49}{##1}}}
\@namedef{PY@tok@no}{\def\PY@tc##1{\textcolor[rgb]{0.53,0.00,0.00}{##1}}}
\@namedef{PY@tok@nl}{\def\PY@tc##1{\textcolor[rgb]{0.63,0.63,0.00}{##1}}}
\@namedef{PY@tok@ni}{\let\PY@bf=\textbf\def\PY@tc##1{\textcolor[rgb]{0.60,0.60,0.60}{##1}}}
\@namedef{PY@tok@na}{\def\PY@tc##1{\textcolor[rgb]{0.49,0.56,0.16}{##1}}}
\@namedef{PY@tok@nt}{\let\PY@bf=\textbf\def\PY@tc##1{\textcolor[rgb]{0.00,0.50,0.00}{##1}}}
\@namedef{PY@tok@nd}{\def\PY@tc##1{\textcolor[rgb]{0.67,0.13,1.00}{##1}}}
\@namedef{PY@tok@s}{\def\PY@tc##1{\textcolor[rgb]{0.73,0.13,0.13}{##1}}}
\@namedef{PY@tok@sd}{\let\PY@it=\textit\def\PY@tc##1{\textcolor[rgb]{0.73,0.13,0.13}{##1}}}
\@namedef{PY@tok@si}{\let\PY@bf=\textbf\def\PY@tc##1{\textcolor[rgb]{0.73,0.40,0.53}{##1}}}
\@namedef{PY@tok@se}{\let\PY@bf=\textbf\def\PY@tc##1{\textcolor[rgb]{0.73,0.40,0.13}{##1}}}
\@namedef{PY@tok@sr}{\def\PY@tc##1{\textcolor[rgb]{0.73,0.40,0.53}{##1}}}
\@namedef{PY@tok@ss}{\def\PY@tc##1{\textcolor[rgb]{0.10,0.09,0.49}{##1}}}
\@namedef{PY@tok@sx}{\def\PY@tc##1{\textcolor[rgb]{0.00,0.50,0.00}{##1}}}
\@namedef{PY@tok@m}{\def\PY@tc##1{\textcolor[rgb]{0.40,0.40,0.40}{##1}}}
\@namedef{PY@tok@gh}{\let\PY@bf=\textbf\def\PY@tc##1{\textcolor[rgb]{0.00,0.00,0.50}{##1}}}
\@namedef{PY@tok@gu}{\let\PY@bf=\textbf\def\PY@tc##1{\textcolor[rgb]{0.50,0.00,0.50}{##1}}}
\@namedef{PY@tok@gd}{\def\PY@tc##1{\textcolor[rgb]{0.63,0.00,0.00}{##1}}}
\@namedef{PY@tok@gi}{\def\PY@tc##1{\textcolor[rgb]{0.00,0.63,0.00}{##1}}}
\@namedef{PY@tok@gr}{\def\PY@tc##1{\textcolor[rgb]{1.00,0.00,0.00}{##1}}}
\@namedef{PY@tok@ge}{\let\PY@it=\textit}
\@namedef{PY@tok@gs}{\let\PY@bf=\textbf}
\@namedef{PY@tok@gp}{\let\PY@bf=\textbf\def\PY@tc##1{\textcolor[rgb]{0.00,0.00,0.50}{##1}}}
\@namedef{PY@tok@go}{\def\PY@tc##1{\textcolor[rgb]{0.53,0.53,0.53}{##1}}}
\@namedef{PY@tok@gt}{\def\PY@tc##1{\textcolor[rgb]{0.00,0.27,0.87}{##1}}}
\@namedef{PY@tok@err}{\def\PY@bc##1{{\setlength{\fboxsep}{\string -\fboxrule}\fcolorbox[rgb]{1.00,0.00,0.00}{1,1,1}{\strut ##1}}}}
\@namedef{PY@tok@kc}{\let\PY@bf=\textbf\def\PY@tc##1{\textcolor[rgb]{0.00,0.50,0.00}{##1}}}
\@namedef{PY@tok@kd}{\let\PY@bf=\textbf\def\PY@tc##1{\textcolor[rgb]{0.00,0.50,0.00}{##1}}}
\@namedef{PY@tok@kn}{\let\PY@bf=\textbf\def\PY@tc##1{\textcolor[rgb]{0.00,0.50,0.00}{##1}}}
\@namedef{PY@tok@kr}{\let\PY@bf=\textbf\def\PY@tc##1{\textcolor[rgb]{0.00,0.50,0.00}{##1}}}
\@namedef{PY@tok@bp}{\def\PY@tc##1{\textcolor[rgb]{0.00,0.50,0.00}{##1}}}
\@namedef{PY@tok@fm}{\def\PY@tc##1{\textcolor[rgb]{0.00,0.00,1.00}{##1}}}
\@namedef{PY@tok@vc}{\def\PY@tc##1{\textcolor[rgb]{0.10,0.09,0.49}{##1}}}
\@namedef{PY@tok@vg}{\def\PY@tc##1{\textcolor[rgb]{0.10,0.09,0.49}{##1}}}
\@namedef{PY@tok@vi}{\def\PY@tc##1{\textcolor[rgb]{0.10,0.09,0.49}{##1}}}
\@namedef{PY@tok@vm}{\def\PY@tc##1{\textcolor[rgb]{0.10,0.09,0.49}{##1}}}
\@namedef{PY@tok@sa}{\def\PY@tc##1{\textcolor[rgb]{0.73,0.13,0.13}{##1}}}
\@namedef{PY@tok@sb}{\def\PY@tc##1{\textcolor[rgb]{0.73,0.13,0.13}{##1}}}
\@namedef{PY@tok@sc}{\def\PY@tc##1{\textcolor[rgb]{0.73,0.13,0.13}{##1}}}
\@namedef{PY@tok@dl}{\def\PY@tc##1{\textcolor[rgb]{0.73,0.13,0.13}{##1}}}
\@namedef{PY@tok@s2}{\def\PY@tc##1{\textcolor[rgb]{0.73,0.13,0.13}{##1}}}
\@namedef{PY@tok@sh}{\def\PY@tc##1{\textcolor[rgb]{0.73,0.13,0.13}{##1}}}
\@namedef{PY@tok@s1}{\def\PY@tc##1{\textcolor[rgb]{0.73,0.13,0.13}{##1}}}
\@namedef{PY@tok@mb}{\def\PY@tc##1{\textcolor[rgb]{0.40,0.40,0.40}{##1}}}
\@namedef{PY@tok@mf}{\def\PY@tc##1{\textcolor[rgb]{0.40,0.40,0.40}{##1}}}
\@namedef{PY@tok@mh}{\def\PY@tc##1{\textcolor[rgb]{0.40,0.40,0.40}{##1}}}
\@namedef{PY@tok@mi}{\def\PY@tc##1{\textcolor[rgb]{0.40,0.40,0.40}{##1}}}
\@namedef{PY@tok@il}{\def\PY@tc##1{\textcolor[rgb]{0.40,0.40,0.40}{##1}}}
\@namedef{PY@tok@mo}{\def\PY@tc##1{\textcolor[rgb]{0.40,0.40,0.40}{##1}}}
\@namedef{PY@tok@ch}{\let\PY@it=\textit\def\PY@tc##1{\textcolor[rgb]{0.25,0.50,0.50}{##1}}}
\@namedef{PY@tok@cm}{\let\PY@it=\textit\def\PY@tc##1{\textcolor[rgb]{0.25,0.50,0.50}{##1}}}
\@namedef{PY@tok@cpf}{\let\PY@it=\textit\def\PY@tc##1{\textcolor[rgb]{0.25,0.50,0.50}{##1}}}
\@namedef{PY@tok@c1}{\let\PY@it=\textit\def\PY@tc##1{\textcolor[rgb]{0.25,0.50,0.50}{##1}}}
\@namedef{PY@tok@cs}{\let\PY@it=\textit\def\PY@tc##1{\textcolor[rgb]{0.25,0.50,0.50}{##1}}}

\def\PYZbs{\char`\\}
\def\PYZus{\char`\_}
\def\PYZob{\char`\{}
\def\PYZcb{\char`\}}
\def\PYZca{\char`\^}
\def\PYZam{\char`\&}
\def\PYZlt{\char`\<}
\def\PYZgt{\char`\>}
\def\PYZsh{\char`\#}
\def\PYZpc{\char`\%}
\def\PYZdl{\char`\$}
\def\PYZhy{\char`\-}
\def\PYZsq{\char`\'}
\def\PYZdq{\char`\"}
\def\PYZti{\char`\~}
% for compatibility with earlier versions
\def\PYZat{@}
\def\PYZlb{[}
\def\PYZrb{]}
\makeatother


    % For linebreaks inside Verbatim environment from package fancyvrb. 
    \makeatletter
        \newbox\Wrappedcontinuationbox 
        \newbox\Wrappedvisiblespacebox 
        \newcommand*\Wrappedvisiblespace {\textcolor{red}{\textvisiblespace}} 
        \newcommand*\Wrappedcontinuationsymbol {\textcolor{red}{\llap{\tiny$\m@th\hookrightarrow$}}} 
        \newcommand*\Wrappedcontinuationindent {3ex } 
        \newcommand*\Wrappedafterbreak {\kern\Wrappedcontinuationindent\copy\Wrappedcontinuationbox} 
        % Take advantage of the already applied Pygments mark-up to insert 
        % potential linebreaks for TeX processing. 
        %        {, <, #, %, $, ' and ": go to next line. 
        %        _, }, ^, &, >, - and ~: stay at end of broken line. 
        % Use of \textquotesingle for straight quote. 
        \newcommand*\Wrappedbreaksatspecials {% 
            \def\PYGZus{\discretionary{\char`\_}{\Wrappedafterbreak}{\char`\_}}% 
            \def\PYGZob{\discretionary{}{\Wrappedafterbreak\char`\{}{\char`\{}}% 
            \def\PYGZcb{\discretionary{\char`\}}{\Wrappedafterbreak}{\char`\}}}% 
            \def\PYGZca{\discretionary{\char`\^}{\Wrappedafterbreak}{\char`\^}}% 
            \def\PYGZam{\discretionary{\char`\&}{\Wrappedafterbreak}{\char`\&}}% 
            \def\PYGZlt{\discretionary{}{\Wrappedafterbreak\char`\<}{\char`\<}}% 
            \def\PYGZgt{\discretionary{\char`\>}{\Wrappedafterbreak}{\char`\>}}% 
            \def\PYGZsh{\discretionary{}{\Wrappedafterbreak\char`\#}{\char`\#}}% 
            \def\PYGZpc{\discretionary{}{\Wrappedafterbreak\char`\%}{\char`\%}}% 
            \def\PYGZdl{\discretionary{}{\Wrappedafterbreak\char`\$}{\char`\$}}% 
            \def\PYGZhy{\discretionary{\char`\-}{\Wrappedafterbreak}{\char`\-}}% 
            \def\PYGZsq{\discretionary{}{\Wrappedafterbreak\textquotesingle}{\textquotesingle}}% 
            \def\PYGZdq{\discretionary{}{\Wrappedafterbreak\char`\"}{\char`\"}}% 
            \def\PYGZti{\discretionary{\char`\~}{\Wrappedafterbreak}{\char`\~}}% 
        } 
        % Some characters . , ; ? ! / are not pygmentized. 
        % This macro makes them "active" and they will insert potential linebreaks 
        \newcommand*\Wrappedbreaksatpunct {% 
            \lccode`\~`\.\lowercase{\def~}{\discretionary{\hbox{\char`\.}}{\Wrappedafterbreak}{\hbox{\char`\.}}}% 
            \lccode`\~`\,\lowercase{\def~}{\discretionary{\hbox{\char`\,}}{\Wrappedafterbreak}{\hbox{\char`\,}}}% 
            \lccode`\~`\;\lowercase{\def~}{\discretionary{\hbox{\char`\;}}{\Wrappedafterbreak}{\hbox{\char`\;}}}% 
            \lccode`\~`\:\lowercase{\def~}{\discretionary{\hbox{\char`\:}}{\Wrappedafterbreak}{\hbox{\char`\:}}}% 
            \lccode`\~`\?\lowercase{\def~}{\discretionary{\hbox{\char`\?}}{\Wrappedafterbreak}{\hbox{\char`\?}}}% 
            \lccode`\~`\!\lowercase{\def~}{\discretionary{\hbox{\char`\!}}{\Wrappedafterbreak}{\hbox{\char`\!}}}% 
            \lccode`\~`\/\lowercase{\def~}{\discretionary{\hbox{\char`\/}}{\Wrappedafterbreak}{\hbox{\char`\/}}}% 
            \catcode`\.\active
            \catcode`\,\active 
            \catcode`\;\active
            \catcode`\:\active
            \catcode`\?\active
            \catcode`\!\active
            \catcode`\/\active 
            \lccode`\~`\~ 	
        }
    \makeatother

    \let\OriginalVerbatim=\Verbatim
    \makeatletter
    \renewcommand{\Verbatim}[1][1]{%
        %\parskip\z@skip
        \sbox\Wrappedcontinuationbox {\Wrappedcontinuationsymbol}%
        \sbox\Wrappedvisiblespacebox {\FV@SetupFont\Wrappedvisiblespace}%
        \def\FancyVerbFormatLine ##1{\hsize\linewidth
            \vtop{\raggedright\hyphenpenalty\z@\exhyphenpenalty\z@
                \doublehyphendemerits\z@\finalhyphendemerits\z@
                \strut ##1\strut}%
        }%
        % If the linebreak is at a space, the latter will be displayed as visible
        % space at end of first line, and a continuation symbol starts next line.
        % Stretch/shrink are however usually zero for typewriter font.
        \def\FV@Space {%
            \nobreak\hskip\z@ plus\fontdimen3\font minus\fontdimen4\font
            \discretionary{\copy\Wrappedvisiblespacebox}{\Wrappedafterbreak}
            {\kern\fontdimen2\font}%
        }%
        
        % Allow breaks at special characters using \PYG... macros.
        \Wrappedbreaksatspecials
        % Breaks at punctuation characters . , ; ? ! and / need catcode=\active 	
        \OriginalVerbatim[#1,codes*=\Wrappedbreaksatpunct]%
    }
    \makeatother

    % Exact colors from NB
    \definecolor{incolor}{HTML}{303F9F}
    \definecolor{outcolor}{HTML}{D84315}
    \definecolor{cellborder}{HTML}{CFCFCF}
    \definecolor{cellbackground}{HTML}{F7F7F7}
    
    % prompt
    \makeatletter
    \newcommand{\boxspacing}{\kern\kvtcb@left@rule\kern\kvtcb@boxsep}
    \makeatother
    \newcommand{\prompt}[4]{
        {\ttfamily\llap{{\color{#2}[#3]:\hspace{3pt}#4}}\vspace{-\baselineskip}}
    }
    

    
    % Prevent overflowing lines due to hard-to-break entities
    \sloppy 
    % Setup hyperref package
    \hypersetup{
      breaklinks=true,  % so long urls are correctly broken across lines
      colorlinks=true,
      urlcolor=urlcolor,
      linkcolor=linkcolor,
      citecolor=citecolor,
      }
    % Slightly bigger margins than the latex defaults
    
    \geometry{verbose,tmargin=1in,bmargin=1in,lmargin=1in,rmargin=1in}
    
    

\begin{document}
    
    \maketitle
    
    

    
    \begin{tcolorbox}[breakable, size=fbox, boxrule=1pt, pad at break*=1mm,colback=cellbackground, colframe=cellborder]
\prompt{In}{incolor}{1}{\boxspacing}
\begin{Verbatim}[commandchars=\\\{\}]
\PY{c+c1}{\PYZsh{}Imports and Setup}
\PY{k+kn}{from} \PY{n+nn}{IPython}\PY{n+nn}{.}\PY{n+nn}{display} \PY{k+kn}{import} \PY{n}{display}\PY{p}{,} \PY{n}{Math}\PY{p}{,} \PY{n}{Latex}

\PY{k+kn}{import} \PY{n+nn}{pandas} \PY{k}{as} \PY{n+nn}{pd}
\PY{k+kn}{import} \PY{n+nn}{yfinance} \PY{k}{as} \PY{n+nn}{yf}
\PY{k+kn}{import} \PY{n+nn}{matplotlib}\PY{n+nn}{.}\PY{n+nn}{pyplot} \PY{k}{as} \PY{n+nn}{plt}
\PY{k+kn}{import} \PY{n+nn}{random}
\PY{k+kn}{import} \PY{n+nn}{threading}
\PY{k+kn}{import} \PY{n+nn}{numpy} \PY{k}{as} \PY{n+nn}{np}
\end{Verbatim}
\end{tcolorbox}

    \hypertarget{group-assignment}{%
\subsection{Group Assignment}\label{group-assignment}}

\hypertarget{team-number-18}{%
\subsubsection{Team Number: 18}\label{team-number-18}}

\hypertarget{team-member-names-daniel-kim-kitty-cai-andre-slavescu}{%
\subsubsection{Team Member Names: Daniel Kim, Kitty Cai, Andre
Slavescu}\label{team-member-names-daniel-kim-kitty-cai-andre-slavescu}}

\hypertarget{team-strategy-chosen-safe}{%
\subsubsection{Team Strategy Chosen:
Safe}\label{team-strategy-chosen-safe}}

    \hypertarget{group-assignment}{%
\subsection{Group Assignment:}\label{group-assignment}}

In this assignment, we will be analyzing various factors and ultimately
utilizing these factors to formulate a dynamic portfolio that is safe.
In other words, we will be attempting to create a portfolio whose value
stays as close to the initial investment value as possible.

    \begin{tcolorbox}[breakable, size=fbox, boxrule=1pt, pad at break*=1mm,colback=cellbackground, colframe=cellborder]
\prompt{In}{incolor}{2}{\boxspacing}
\begin{Verbatim}[commandchars=\\\{\}]
\PY{c+c1}{\PYZsh{} Reading in the stocks from the CSV file}
\PY{n}{stocks} \PY{o}{=} \PY{n}{pd}\PY{o}{.}\PY{n}{read\PYZus{}csv}\PY{p}{(}\PY{l+s+s2}{\PYZdq{}}\PY{l+s+s2}{Tickers.csv}\PY{l+s+s2}{\PYZdq{}}\PY{p}{,} \PY{n}{header} \PY{o}{=} \PY{k+kc}{None}\PY{p}{)}
\PY{n}{stocks}\PY{o}{.}\PY{n}{columns} \PY{o}{=} \PY{p}{[}\PY{l+s+s1}{\PYZsq{}}\PY{l+s+s1}{ticker}\PY{l+s+s1}{\PYZsq{}}\PY{p}{]}

\PY{n}{stocklist} \PY{o}{=} \PY{n}{stocks}\PY{o}{.}\PY{n}{ticker}\PY{o}{.}\PY{n}{tolist}\PY{p}{(}\PY{p}{)}

\PY{c+c1}{\PYZsh{}Check if the ticker is a valid US listed stock}
\PY{n}{US\PYZus{}Tickers} \PY{o}{=} \PY{p}{[}\PY{p}{]}
\PY{k}{def} \PY{n+nf}{currency\PYZus{}checker}\PY{p}{(}\PY{n}{stockstr}\PY{p}{)}\PY{p}{:}
    \PY{n}{stock} \PY{o}{=} \PY{n}{yf}\PY{o}{.}\PY{n}{Ticker}\PY{p}{(}\PY{n}{stockstr}\PY{p}{)}
    \PY{k}{if} \PY{p}{(}\PY{l+s+s1}{\PYZsq{}}\PY{l+s+s1}{currency}\PY{l+s+s1}{\PYZsq{}} \PY{o+ow}{in} \PY{n}{stock}\PY{o}{.}\PY{n}{info}\PY{p}{)} \PY{o+ow}{and} \PY{p}{(}\PY{n}{stock}\PY{o}{.}\PY{n}{info}\PY{p}{[}\PY{l+s+s1}{\PYZsq{}}\PY{l+s+s1}{currency}\PY{l+s+s1}{\PYZsq{}}\PY{p}{]} \PY{o}{==} \PY{l+s+s1}{\PYZsq{}}\PY{l+s+s1}{USD}\PY{l+s+s1}{\PYZsq{}}\PY{p}{)} \PY{o+ow}{and} \PY{p}{(}\PY{n}{stock}\PY{o}{.}\PY{n}{info}\PY{p}{[}\PY{l+s+s1}{\PYZsq{}}\PY{l+s+s1}{market}\PY{l+s+s1}{\PYZsq{}}\PY{p}{]} \PY{o}{==} \PY{l+s+s1}{\PYZsq{}}\PY{l+s+s1}{us\PYZus{}market}\PY{l+s+s1}{\PYZsq{}}\PY{p}{)}\PY{p}{:}
        \PY{n}{US\PYZus{}Tickers}\PY{o}{.}\PY{n}{append}\PY{p}{(}\PY{n}{stockstr}\PY{p}{)}

\PY{c+c1}{\PYZsh{} Threading:}
\PY{n}{threads} \PY{o}{=} \PY{p}{[}\PY{p}{]}

\PY{k}{for} \PY{n}{i} \PY{o+ow}{in} \PY{n+nb}{range}\PY{p}{(}\PY{n+nb}{len}\PY{p}{(}\PY{n}{stocklist}\PY{p}{)}\PY{p}{)}\PY{p}{:}
    \PY{n}{stock} \PY{o}{=} \PY{n}{stocklist}\PY{p}{[}\PY{n}{i}\PY{p}{]}
    \PY{n}{t} \PY{o}{=} \PY{n}{threading}\PY{o}{.}\PY{n}{Thread}\PY{p}{(}\PY{n}{target} \PY{o}{=} \PY{n}{currency\PYZus{}checker}\PY{p}{,} \PY{n}{args} \PY{o}{=} \PY{p}{(}\PY{n}{stock}\PY{p}{,}\PY{p}{)}\PY{p}{)}
    \PY{n}{threads}\PY{o}{.}\PY{n}{append}\PY{p}{(}\PY{n}{t}\PY{p}{)}
    \PY{n}{t}\PY{o}{.}\PY{n}{start}\PY{p}{(}\PY{p}{)}

\PY{k}{for} \PY{n}{thread} \PY{o+ow}{in} \PY{n}{threads}\PY{p}{:}
    \PY{n}{thread}\PY{o}{.}\PY{n}{join}\PY{p}{(}\PY{p}{)}
\end{Verbatim}
\end{tcolorbox}

    \begin{tcolorbox}[breakable, size=fbox, boxrule=1pt, pad at break*=1mm,colback=cellbackground, colframe=cellborder]
\prompt{In}{incolor}{3}{\boxspacing}
\begin{Verbatim}[commandchars=\\\{\}]
\PY{c+c1}{\PYZsh{}Make sure calculations are only based on the time interval of of July 02, 2021 to October 22, 2021}
\PY{n}{start\PYZus{}date} \PY{o}{=} \PY{l+s+s1}{\PYZsq{}}\PY{l+s+s1}{2021\PYZhy{}07\PYZhy{}02}\PY{l+s+s1}{\PYZsq{}}
\PY{n}{end\PYZus{}date} \PY{o}{=} \PY{l+s+s1}{\PYZsq{}}\PY{l+s+s1}{2021\PYZhy{}10\PYZhy{}22}\PY{l+s+s1}{\PYZsq{}}

\PY{c+c1}{\PYZsh{}Making a list of stocks that is the list of valid stocks according to the guidelines in the Document}
\PY{n}{list\PYZus{}of\PYZus{}valid\PYZus{}stocks} \PY{o}{=} \PY{p}{[}\PY{p}{]}
\PY{n}{threads} \PY{o}{=} \PY{p}{[}\PY{p}{]}

\PY{c+c1}{\PYZsh{}Checks if the average daily volume of the stock is greater or equal to 10 000 shares}
\PY{k}{def} \PY{n+nf}{volume\PYZus{}checker}\PY{p}{(}\PY{n}{stock\PYZus{}str}\PY{p}{)}\PY{p}{:}
    \PY{n}{volume} \PY{o}{=} \PY{n}{yf}\PY{o}{.}\PY{n}{Ticker}\PY{p}{(}\PY{n}{stock\PYZus{}str}\PY{p}{)}\PY{o}{.}\PY{n}{history}\PY{p}{(}\PY{n}{start} \PY{o}{=} \PY{n}{start\PYZus{}date}\PY{p}{,} \PY{n}{close} \PY{o}{=} \PY{n}{end\PYZus{}date}\PY{p}{)}\PY{o}{.}\PY{n}{Volume}\PY{o}{.}\PY{n}{mean}\PY{p}{(}\PY{p}{)}
    \PY{k}{if} \PY{n}{volume} \PY{o}{\PYZgt{}}\PY{o}{=} \PY{l+m+mi}{10000}\PY{p}{:}
        \PY{n}{list\PYZus{}of\PYZus{}valid\PYZus{}stocks}\PY{o}{.}\PY{n}{append}\PY{p}{(}\PY{n}{stock\PYZus{}str}\PY{p}{)}

\PY{c+c1}{\PYZsh{}Checks for each stock in the stock list of US listed stocks}
\PY{k}{for} \PY{n}{i} \PY{o+ow}{in} \PY{n+nb}{range}\PY{p}{(}\PY{n+nb}{len}\PY{p}{(}\PY{n}{US\PYZus{}Tickers}\PY{p}{)}\PY{p}{)}\PY{p}{:}
    \PY{n}{stock} \PY{o}{=} \PY{n}{US\PYZus{}Tickers}\PY{p}{[}\PY{n}{i}\PY{p}{]}
    \PY{n}{t} \PY{o}{=} \PY{n}{threading}\PY{o}{.}\PY{n}{Thread}\PY{p}{(}\PY{n}{target} \PY{o}{=} \PY{n}{volume\PYZus{}checker}\PY{p}{,} \PY{n}{args} \PY{o}{=} \PY{p}{(}\PY{n}{stock}\PY{p}{,}\PY{p}{)}\PY{p}{)}
    \PY{n}{threads}\PY{o}{.}\PY{n}{append}\PY{p}{(}\PY{n}{t}\PY{p}{)}
    \PY{n}{t}\PY{o}{.}\PY{n}{start}\PY{p}{(}\PY{p}{)}

\PY{k}{for} \PY{n}{thread} \PY{o+ow}{in} \PY{n}{threads}\PY{p}{:}
    \PY{n}{thread}\PY{o}{.}\PY{n}{join}\PY{p}{(}\PY{p}{)}
\end{Verbatim}
\end{tcolorbox}

    \begin{tcolorbox}[breakable, size=fbox, boxrule=1pt, pad at break*=1mm,colback=cellbackground, colframe=cellborder]
\prompt{In}{incolor}{4}{\boxspacing}
\begin{Verbatim}[commandchars=\\\{\}]
\PY{c+c1}{\PYZsh{} Removes Duplicates}
\PY{n}{list\PYZus{}of\PYZus{}valid\PYZus{}stocks} \PY{o}{=} \PY{n+nb}{list}\PY{p}{(}\PY{n+nb}{dict}\PY{o}{.}\PY{n}{fromkeys}\PY{p}{(}\PY{n}{list\PYZus{}of\PYZus{}valid\PYZus{}stocks}\PY{p}{)}\PY{p}{)}
\end{Verbatim}
\end{tcolorbox}

    \hypertarget{filtering-discussion}{%
\subsection{Filtering Discussion:}\label{filtering-discussion}}

Our portfolio's filtering process is based on the fact that we want to
aim for a safe portfolio, which does not fluctuate too much. In previous
assignments, we noticed that when there were more stocks in a portfolio,
if one of the stocks rises or falls sharply, it does not move the
portfolio as much. This is because the stock will only be a portion of
the portfolio and if there are more stocks, an individual stock will
make up an even smaller proportion of the portfolio. Since it is only a
small proportion, when it fluctuates, it will not influence the entire
portfolio. Thus, in order to prevent and divert the risk of an
individual stock, we decided to diversify our portfolio by aiming for a
portfolio of 20 stocks (the maximum allowed). In the case where there
are less than 20 valid stocks after the filtering, they will all be
taken into consideration however risky stocks will be weighted less. It
is not worth it to invest in something risky which may negatively impact
our chances of having a safe portfolio. Investing in as many stocks as
possible allows the portfolio to be less susceptible to the risk of an
individual stock. In addition, investing in many stocks gives higher
chances of diversification. This could mean that our program will have
higher chances of picking stocks from different industries (especially
through the comparisons of correlations.)

    \hypertarget{standard-deviation}{%
\subsection{Standard Deviation:}\label{standard-deviation}}

In order to create a portfolio that has low returns, one crucial factor
to consider is the standard deviation of the stock. The formula for the
standard deviation is given by: \begin{align*}
\sigma_X=\sqrt{\frac{\sum(x_i-\overline{X})}{N}}
\end{align*} In finance, standard deviation is a measure of the total
risk associated with the expected return. Consequently, less risky
stocks have lower standard deviation values compared to their riskier
counterparts as the lower standard deviation value indicates that the
security possesses a lower trading range and has fewer spikes in its
prices. Since we are attempting to make a safe portfolio, we want to
choose stocks that have low standard deviation values. By doing so, we
are choosing stocks that have a lower trading range and makes the
portfolio less susceptible to large-scale losses or gains. This follows
the fundamental relationship in finance between risk and return. By
minimizing our risk, we are decreasing the opportunity for our portfolio
to undergo large changes in its value. Conversely, by adding in stocks
that have low risk in their expected returns, our portfolio will be
safer and will thereby have less fluctuations in the value of our
portfolio.

Consequently, in this section, we will be calculating the standard
deviation of the returns of each of the stocks that are in the CSV file.
Afterwards, we will be finding which stocks have the lowest standard
deviation values, which indicates that they are the safest stocks. From
these results, we will be assigning points to the stocks that have the
lowest standard deviation values (with the stocks with the lowest
standard deviation values earning the most points).

    \begin{tcolorbox}[breakable, size=fbox, boxrule=1pt, pad at break*=1mm,colback=cellbackground, colframe=cellborder]
\prompt{In}{incolor}{5}{\boxspacing}
\begin{Verbatim}[commandchars=\\\{\}]
\PY{c+c1}{\PYZsh{}Return a dataframe with all of the valid stocks and their closing prices}
\PY{n}{stock\PYZus{}closing} \PY{o}{=} \PY{n}{pd}\PY{o}{.}\PY{n}{DataFrame}\PY{p}{(}\PY{p}{)}

\PY{k}{for} \PY{n}{ticker} \PY{o+ow}{in} \PY{n}{list\PYZus{}of\PYZus{}valid\PYZus{}stocks}\PY{p}{:}
    \PY{c+c1}{\PYZsh{}Get the ticker}
    \PY{n}{x} \PY{o}{=} \PY{n}{yf}\PY{o}{.}\PY{n}{Ticker}\PY{p}{(}\PY{n}{ticker}\PY{p}{)}
    \PY{c+c1}{\PYZsh{}Get ticker history}
    \PY{n}{y} \PY{o}{=} \PY{n}{x}\PY{o}{.}\PY{n}{history}\PY{p}{(}\PY{n}{start} \PY{o}{=} \PY{n}{start\PYZus{}date}\PY{p}{,} \PY{n}{end} \PY{o}{=} \PY{n}{end\PYZus{}date}\PY{p}{)}
    \PY{c+c1}{\PYZsh{}Return the closing price of the ticker during the period of time}
    \PY{n}{stock\PYZus{}closing}\PY{p}{[}\PY{n}{ticker}\PY{p}{]} \PY{o}{=} \PY{n}{y}\PY{p}{[}\PY{l+s+s1}{\PYZsq{}}\PY{l+s+s1}{Close}\PY{l+s+s1}{\PYZsq{}}\PY{p}{]}

\PY{n}{stock\PYZus{}closing}
\end{Verbatim}
\end{tcolorbox}

            \begin{tcolorbox}[breakable, size=fbox, boxrule=.5pt, pad at break*=1mm, opacityfill=0]
\prompt{Out}{outcolor}{5}{\boxspacing}
\begin{Verbatim}[commandchars=\\\{\}]
                   BLK        ABBV         COF         GOOG         CL  \textbackslash{}
Date
2021-07-02  889.747803  112.535881  155.646042  2574.379883  80.773155
2021-07-06  888.801941  113.083076  153.767212  2595.419922  81.040092
2021-07-07  897.026062  114.079750  153.312332  2601.550049  81.722275
2021-07-08  872.692261  113.688896  150.761093  2583.540039  81.405899
2021-07-09  897.394470  113.913635  158.760941  2591.489990  81.494873
{\ldots}                {\ldots}         {\ldots}         {\ldots}          {\ldots}        {\ldots}
2021-10-15  907.260010  109.330002  167.737122  2833.500000  75.791153
2021-10-18  896.309998  107.430000  168.514145  2859.209961  74.826988
2021-10-19  901.650024  107.449997  169.141724  2876.439941  74.330002
2021-10-20  902.929993  108.410004  172.588531  2848.300049  75.099998
2021-10-21  903.719971  108.760002  165.057388  2855.610107  74.889999

                   GM        CVS       CSCO          C        KMI  {\ldots}  \textbackslash{}
Date                                                               {\ldots}
2021-07-02  58.959999  81.460464  53.180737  69.385284  18.005268  {\ldots}
2021-07-06  57.459999  80.225159  52.624496  67.217293  17.879221  {\ldots}
2021-07-07  56.590000  80.383278  52.902615  66.941368  17.743477  {\ldots}
2021-07-08  56.060001  79.276451  52.902615  65.758842  17.617432  {\ldots}
2021-07-09  58.759998  80.363510  53.379398  67.453804  18.102228  {\ldots}
{\ldots}               {\ldots}        {\ldots}        {\ldots}        {\ldots}        {\ldots}  {\ldots}
2021-10-15  58.000000  85.344101  55.250000  71.769859  18.172726  {\ldots}
2021-10-18  56.889999  83.991959  55.189999  71.094749  18.182570  {\ldots}
2021-10-19  56.849998  84.399590  55.740002  71.243675  18.162884  {\ldots}
2021-10-20  57.669998  86.050003  56.200001  71.303238  18.359770  {\ldots}
2021-10-21  58.410000  86.860001  55.689999  70.002663  17.257200  {\ldots}

                   ACN          T        USB         SPG        NEE  \textbackslash{}
Date
2021-07-02  303.082123  28.158052  57.131699  128.185181  73.615631
2021-07-06  303.638855  27.955755  55.583572  126.217354  74.458031
2021-07-07  307.824585  27.869055  55.563728  124.387978  74.656250
2021-07-08  307.367279  27.643435  54.620960  123.695770  74.458031
2021-07-09  310.817230  27.908295  56.536266  128.244507  74.319283
{\ldots}                {\ldots}        {\ldots}        {\ldots}         {\ldots}        {\ldots}
2021-10-15  341.820007  25.700001  60.230000  140.220001  81.309868
2021-10-18  343.179993  25.330000  60.580002  144.529999  80.632866
2021-10-19  349.739990  25.590000  61.349998  142.839996  81.668282
2021-10-20  345.799988  25.910000  62.580002  144.960007  83.549950
2021-10-21  347.119995  25.760000  62.180000  143.960007  83.440430

                   KO        OXY        QCOM        SBUX         PM
Date
2021-07-02  53.774158  32.566921  141.919052  114.033791  99.076950
2021-07-06  53.476406  30.497753  140.535492  114.787613  97.447075
2021-07-07  53.913109  29.468164  139.321152  116.186119  97.585365
2021-07-08  53.724533  29.648092  137.648926  115.045486  97.170486
2021-07-09  54.052059  30.227859  140.774368  116.513443  98.187927
{\ldots}               {\ldots}        {\ldots}         {\ldots}         {\ldots}        {\ldots}
2021-10-15  54.480000  31.660000  130.199997  110.971504  98.370003
2021-10-18  53.939999  32.930000  130.119995  112.883270  97.449997
2021-10-19  54.150002  32.939999  132.500000  113.002747  95.790001
2021-10-20  54.630001  33.360001  132.160004  113.371162  97.019997
2021-10-21  54.349998  32.799999  133.050003  113.948677  96.519997

[78 rows x 55 columns]
\end{Verbatim}
\end{tcolorbox}
        
    \begin{tcolorbox}[breakable, size=fbox, boxrule=1pt, pad at break*=1mm,colback=cellbackground, colframe=cellborder]
\prompt{In}{incolor}{6}{\boxspacing}
\begin{Verbatim}[commandchars=\\\{\}]
\PY{c+c1}{\PYZsh{} Getting the Daily Returns of each stock that is a valid stock.}
\PY{n}{returns} \PY{o}{=} \PY{l+m+mi}{100} \PY{o}{*} \PY{n}{stock\PYZus{}closing}\PY{o}{.}\PY{n}{resample}\PY{p}{(}\PY{l+s+s1}{\PYZsq{}}\PY{l+s+s1}{D}\PY{l+s+s1}{\PYZsq{}}\PY{p}{)}\PY{o}{.}\PY{n}{first}\PY{p}{(}\PY{p}{)}\PY{o}{.}\PY{n}{pct\PYZus{}change}\PY{p}{(}\PY{p}{)}
\PY{n}{returns} \PY{o}{=} \PY{n}{returns}\PY{o}{.}\PY{n}{iloc}\PY{p}{[}\PY{l+m+mi}{1}\PY{p}{:}\PY{p}{]}
\PY{n}{returns}\PY{o}{.}\PY{n}{dropna}\PY{p}{(}\PY{n}{axis}\PY{o}{=}\PY{l+m+mi}{1}\PY{p}{,} \PY{n}{how}\PY{o}{=}\PY{l+s+s2}{\PYZdq{}}\PY{l+s+s2}{all}\PY{l+s+s2}{\PYZdq{}}\PY{p}{,} \PY{n}{inplace} \PY{o}{=} \PY{k+kc}{True}\PY{p}{)}
\PY{c+c1}{\PYZsh{} Calculate the standard deviation and making a dataframe of the standard deviations of each of the stocks.}
\PY{n}{std} \PY{o}{=} \PY{n}{pd}\PY{o}{.}\PY{n}{DataFrame}\PY{p}{(}\PY{n}{returns}\PY{o}{.}\PY{n}{std}\PY{p}{(}\PY{p}{)}\PY{p}{)}
\PY{n}{std} \PY{o}{=} \PY{n}{std}\PY{o}{.}\PY{n}{reset\PYZus{}index}\PY{p}{(}\PY{p}{)}
\PY{n}{std} \PY{o}{=} \PY{n}{std}\PY{o}{.}\PY{n}{rename}\PY{p}{(}\PY{n}{columns}\PY{o}{=}\PY{p}{\PYZob{}}\PY{l+s+s2}{\PYZdq{}}\PY{l+s+s2}{index}\PY{l+s+s2}{\PYZdq{}}\PY{p}{:} \PY{l+s+s2}{\PYZdq{}}\PY{l+s+s2}{Stocks}\PY{l+s+s2}{\PYZdq{}}\PY{p}{,} \PY{l+m+mi}{0}\PY{p}{:} \PY{l+s+s2}{\PYZdq{}}\PY{l+s+s2}{Standard Deviation}\PY{l+s+s2}{\PYZdq{}}\PY{p}{\PYZcb{}}\PY{p}{)}
\PY{n}{std}\PY{o}{.}\PY{n}{dropna}\PY{p}{(}\PY{n}{inplace} \PY{o}{=} \PY{k+kc}{True}\PY{p}{)}
\PY{c+c1}{\PYZsh{} Sorting in increasing order of the standard deviations}
\PY{n}{std} \PY{o}{=} \PY{n}{std}\PY{o}{.}\PY{n}{sort\PYZus{}values}\PY{p}{(}\PY{n}{by} \PY{o}{=} \PY{l+s+s1}{\PYZsq{}}\PY{l+s+s1}{Standard Deviation}\PY{l+s+s1}{\PYZsq{}}\PY{p}{)}
\PY{n}{std} \PY{o}{=} \PY{n}{std}\PY{o}{.}\PY{n}{reset\PYZus{}index}\PY{p}{(}\PY{n}{drop} \PY{o}{=} \PY{k+kc}{True}\PY{p}{)}
\PY{n}{std}\PY{o}{.}\PY{n}{head}\PY{p}{(}\PY{p}{)}
\end{Verbatim}
\end{tcolorbox}

            \begin{tcolorbox}[breakable, size=fbox, boxrule=.5pt, pad at break*=1mm, opacityfill=0]
\prompt{Out}{outcolor}{6}{\boxspacing}
\begin{Verbatim}[commandchars=\\\{\}]
  Stocks  Standard Deviation
0    MON            0.348784
1     PG            0.609157
2     KO            0.631825
3    PEP            0.652622
4     SO            0.654017
\end{Verbatim}
\end{tcolorbox}
        
    \hypertarget{create-a-weighted-portfolio}{%
\subsection{Create a Weighted
Portfolio}\label{create-a-weighted-portfolio}}

We decided to use a point system to create our portfolio. Our portfolio
is dependent on 4 factors: standard deviation, expected returns, beta,
and correlation.

First we calculated the standard deviation of all of the valid stocks
and ordered them from smallest standard deviation to largest. Then we
give the smallest standard deviation the most points, 20 points. This is
because we are aiming for a portfolio with the largest amount of stocks
within the limit of 20. From there, each following ticker will receive
one less point and after the first twenty stocks, all other tickers will
not receive any points.

Similarly, this was done for the expected returns and betas. The
expected returns were listed from smallest to largest because we our
hoping our portfolio will have nearly \$0 of return after the period.
The lowest expected returns received 20 points and so on.

This point system was concluded after a long discussion, however we did
not want to completely weigh our portfolio depending on the point
system. Since each stock must make up a minimum of (100/(2n))\%, where n
is the number of stocks we choose for our stocks. We calculate how much
each stock should minimally weigh and then calculate a second percentage
of how much more we should make each stock weigh. This is done so to
prevent an individual stock from weighing too much and skewing the data.

The fourth factor we considered is correlation however this was done
differently from the other three factors. Correlation was calculated by
comparing a stock to the rest of the portfolio generated without the
stock. If the correlation is below 0.5, then the stock receives 10
points. This is because it is nearly impossible for the correlation to
be negative for a stock in comparison to the generated portfolio and we
realized that 0.5 is a very valid low correlation in comparison.

Note: \textbf{Addresses the 35\% maximum limitation}

After lots of calculation, we concluded that it is impossible for a
stock to weigh more than 35\% so that was not a limitation we put into
our code. This is because for each of the three indicators, 210 points
are given out.

(1 + 2 + 3 + 4 + 5 + 6 + 7 + 8 + 9 + 10 + 11 + 12 + 13 + 14 + 15 + 16 +
17 + 18 + 19 + 20) = (20 + 1) * 10 = 21 * 10 = 210

Say there is a stock which gets the maximum of 20 points every time and
it also receives the 10 points from correlation. It would have a total
of (20 + 20 + 20 + 10 = 40 + 30 = 70) 70 points. Overall, there should
be a sum of approximately 210 * 3 = 630 points although there may be
more or less in the portfolio depending on how many points each stock
gets and if there are more occurrences of correlation or less. 70/630 =
1/9 = 0.1111111, this is approximately 11.111\%. After adding the base
weighting, there is no way it would ever be above 35\% - not even close.

    \begin{tcolorbox}[breakable, size=fbox, boxrule=1pt, pad at break*=1mm,colback=cellbackground, colframe=cellborder]
\prompt{In}{incolor}{7}{\boxspacing}
\begin{Verbatim}[commandchars=\\\{\}]
\PY{c+c1}{\PYZsh{}Address points for standard deviation}
\PY{c+c1}{\PYZsh{}Loop through all indices in the condensed list and produce a point score for each stock}
\PY{c+c1}{\PYZsh{}A stock recieves more points for having a low standard deviation because this indicates that it did not fluctuate as much}
\PY{c+c1}{\PYZsh{}The goal of the portfolio is to be safe and have a return close to \PYZdl{}0 as possible so stocks with low standard deviation will recieve more points}
\PY{n}{std\PYZus{}points} \PY{o}{=} \PY{p}{[}\PY{p}{]}
\PY{k}{for} \PY{n}{i} \PY{o+ow}{in} \PY{n+nb}{range} \PY{p}{(}\PY{n+nb}{len}\PY{p}{(}\PY{n}{std}\PY{p}{)}\PY{p}{)}\PY{p}{:}
    \PY{k}{if} \PY{n}{i} \PY{o}{\PYZlt{}} \PY{l+m+mi}{20}\PY{p}{:}  
        \PY{n}{std\PYZus{}points}\PY{o}{.}\PY{n}{append}\PY{p}{(}\PY{l+m+mi}{20}\PY{o}{\PYZhy{}}\PY{n}{i}\PY{p}{)}              
    \PY{k}{else}\PY{p}{:}
        \PY{n}{std\PYZus{}points}\PY{o}{.}\PY{n}{append}\PY{p}{(}\PY{l+m+mi}{0}\PY{p}{)}
\PY{c+c1}{\PYZsh{}print(std\PYZus{}points)}
\end{Verbatim}
\end{tcolorbox}

    \begin{tcolorbox}[breakable, size=fbox, boxrule=1pt, pad at break*=1mm,colback=cellbackground, colframe=cellborder]
\prompt{In}{incolor}{8}{\boxspacing}
\begin{Verbatim}[commandchars=\\\{\}]
\PY{c+c1}{\PYZsh{} Initialize points column for standard deviation}
\PY{n}{std}\PY{p}{[}\PY{l+s+s1}{\PYZsq{}}\PY{l+s+s1}{PointsS}\PY{l+s+s1}{\PYZsq{}}\PY{p}{]} \PY{o}{=} \PY{n}{std\PYZus{}points}
\PY{n}{std}\PY{o}{.}\PY{n}{head}\PY{p}{(}\PY{p}{)}
\end{Verbatim}
\end{tcolorbox}

            \begin{tcolorbox}[breakable, size=fbox, boxrule=.5pt, pad at break*=1mm, opacityfill=0]
\prompt{Out}{outcolor}{8}{\boxspacing}
\begin{Verbatim}[commandchars=\\\{\}]
  Stocks  Standard Deviation  PointsS
0    MON            0.348784       20
1     PG            0.609157       19
2     KO            0.631825       18
3    PEP            0.652622       17
4     SO            0.654017       16
\end{Verbatim}
\end{tcolorbox}
        
    \hypertarget{standard-deviation-discussion}{%
\subsection{Standard Deviation
Discussion:}\label{standard-deviation-discussion}}

In our final dataframe named std, we have arranged the stocks in a
manner such that the stock with lowest standard deviation values, hence
the least riskiest stock, is first and it increases in ascending order
of standard deviation values. The first in our dataframe earned the most
points as it is the safest and we would like to take this into
consideration later on when we determine the weighting of the different
stocks in our portfolio.

To further solidfy the idea that stocks with lower standard deviations
have smaller fluctuations in their returns and are thereby safer, we
will plot a graph of the daily returns of the stock with the lowest
standard deviation value and the stock with the highest standard
deviation value and compare the graphs with one another and observe
important observations.

    \begin{tcolorbox}[breakable, size=fbox, boxrule=1pt, pad at break*=1mm,colback=cellbackground, colframe=cellborder]
\prompt{In}{incolor}{9}{\boxspacing}
\begin{Verbatim}[commandchars=\\\{\}]
\PY{c+c1}{\PYZsh{} Getting the two stocks:}
\PY{n}{lowest\PYZus{}std\PYZus{}stock} \PY{o}{=} \PY{n}{yf}\PY{o}{.}\PY{n}{Ticker}\PY{p}{(}\PY{n+nb}{str}\PY{p}{(}\PY{n}{std}\PY{p}{[}\PY{l+s+s2}{\PYZdq{}}\PY{l+s+s2}{Stocks}\PY{l+s+s2}{\PYZdq{}}\PY{p}{]}\PY{o}{.}\PY{n}{iloc}\PY{p}{[}\PY{l+m+mi}{0}\PY{p}{]}\PY{p}{)}\PY{p}{)}
\PY{n}{highest\PYZus{}std\PYZus{}stock} \PY{o}{=} \PY{n}{yf}\PY{o}{.}\PY{n}{Ticker}\PY{p}{(}\PY{n+nb}{str}\PY{p}{(}\PY{n}{std}\PY{p}{[}\PY{l+s+s2}{\PYZdq{}}\PY{l+s+s2}{Stocks}\PY{l+s+s2}{\PYZdq{}}\PY{p}{]}\PY{o}{.}\PY{n}{iloc}\PY{p}{[}\PY{o}{\PYZhy{}}\PY{l+m+mi}{1}\PY{p}{]}\PY{p}{)}\PY{p}{)}

\PY{c+c1}{\PYZsh{} Getting their returns from the historical data}
\PY{n}{lowest\PYZus{}std\PYZus{}stock} \PY{o}{=} \PY{n}{yf}\PY{o}{.}\PY{n}{Ticker}\PY{p}{(}\PY{n+nb}{str}\PY{p}{(}\PY{n}{std}\PY{p}{[}\PY{l+s+s2}{\PYZdq{}}\PY{l+s+s2}{Stocks}\PY{l+s+s2}{\PYZdq{}}\PY{p}{]}\PY{o}{.}\PY{n}{iloc}\PY{p}{[}\PY{l+m+mi}{0}\PY{p}{]}\PY{p}{)}\PY{p}{)}
\PY{n}{highest\PYZus{}std\PYZus{}stock} \PY{o}{=} \PY{n}{yf}\PY{o}{.}\PY{n}{Ticker}\PY{p}{(}\PY{n+nb}{str}\PY{p}{(}\PY{n}{std}\PY{p}{[}\PY{l+s+s2}{\PYZdq{}}\PY{l+s+s2}{Stocks}\PY{l+s+s2}{\PYZdq{}}\PY{p}{]}\PY{o}{.}\PY{n}{iloc}\PY{p}{[}\PY{o}{\PYZhy{}}\PY{l+m+mi}{1}\PY{p}{]}\PY{p}{)}\PY{p}{)}
\PY{n}{lowest\PYZus{}std\PYZus{}stock\PYZus{}hist\PYZus{}returns\PYZus{}df} \PY{o}{=} \PY{n}{pd}\PY{o}{.}\PY{n}{DataFrame}\PY{p}{(}\PY{l+m+mi}{100} \PY{o}{*} \PY{n}{lowest\PYZus{}std\PYZus{}stock}\PY{o}{.}\PY{n}{history}\PY{p}{(}\PY{n}{start} \PY{o}{=} \PY{n}{start\PYZus{}date}\PY{p}{,} \PY{n}{end} \PY{o}{=} \PY{n}{end\PYZus{}date}\PY{p}{)}\PY{o}{.}\PY{n}{Close}\PY{o}{.}\PY{n}{pct\PYZus{}change}\PY{p}{(}\PY{p}{)}\PY{o}{.}\PY{n}{iloc}\PY{p}{[}\PY{l+m+mi}{1}\PY{p}{:}\PY{p}{]}\PY{p}{)}
\PY{n}{highest\PYZus{}std\PYZus{}stock\PYZus{}hist\PYZus{}returns\PYZus{}df} \PY{o}{=} \PY{n}{pd}\PY{o}{.}\PY{n}{DataFrame}\PY{p}{(}\PY{l+m+mi}{100} \PY{o}{*} \PY{n}{highest\PYZus{}std\PYZus{}stock}\PY{o}{.}\PY{n}{history}\PY{p}{(}\PY{n}{start} \PY{o}{=} \PY{n}{start\PYZus{}date}\PY{p}{,} \PY{n}{end} \PY{o}{=} \PY{n}{end\PYZus{}date}\PY{p}{)}\PY{o}{.}\PY{n}{Close}\PY{o}{.}\PY{n}{pct\PYZus{}change}\PY{p}{(}\PY{p}{)}\PY{o}{.}\PY{n}{iloc}\PY{p}{[}\PY{l+m+mi}{1}\PY{p}{:}\PY{p}{]}\PY{p}{)}

\PY{c+c1}{\PYZsh{} For the subplot:}
\PY{n}{fig}\PY{p}{,} \PY{p}{(}\PY{p}{(}\PY{n}{ax1}\PY{p}{)}\PY{p}{,} \PY{p}{(}\PY{n}{ax2}\PY{p}{)}\PY{p}{)} \PY{o}{=} \PY{n}{plt}\PY{o}{.}\PY{n}{subplots}\PY{p}{(}\PY{l+m+mi}{2}\PY{p}{,}\PY{l+m+mi}{1}\PY{p}{)}
\PY{n}{fig}\PY{o}{.}\PY{n}{set\PYZus{}size\PYZus{}inches}\PY{p}{(}\PY{l+m+mi}{20}\PY{p}{,} \PY{l+m+mi}{15}\PY{p}{)}

\PY{c+c1}{\PYZsh{} Overall title}
\PY{n}{fig}\PY{o}{.}\PY{n}{suptitle}\PY{p}{(}\PY{l+s+s1}{\PYZsq{}}\PY{l+s+s1}{Comparison between Lowest STD Stock and Highest STD Stock}\PY{l+s+s1}{\PYZsq{}}\PY{p}{)}

\PY{c+c1}{\PYZsh{} Subplot 1: The Price Weighted Index}
\PY{n}{ax1}\PY{o}{.}\PY{n}{plot}\PY{p}{(}\PY{n}{lowest\PYZus{}std\PYZus{}stock\PYZus{}hist\PYZus{}returns\PYZus{}df}\PY{o}{.}\PY{n}{index}\PY{p}{,} \PY{n}{lowest\PYZus{}std\PYZus{}stock\PYZus{}hist\PYZus{}returns\PYZus{}df}\PY{o}{.}\PY{n}{Close}\PY{p}{)}
\PY{n}{ax1}\PY{o}{.}\PY{n}{set\PYZus{}xlabel}\PY{p}{(}\PY{l+s+s2}{\PYZdq{}}\PY{l+s+s2}{Dates}\PY{l+s+s2}{\PYZdq{}}\PY{p}{)}
\PY{n}{ax1}\PY{o}{.}\PY{n}{set\PYZus{}ylabel}\PY{p}{(}\PY{l+s+s2}{\PYZdq{}}\PY{l+s+s2}{Daily Returns}\PY{l+s+s2}{\PYZdq{}}\PY{p}{)}
\PY{n}{ax1}\PY{o}{.}\PY{n}{title}\PY{o}{.}\PY{n}{set\PYZus{}text}\PY{p}{(}\PY{l+s+s2}{\PYZdq{}}\PY{l+s+s2}{Lowest STD Stock Daily Returns}\PY{l+s+s2}{\PYZdq{}}\PY{p}{)}

\PY{c+c1}{\PYZsh{} Subplot 2: The Market Weighted Index}
\PY{n}{ax2}\PY{o}{.}\PY{n}{plot}\PY{p}{(}\PY{n}{highest\PYZus{}std\PYZus{}stock\PYZus{}hist\PYZus{}returns\PYZus{}df}\PY{o}{.}\PY{n}{index}\PY{p}{,} \PY{n}{highest\PYZus{}std\PYZus{}stock\PYZus{}hist\PYZus{}returns\PYZus{}df}\PY{o}{.}\PY{n}{Close}\PY{p}{)}
\PY{n}{ax2}\PY{o}{.}\PY{n}{set\PYZus{}xlabel}\PY{p}{(}\PY{l+s+s2}{\PYZdq{}}\PY{l+s+s2}{Dates}\PY{l+s+s2}{\PYZdq{}}\PY{p}{)}
\PY{n}{ax2}\PY{o}{.}\PY{n}{set\PYZus{}ylabel}\PY{p}{(}\PY{l+s+s2}{\PYZdq{}}\PY{l+s+s2}{Daily Returns}\PY{l+s+s2}{\PYZdq{}}\PY{p}{)}
\PY{n}{ax2}\PY{o}{.}\PY{n}{title}\PY{o}{.}\PY{n}{set\PYZus{}text}\PY{p}{(}\PY{l+s+s2}{\PYZdq{}}\PY{l+s+s2}{Highest STD Stock Daily Returns}\PY{l+s+s2}{\PYZdq{}}\PY{p}{)}
\end{Verbatim}
\end{tcolorbox}

    \begin{center}
    \adjustimage{max size={0.9\linewidth}{0.9\paperheight}}{output_14_0.png}
    \end{center}
    { \hspace*{\fill} \\}
    
    \hypertarget{standard-deviation-discussion-contd}{%
\subsection{Standard Deviation Discussion
Cont'd:}\label{standard-deviation-discussion-contd}}

As we can see from our two graphs, generally, the returns of the stock
with the lower standard deviation have a smaller range and is generally
much less volatile compared to its higher standard deviation
counterpart. The stock with the highest standard deviation possesses
much more spikes in its returns compared to the safer stock. From this
graph, we can see that the stock with the lower STD can be seen as a
safer stock as its returns fluctuate significantly less than that of a
stock with a higher STD. Consequently, by assigning more points, and
therefore greater weighting on stocks that have a lower standard
deviation, we are effectively making our portfolio safer. In addition,
notice that although the fluctuations seem wide, the changes of the
daily returns for the safest stock only fluctuates between -2.0 and 2.0,
whereas the returns of the stock with the highest standard deviation
actually fluctuates with -8 and 8, this is a 4 time difference.

    \begin{tcolorbox}[breakable, size=fbox, boxrule=1pt, pad at break*=1mm,colback=cellbackground, colframe=cellborder]
\prompt{In}{incolor}{10}{\boxspacing}
\begin{Verbatim}[commandchars=\\\{\}]
\PY{o}{\PYZpc{}\PYZpc{}latex}
\PY{k}{\PYZbs{}newpage}
\end{Verbatim}
\end{tcolorbox}

    \newpage


    
    \hypertarget{expected-returns}{%
\subsection{Expected Returns:}\label{expected-returns}}

Another key factor to consider is the expected returns of each of the
individual stocks. For our safe portfolio, we want the expected returns
of our portfolio to be as close to 0 as possible. The expected returns
of a portfolio can be calculated using the following formula:

\begin{align*}
E(X) Portfolio = (Security1_{Expected Return} \times Security1_{Weight}) + (Security2_{Expected Return} \times Security2_{Weight}) + ... + (Security n_{Expected Return} \times Security n_{Weight})
\end{align*}

Consequently, to have the lowest expected returns for the portfolio, we
want the expected returns of each of the selected stocks to be as low as
possible. Therefore, we will be attempting to find the expected returns
of each of the stocks that are valid in our CSV file and then awarding
points to the stocks that have the lowest expected returns are selected
and have a higher weighting so that the expected return of our portfolio
is as close to 0 as possible.

    \begin{tcolorbox}[breakable, size=fbox, boxrule=1pt, pad at break*=1mm,colback=cellbackground, colframe=cellborder]
\prompt{In}{incolor}{11}{\boxspacing}
\begin{Verbatim}[commandchars=\\\{\}]
\PY{c+c1}{\PYZsh{}Get the expected returns}
\PY{c+c1}{\PYZsh{}We will be taking the absolute value as we want to consider the stock with the smallest expected return in magnitude}
\PY{n}{expected\PYZus{}returns} \PY{o}{=} \PY{n}{pd}\PY{o}{.}\PY{n}{DataFrame}\PY{p}{(}\PY{l+m+mi}{100} \PY{o}{*} \PY{n}{returns}\PY{o}{.}\PY{n}{mean}\PY{p}{(}\PY{p}{)}\PY{o}{.}\PY{n}{abs}\PY{p}{(}\PY{p}{)}\PY{p}{)}
\PY{c+c1}{\PYZsh{}Calculate the expected returns and plot it on dataframe}
\PY{n}{expected\PYZus{}returns}\PY{o}{.}\PY{n}{dropna}\PY{p}{(}\PY{n}{inplace} \PY{o}{=} \PY{k+kc}{True}\PY{p}{)}
\PY{n}{expected\PYZus{}returns} \PY{o}{=} \PY{n}{expected\PYZus{}returns}\PY{o}{.}\PY{n}{reset\PYZus{}index}\PY{p}{(}\PY{p}{)}
\PY{n}{expected\PYZus{}returns} \PY{o}{=} \PY{n}{expected\PYZus{}returns}\PY{o}{.}\PY{n}{rename}\PY{p}{(}\PY{n}{columns}\PY{o}{=}\PY{p}{\PYZob{}}\PY{l+s+s2}{\PYZdq{}}\PY{l+s+s2}{index}\PY{l+s+s2}{\PYZdq{}}\PY{p}{:} \PY{l+s+s2}{\PYZdq{}}\PY{l+s+s2}{Stocks}\PY{l+s+s2}{\PYZdq{}}\PY{p}{,} \PY{l+m+mi}{0}\PY{p}{:} \PY{l+s+s2}{\PYZdq{}}\PY{l+s+s2}{Expected Returns}\PY{l+s+s2}{\PYZdq{}}\PY{p}{\PYZcb{}}\PY{p}{)}
\PY{c+c1}{\PYZsh{}Sorted in increasing order of expected returns of each of the stocks}
\PY{n}{expected\PYZus{}returns} \PY{o}{=} \PY{n}{expected\PYZus{}returns}\PY{o}{.}\PY{n}{sort\PYZus{}values}\PY{p}{(}\PY{n}{by} \PY{o}{=} \PY{l+s+s2}{\PYZdq{}}\PY{l+s+s2}{Expected Returns}\PY{l+s+s2}{\PYZdq{}}\PY{p}{)}
\PY{n}{expected\PYZus{}returns} \PY{o}{=} \PY{n}{expected\PYZus{}returns}\PY{o}{.}\PY{n}{reset\PYZus{}index}\PY{p}{(}\PY{n}{drop} \PY{o}{=} \PY{k+kc}{True}\PY{p}{)}
\PY{n}{expected\PYZus{}returns}\PY{o}{.}\PY{n}{head}\PY{p}{(}\PY{p}{)}
\end{Verbatim}
\end{tcolorbox}

            \begin{tcolorbox}[breakable, size=fbox, boxrule=.5pt, pad at break*=1mm, opacityfill=0]
\prompt{Out}{outcolor}{11}{\boxspacing}
\begin{Verbatim}[commandchars=\\\{\}]
  Stocks  Expected Returns
0   SBUX          0.396995
1     GM          0.778134
2    MON          0.942456
3    LMT          1.049853
4     KO          1.157634
\end{Verbatim}
\end{tcolorbox}
        
    \begin{tcolorbox}[breakable, size=fbox, boxrule=1pt, pad at break*=1mm,colback=cellbackground, colframe=cellborder]
\prompt{In}{incolor}{12}{\boxspacing}
\begin{Verbatim}[commandchars=\\\{\}]
\PY{c+c1}{\PYZsh{}Address points for the percent expected returns}
\PY{c+c1}{\PYZsh{}Loop through all indices in the condensed list and produce a point score for each stock}
\PY{c+c1}{\PYZsh{}A stock recieves more points for having a low expected return because this indicates that the stock is not very volatile}
\PY{c+c1}{\PYZsh{}The goal of the portfolio is to be safe and have a return close to \PYZdl{}0 as possible so stocks with low expected returns will recieve more points}
\PY{n}{returns\PYZus{}points} \PY{o}{=} \PY{p}{[}\PY{p}{]}
\PY{k}{for} \PY{n}{i} \PY{o+ow}{in} \PY{n+nb}{range} \PY{p}{(}\PY{n+nb}{len}\PY{p}{(}\PY{n}{expected\PYZus{}returns}\PY{p}{)}\PY{p}{)}\PY{p}{:}
    \PY{k}{if} \PY{n}{i} \PY{o}{\PYZlt{}} \PY{l+m+mi}{20}\PY{p}{:}
        \PY{n}{returns\PYZus{}points}\PY{o}{.}\PY{n}{append}\PY{p}{(}\PY{l+m+mi}{20}\PY{o}{\PYZhy{}}\PY{n}{i}\PY{p}{)}             
    \PY{k}{else}\PY{p}{:}
        \PY{n}{returns\PYZus{}points}\PY{o}{.}\PY{n}{append}\PY{p}{(}\PY{l+m+mi}{0}\PY{p}{)}
\end{Verbatim}
\end{tcolorbox}

    \begin{tcolorbox}[breakable, size=fbox, boxrule=1pt, pad at break*=1mm,colback=cellbackground, colframe=cellborder]
\prompt{In}{incolor}{13}{\boxspacing}
\begin{Verbatim}[commandchars=\\\{\}]
\PY{c+c1}{\PYZsh{}Initialize points column for returns }
\PY{n}{expected\PYZus{}returns}\PY{p}{[}\PY{l+s+s1}{\PYZsq{}}\PY{l+s+s1}{PointsR}\PY{l+s+s1}{\PYZsq{}}\PY{p}{]} \PY{o}{=} \PY{n}{returns\PYZus{}points}
\PY{n}{expected\PYZus{}returns}\PY{o}{.}\PY{n}{head}\PY{p}{(}\PY{p}{)}
\end{Verbatim}
\end{tcolorbox}

            \begin{tcolorbox}[breakable, size=fbox, boxrule=.5pt, pad at break*=1mm, opacityfill=0]
\prompt{Out}{outcolor}{13}{\boxspacing}
\begin{Verbatim}[commandchars=\\\{\}]
  Stocks  Expected Returns  PointsR
0   SBUX          0.396995       20
1     GM          0.778134       19
2    MON          0.942456       18
3    LMT          1.049853       17
4     KO          1.157634       16
\end{Verbatim}
\end{tcolorbox}
        
    \hypertarget{expected-returns-discussion}{%
\subsection{Expected Returns
Discussion:}\label{expected-returns-discussion}}

In our expected\_returns dataframe, we have arranged the stocks in a
manner such that the stock with the lowest expected returns is placed
first in our dataframe and then the rest are sorted in increasing order
of expected returns.

In order to showcase how the expected returns of a portfolio will
change, we will graph the expected returns of a portfolio with the first
5 stocks in our dataframe, otherwise the 5 stocks with the lowest
expected returns, and we will also graph a portfolio with the last 5
stocks in our dataframe, otherwise the 5 stocks with the highest
expected returns. We will be calling on functions that get the closing
prices and make equally weighted portfolios. Then, we will be getting
the expected returns of both of the portfolios and comparing the two.

\textbf{Note}: Please note that these generated portfolios are not
entirely accurate nor ideal, as they will be made with each stock being
weighted equally as an assumption. Consequently, these graphs are here
to serve for general analysis to show the discrepancies between a
portfolio that has stocks with high expected returns compared to a
portfolio that has low expected returns. The assumed initial investment
value will be \$100,000.

    \begin{tcolorbox}[breakable, size=fbox, boxrule=1pt, pad at break*=1mm,colback=cellbackground, colframe=cellborder]
\prompt{In}{incolor}{14}{\boxspacing}
\begin{Verbatim}[commandchars=\\\{\}]
\PY{c+c1}{\PYZsh{}Getting the first 5 and the last 5 stocks from our dataframe into a list. }
\PY{n}{low\PYZus{}er\PYZus{}stocks} \PY{o}{=} \PY{n}{expected\PYZus{}returns}\PY{p}{[}\PY{l+s+s2}{\PYZdq{}}\PY{l+s+s2}{Stocks}\PY{l+s+s2}{\PYZdq{}}\PY{p}{]}\PY{o}{.}\PY{n}{iloc}\PY{p}{[}\PY{p}{:}\PY{l+m+mi}{5}\PY{p}{]}\PY{o}{.}\PY{n}{tolist}\PY{p}{(}\PY{p}{)}
\PY{n}{high\PYZus{}er\PYZus{}stocks} \PY{o}{=} \PY{n}{expected\PYZus{}returns}\PY{p}{[}\PY{l+s+s2}{\PYZdq{}}\PY{l+s+s2}{Stocks}\PY{l+s+s2}{\PYZdq{}}\PY{p}{]}\PY{o}{.}\PY{n}{iloc}\PY{p}{[}\PY{o}{\PYZhy{}}\PY{l+m+mi}{5}\PY{p}{:}\PY{p}{]}\PY{o}{.}\PY{n}{tolist}\PY{p}{(}\PY{p}{)}
\end{Verbatim}
\end{tcolorbox}

    \begin{tcolorbox}[breakable, size=fbox, boxrule=1pt, pad at break*=1mm,colback=cellbackground, colframe=cellborder]
\prompt{In}{incolor}{15}{\boxspacing}
\begin{Verbatim}[commandchars=\\\{\}]
\PY{c+c1}{\PYZsh{}A function that produces a dataframe that displays all of the closing prices in a list of stocks}
\PY{k}{def} \PY{n+nf}{get\PYZus{}closing\PYZus{}prices}\PY{p}{(}\PY{n}{list\PYZus{}of\PYZus{}stocks}\PY{p}{,} \PY{n}{start\PYZus{}date}\PY{p}{,} \PY{n}{end\PYZus{}date}\PY{p}{)}\PY{p}{:}
    \PY{n}{stocks\PYZus{}close\PYZus{}df} \PY{o}{=} \PY{n}{pd}\PY{o}{.}\PY{n}{DataFrame}\PY{p}{(}\PY{p}{)}
    \PY{k}{for} \PY{n}{i} \PY{o+ow}{in} \PY{n+nb}{range}\PY{p}{(}\PY{n+nb}{len}\PY{p}{(}\PY{n}{list\PYZus{}of\PYZus{}stocks}\PY{p}{)}\PY{p}{)}\PY{p}{:} 
        \PY{n}{stock} \PY{o}{=} \PY{n}{yf}\PY{o}{.}\PY{n}{Ticker}\PY{p}{(}\PY{n}{list\PYZus{}of\PYZus{}stocks}\PY{p}{[}\PY{n}{i}\PY{p}{]}\PY{p}{)}
        \PY{n}{stock\PYZus{}historical} \PY{o}{=} \PY{n}{stock}\PY{o}{.}\PY{n}{history}\PY{p}{(}\PY{n}{start}\PY{o}{=} \PY{n}{start\PYZus{}date} \PY{p}{,} \PY{n}{end} \PY{o}{=} \PY{n}{end\PYZus{}date} \PY{p}{,} \PY{n}{threads}\PY{o}{=}\PY{k+kc}{True}\PY{p}{)}
        \PY{n}{stocks\PYZus{}close\PYZus{}df}\PY{p}{[}\PY{n}{list\PYZus{}of\PYZus{}stocks}\PY{p}{[}\PY{n}{i}\PY{p}{]}\PY{p}{]} \PY{o}{=} \PY{n}{stock\PYZus{}historical}\PY{p}{[}\PY{l+s+s2}{\PYZdq{}}\PY{l+s+s2}{Close}\PY{l+s+s2}{\PYZdq{}}\PY{p}{]}
        \PY{n}{stocks\PYZus{}close\PYZus{}daily} \PY{o}{=} \PY{n}{stocks\PYZus{}close\PYZus{}df}\PY{o}{.}\PY{n}{resample}\PY{p}{(}\PY{l+s+s2}{\PYZdq{}}\PY{l+s+s2}{D}\PY{l+s+s2}{\PYZdq{}}\PY{p}{)}\PY{o}{.}\PY{n}{first}\PY{p}{(}\PY{p}{)}
        \PY{n}{stocks\PYZus{}close\PYZus{}daily} \PY{o}{=} \PY{n}{stocks\PYZus{}close\PYZus{}daily}\PY{o}{.}\PY{n}{dropna}\PY{p}{(}\PY{p}{)}
    \PY{k}{return} \PY{n}{stocks\PYZus{}close\PYZus{}daily}
\PY{c+c1}{\PYZsh{}For the low expected returns stocks:}
\PY{n}{low\PYZus{}er\PYZus{}close} \PY{o}{=} \PY{n}{get\PYZus{}closing\PYZus{}prices}\PY{p}{(}\PY{n}{low\PYZus{}er\PYZus{}stocks}\PY{p}{,} \PY{n}{start\PYZus{}date}\PY{p}{,} \PY{n}{end\PYZus{}date}\PY{p}{)}
\PY{c+c1}{\PYZsh{}For the high expected returns stocks:}
\PY{n}{high\PYZus{}er\PYZus{}close} \PY{o}{=} \PY{n}{get\PYZus{}closing\PYZus{}prices}\PY{p}{(}\PY{n}{high\PYZus{}er\PYZus{}stocks}\PY{p}{,} \PY{n}{start\PYZus{}date}\PY{p}{,} \PY{n}{end\PYZus{}date}\PY{p}{)}
\end{Verbatim}
\end{tcolorbox}

    \begin{tcolorbox}[breakable, size=fbox, boxrule=1pt, pad at break*=1mm,colback=cellbackground, colframe=cellborder]
\prompt{In}{incolor}{16}{\boxspacing}
\begin{Verbatim}[commandchars=\\\{\}]
\PY{c+c1}{\PYZsh{}  A function that consumes a list of stocks, and an initial investment size and a dataframe of closing prices of the list of stocks and produces a hypothetical portfolio where each stock is weighted the same. This hypothetical portfolio (which is not the final portfolio), will be used to calculate the correlation of each of the stocks to this hypothetical portfolio to see whether or not the stock in question has a low correlation with the rest of the stocks in the portfolio. }
\PY{k}{def} \PY{n+nf}{get\PYZus{}data}\PY{p}{(}\PY{n}{list\PYZus{}of\PYZus{}stocks}\PY{p}{,}\PY{n}{initial\PYZus{}investement}\PY{p}{,} \PY{n}{stock\PYZus{}close}\PY{p}{)}\PY{p}{:}
    \PY{n}{number\PYZus{}of\PYZus{}stocks} \PY{o}{=}\PY{p}{[}\PY{p}{]}
    \PY{n}{dataframe} \PY{o}{=} \PY{n}{pd}\PY{o}{.}\PY{n}{DataFrame}\PY{p}{(}\PY{p}{)}
    \PY{c+c1}{\PYZsh{} Here we are calculating the number of stocks that we will buy initially and their values. }
    \PY{k}{for} \PY{n}{i} \PY{o+ow}{in} \PY{n+nb}{range}\PY{p}{(}\PY{n+nb}{len}\PY{p}{(}\PY{n}{list\PYZus{}of\PYZus{}stocks}\PY{p}{)}\PY{p}{)}\PY{p}{:}
        \PY{n}{number\PYZus{}of\PYZus{}stocks} \PY{o}{=} \PY{p}{(}\PY{n}{initial\PYZus{}investement}\PY{o}{/}\PY{n+nb}{len}\PY{p}{(}\PY{n}{list\PYZus{}of\PYZus{}stocks}\PY{p}{)}\PY{p}{)}\PY{o}{/}\PY{n}{stock\PYZus{}close}\PY{o}{.}\PY{n}{iloc}\PY{p}{[}\PY{l+m+mi}{0}\PY{p}{,} \PY{n}{i}\PY{p}{]}
        \PY{n}{dataframe}\PY{p}{[}\PY{n}{list\PYZus{}of\PYZus{}stocks}\PY{p}{[}\PY{n}{i}\PY{p}{]} \PY{o}{+} \PY{l+s+s2}{\PYZdq{}}\PY{l+s+s2}{ value}\PY{l+s+s2}{\PYZdq{}}\PY{p}{]} \PY{o}{=} \PY{n}{stock\PYZus{}close}\PY{p}{[}\PY{n}{list\PYZus{}of\PYZus{}stocks}\PY{p}{[}\PY{n}{i}\PY{p}{]}\PY{p}{]} \PY{o}{*} \PY{n}{number\PYZus{}of\PYZus{}stocks}

    \PY{n}{returns} \PY{o}{=} \PY{n}{pd}\PY{o}{.}\PY{n}{DataFrame}\PY{p}{(}\PY{n}{index} \PY{o}{=} \PY{n}{stock\PYZus{}close}\PY{o}{.}\PY{n}{index}\PY{p}{)}
    \PY{n}{returns} \PY{o}{=} \PY{n}{pd}\PY{o}{.}\PY{n}{concat}\PY{p}{(}\PY{p}{[}\PY{n}{returns}\PY{p}{,} \PY{n}{dataframe}\PY{p}{]}\PY{p}{,} \PY{n}{join} \PY{o}{=} \PY{l+s+s2}{\PYZdq{}}\PY{l+s+s2}{outer}\PY{l+s+s2}{\PYZdq{}}\PY{p}{,} \PY{n}{axis} \PY{o}{=} \PY{l+m+mi}{1}\PY{p}{)}
    \PY{c+c1}{\PYZsh{} We are now summing all of the values of each stock to find the value of the total portfolio}
    \PY{n}{returns}\PY{p}{[}\PY{l+s+s2}{\PYZdq{}}\PY{l+s+s2}{Portfolio Value}\PY{l+s+s2}{\PYZdq{}}\PY{p}{]}\PY{o}{=} \PY{n}{returns}\PY{o}{.}\PY{n}{sum}\PY{p}{(}\PY{n}{axis}\PY{o}{=}\PY{l+m+mi}{1}\PY{p}{)}
    \PY{c+c1}{\PYZsh{} To find the returns, we are finding the \PYZpc{} change of the portfolio value. }
    \PY{n}{returns}\PY{p}{[}\PY{l+s+s2}{\PYZdq{}}\PY{l+s+s2}{Returns}\PY{l+s+s2}{\PYZdq{}}\PY{p}{]} \PY{o}{=} \PY{l+m+mi}{100} \PY{o}{*} \PY{n}{returns}\PY{p}{[}\PY{l+s+s2}{\PYZdq{}}\PY{l+s+s2}{Portfolio Value}\PY{l+s+s2}{\PYZdq{}}\PY{p}{]}\PY{o}{.}\PY{n}{pct\PYZus{}change}\PY{p}{(}\PY{p}{)}
    \PY{k}{return} \PY{n}{returns}

\PY{c+c1}{\PYZsh{} For the low expected returns portfolio:}
\PY{n}{low\PYZus{}er\PYZus{}portfolio} \PY{o}{=} \PY{n}{get\PYZus{}data}\PY{p}{(}\PY{n}{low\PYZus{}er\PYZus{}stocks}\PY{p}{,} \PY{l+m+mi}{100000}\PY{p}{,} \PY{n}{low\PYZus{}er\PYZus{}close}\PY{p}{)}
\PY{c+c1}{\PYZsh{} For the high expected returns portfolio:}
\PY{n}{high\PYZus{}er\PYZus{}portfolio} \PY{o}{=} \PY{n}{get\PYZus{}data}\PY{p}{(}\PY{n}{high\PYZus{}er\PYZus{}stocks}\PY{p}{,} \PY{l+m+mi}{100000}\PY{p}{,} \PY{n}{high\PYZus{}er\PYZus{}close}\PY{p}{)}
\end{Verbatim}
\end{tcolorbox}

    \begin{tcolorbox}[breakable, size=fbox, boxrule=1pt, pad at break*=1mm,colback=cellbackground, colframe=cellborder]
\prompt{In}{incolor}{17}{\boxspacing}
\begin{Verbatim}[commandchars=\\\{\}]
\PY{c+c1}{\PYZsh{} Making one dataframe to compare the two:}
\PY{n}{comparison} \PY{o}{=} \PY{n}{pd}\PY{o}{.}\PY{n}{DataFrame}\PY{p}{(}\PY{p}{)}
\PY{n}{comparison}\PY{p}{[}\PY{l+s+s2}{\PYZdq{}}\PY{l+s+s2}{Low ER Portfolio Returns}\PY{l+s+s2}{\PYZdq{}}\PY{p}{]} \PY{o}{=} \PY{n}{low\PYZus{}er\PYZus{}portfolio}\PY{p}{[}\PY{l+s+s2}{\PYZdq{}}\PY{l+s+s2}{Returns}\PY{l+s+s2}{\PYZdq{}}\PY{p}{]}
\PY{n}{comparison}\PY{p}{[}\PY{l+s+s2}{\PYZdq{}}\PY{l+s+s2}{High ER Portfolio Returns}\PY{l+s+s2}{\PYZdq{}}\PY{p}{]} \PY{o}{=} \PY{n}{high\PYZus{}er\PYZus{}portfolio}\PY{p}{[}\PY{l+s+s2}{\PYZdq{}}\PY{l+s+s2}{Returns}\PY{l+s+s2}{\PYZdq{}}\PY{p}{]}

\PY{c+c1}{\PYZsh{} Printing the expected returns:}
\PY{n+nb}{print}\PY{p}{(}\PY{l+s+s1}{\PYZsq{}}\PY{l+s+s1}{Expected Returns:}\PY{l+s+s1}{\PYZsq{}}\PY{p}{)}
\PY{n+nb}{print}\PY{p}{(}\PY{n}{comparison}\PY{o}{.}\PY{n}{mean}\PY{p}{(}\PY{p}{)}\PY{p}{)}
\end{Verbatim}
\end{tcolorbox}

    \begin{Verbatim}[commandchars=\\\{\}]
Expected Returns:
Low ER Portfolio Returns     0.001213
High ER Portfolio Returns    0.139980
dtype: float64
    \end{Verbatim}

    \hypertarget{expected-returns-discussion-contd}{%
\subsection{Expected Returns Discussion
Cont'd:}\label{expected-returns-discussion-contd}}

As we can see from our above code and calculations, the expected returns
of the portfolio that has the stocks with high expected returns, is
significantly higher than the expected returns of the portfolio that is
composed of stocks with low expected returns. Consequently, from this
result and this generalized example, we can see that choosing stocks
with low expected returns will also result in the expected returns of
the portfolio to be lower compared to that of choosing stocks that have
high expected returns. Ultimately, this demonstrates why stocks with
lower expected returns were given more points as we want the stocks with
lower expected returns to be weighted higher so that the expected
returns of our final portfolio are lower.

    \begin{tcolorbox}[breakable, size=fbox, boxrule=1pt, pad at break*=1mm,colback=cellbackground, colframe=cellborder]
\prompt{In}{incolor}{18}{\boxspacing}
\begin{Verbatim}[commandchars=\\\{\}]
\PY{o}{\PYZpc{}\PYZpc{}latex}
\PY{k}{\PYZbs{}newpage}
\end{Verbatim}
\end{tcolorbox}

    \newpage


    
    \hypertarget{beta}{%
\subsection{Beta:}\label{beta}}

Since we are attempting to build a safe portfolio, and thus minimize
risk, we can use another measure of risk known as beta to further
reaffirm that we are choosing the least risky stocks. Beta is a measure
of the risk of a single stock relative to the entire market. The higher
the beta value, the riskier the stock. Beta will tell us how the stock
is moving relative to the movement of the market. For instance, if a
stock has a beta of 1.5, then this means that the returns of this stock
are 1.5 times more volatile than the returns of the market.
Consequently, if we are able to find stocks that have beta values that
are close to 0, this means that the stock would have low expected
returns and would also indicate that the stock's movement is
uncorrelated with the market. These stocks will then be extremely safe
and the returns would not move with the market, thereby minimizing the
returns that are expected.

Consequently, we will be attempting to find stocks with a beta value
close to 0.

    \begin{tcolorbox}[breakable, size=fbox, boxrule=1pt, pad at break*=1mm,colback=cellbackground, colframe=cellborder]
\prompt{In}{incolor}{19}{\boxspacing}
\begin{Verbatim}[commandchars=\\\{\}]
\PY{c+c1}{\PYZsh{} Getting a list of all of the stocks}
\PY{n}{list\PYZus{}of\PYZus{}all\PYZus{}stocks} \PY{o}{=} \PY{n}{returns}\PY{o}{.}\PY{n}{columns}\PY{o}{.}\PY{n}{values}\PY{o}{.}\PY{n}{tolist}\PY{p}{(}\PY{p}{)}
\end{Verbatim}
\end{tcolorbox}

    \begin{tcolorbox}[breakable, size=fbox, boxrule=1pt, pad at break*=1mm,colback=cellbackground, colframe=cellborder]
\prompt{In}{incolor}{20}{\boxspacing}
\begin{Verbatim}[commandchars=\\\{\}]
\PY{c+c1}{\PYZsh{} Making a dataframe of the market index}
\PY{n}{market\PYZus{}index} \PY{o}{=} \PY{n}{pd}\PY{o}{.}\PY{n}{DataFrame}\PY{p}{(}\PY{p}{)}
\PY{c+c1}{\PYZsh{} \PYZsq{}\PYZca{}GSPC is the ticker of the S\PYZam{}P 500, this will be used as a general market index in this assignment}
\PY{n}{MarketIndex} \PY{o}{=} \PY{l+s+s1}{\PYZsq{}}\PY{l+s+s1}{\PYZca{}GSPC}\PY{l+s+s1}{\PYZsq{}}
\PY{n}{MarketIndex\PYZus{}hist} \PY{o}{=} \PY{n}{yf}\PY{o}{.}\PY{n}{Ticker}\PY{p}{(}\PY{n}{MarketIndex}\PY{p}{)}\PY{o}{.}\PY{n}{history}\PY{p}{(}\PY{n}{start}\PY{o}{=}\PY{n}{start\PYZus{}date}\PY{p}{,} \PY{n}{end}\PY{o}{=}\PY{n}{end\PYZus{}date}\PY{p}{)}
\PY{n}{market\PYZus{}index}\PY{p}{[}\PY{n}{MarketIndex}\PY{p}{]} \PY{o}{=} \PY{l+m+mi}{100} \PY{o}{*} \PY{n}{MarketIndex\PYZus{}hist}\PY{p}{[}\PY{l+s+s1}{\PYZsq{}}\PY{l+s+s1}{Close}\PY{l+s+s1}{\PYZsq{}}\PY{p}{]}\PY{o}{.}\PY{n}{resample}\PY{p}{(}\PY{l+s+s1}{\PYZsq{}}\PY{l+s+s1}{D}\PY{l+s+s1}{\PYZsq{}}\PY{p}{)}\PY{o}{.}\PY{n}{ffill}\PY{p}{(}\PY{p}{)}\PY{o}{.}\PY{n}{pct\PYZus{}change}\PY{p}{(}\PY{p}{)}
\PY{n}{market\PYZus{}index} \PY{o}{=} \PY{n}{market\PYZus{}index}\PY{o}{.}\PY{n}{iloc}\PY{p}{[}\PY{l+m+mi}{1}\PY{p}{:}\PY{p}{]}
\PY{c+c1}{\PYZsh{} Calculating the variance of the market}
\PY{n}{MarketVar}\PY{o}{=} \PY{n}{market\PYZus{}index}\PY{p}{[}\PY{n}{MarketIndex}\PY{p}{]}\PY{o}{.}\PY{n}{var}\PY{p}{(}\PY{p}{)}

\PY{c+c1}{\PYZsh{} Creating a dataframe with all of the Betas of each of the stock relative to the market}
\PY{n}{df} \PY{o}{=} \PY{n}{pd}\PY{o}{.}\PY{n}{DataFrame}\PY{p}{(}\PY{p}{)}
\PY{k}{for} \PY{n}{i} \PY{o+ow}{in} \PY{n+nb}{range}\PY{p}{(}\PY{n+nb}{len}\PY{p}{(}\PY{n}{list\PYZus{}of\PYZus{}all\PYZus{}stocks}\PY{p}{)}\PY{p}{)}\PY{p}{:}
    \PY{n}{stock\PYZus{}hist} \PY{o}{=} \PY{n}{yf}\PY{o}{.}\PY{n}{Ticker}\PY{p}{(}\PY{n}{list\PYZus{}of\PYZus{}all\PYZus{}stocks}\PY{p}{[}\PY{n}{i}\PY{p}{]}\PY{p}{)}\PY{o}{.}\PY{n}{history}\PY{p}{(}\PY{n}{start}\PY{o}{=}\PY{n}{start\PYZus{}date}\PY{p}{,} \PY{n}{end}\PY{o}{=}\PY{n}{end\PYZus{}date}\PY{p}{)}
    \PY{n}{stock\PYZus{}returns} \PY{o}{=} \PY{n}{pd}\PY{o}{.}\PY{n}{DataFrame}\PY{p}{(}\PY{p}{)}
    \PY{n}{stock\PYZus{}returns}\PY{p}{[}\PY{n}{list\PYZus{}of\PYZus{}all\PYZus{}stocks}\PY{p}{[}\PY{n}{i}\PY{p}{]} \PY{o}{+} \PY{l+s+s2}{\PYZdq{}}\PY{l+s+s2}{ Returns}\PY{l+s+s2}{\PYZdq{}}\PY{p}{]} \PY{o}{=} \PY{l+m+mi}{100} \PY{o}{*} \PY{n}{stock\PYZus{}hist}\PY{p}{[}\PY{l+s+s1}{\PYZsq{}}\PY{l+s+s1}{Close}\PY{l+s+s1}{\PYZsq{}}\PY{p}{]}\PY{o}{.}\PY{n}{resample}\PY{p}{(}\PY{l+s+s1}{\PYZsq{}}\PY{l+s+s1}{D}\PY{l+s+s1}{\PYZsq{}}\PY{p}{)}\PY{o}{.}\PY{n}{ffill}\PY{p}{(}\PY{p}{)}\PY{o}{.}\PY{n}{pct\PYZus{}change}\PY{p}{(}\PY{p}{)}
    \PY{n}{stock\PYZus{}returns} \PY{o}{=} \PY{n}{stock\PYZus{}returns}\PY{o}{.}\PY{n}{iloc}\PY{p}{[}\PY{l+m+mi}{1}\PY{p}{:}\PY{p}{]}
    \PY{n}{stock\PYZus{}returnsb} \PY{o}{=} \PY{n}{pd}\PY{o}{.}\PY{n}{concat}\PY{p}{(}\PY{p}{[}\PY{n}{stock\PYZus{}returns}\PY{p}{,} \PY{n}{market\PYZus{}index}\PY{p}{]}\PY{p}{,} \PY{n}{join}\PY{o}{=}\PY{l+s+s1}{\PYZsq{}}\PY{l+s+s1}{inner}\PY{l+s+s1}{\PYZsq{}}\PY{p}{,} \PY{n}{axis}\PY{o}{=}\PY{l+m+mi}{1}\PY{p}{)}
    \PY{c+c1}{\PYZsh{} Calculating Beta}
    \PY{c+c1}{\PYZsh{} Note that we will be taking the absolute value of the beta values. Although beta can technically be negative, we are attempting to find the stocks that have a beta value that is closest to 0 and so taking the absolute value of the beta value will allows to find these stocks that have a beta value closest to 0. }
    \PY{n}{Beta}\PY{o}{=} \PY{n+nb}{abs}\PY{p}{(}\PY{n}{stock\PYZus{}returnsb}\PY{o}{.}\PY{n}{cov}\PY{p}{(}\PY{p}{)}\PY{o}{/}\PY{n}{MarketVar}\PY{p}{)}
    \PY{n}{df}\PY{o}{.}\PY{n}{loc}\PY{p}{[}\PY{l+m+mi}{0}\PY{p}{,} \PY{n}{list\PYZus{}of\PYZus{}all\PYZus{}stocks}\PY{p}{[}\PY{n}{i}\PY{p}{]}\PY{p}{]} \PY{o}{=} \PY{n}{Beta}\PY{o}{.}\PY{n}{iat}\PY{p}{[}\PY{l+m+mi}{0}\PY{p}{,}\PY{l+m+mi}{1}\PY{p}{]}

\PY{n}{df}
\end{Verbatim}
\end{tcolorbox}

            \begin{tcolorbox}[breakable, size=fbox, boxrule=.5pt, pad at break*=1mm, opacityfill=0]
\prompt{Out}{outcolor}{20}{\boxspacing}
\begin{Verbatim}[commandchars=\\\{\}]
        BLK      ABBV     COF      GOOG        CL        GM       CVS  \textbackslash{}
0  1.461315  0.679155  1.4354  1.288522  0.345618  1.222352  0.764027

       CSCO        C       KMI  {\ldots}      ACN        T       USB       SPG  \textbackslash{}
0  0.881408  1.28103  1.031046  {\ldots}  1.03159  0.67083  1.226881  1.532603

        NEE        KO      OXY      QCOM      SBUX        PM
0  0.415869  0.543359  2.02342  1.097116  0.828057  0.657658

[1 rows x 55 columns]
\end{Verbatim}
\end{tcolorbox}
        
    \begin{tcolorbox}[breakable, size=fbox, boxrule=1pt, pad at break*=1mm,colback=cellbackground, colframe=cellborder]
\prompt{In}{incolor}{21}{\boxspacing}
\begin{Verbatim}[commandchars=\\\{\}]
\PY{c+c1}{\PYZsh{} calculate beta value for each stock in the condensed list}
\PY{n}{beta} \PY{o}{=} \PY{n}{df}\PY{o}{.}\PY{n}{T}
\PY{n}{beta} \PY{o}{=} \PY{n}{beta}\PY{o}{.}\PY{n}{reset\PYZus{}index}\PY{p}{(}\PY{p}{)}
\PY{n}{beta} \PY{o}{=} \PY{n}{beta}\PY{o}{.}\PY{n}{rename}\PY{p}{(}\PY{n}{columns}\PY{o}{=}\PY{p}{\PYZob{}}\PY{l+s+s2}{\PYZdq{}}\PY{l+s+s2}{index}\PY{l+s+s2}{\PYZdq{}}\PY{p}{:} \PY{l+s+s2}{\PYZdq{}}\PY{l+s+s2}{Stocks}\PY{l+s+s2}{\PYZdq{}}\PY{p}{,} \PY{l+m+mi}{0}\PY{p}{:} \PY{l+s+s2}{\PYZdq{}}\PY{l+s+s2}{Beta}\PY{l+s+s2}{\PYZdq{}}\PY{p}{\PYZcb{}}\PY{p}{)}
\PY{n}{beta}\PY{o}{.}\PY{n}{dropna}\PY{p}{(}\PY{n}{inplace} \PY{o}{=} \PY{k+kc}{True}\PY{p}{)}
\PY{c+c1}{\PYZsh{} Sorting all of them in increasing order of Beta Values}
\PY{n}{beta} \PY{o}{=} \PY{n}{beta}\PY{o}{.}\PY{n}{sort\PYZus{}values}\PY{p}{(}\PY{n}{by} \PY{o}{=} \PY{l+s+s1}{\PYZsq{}}\PY{l+s+s1}{Beta}\PY{l+s+s1}{\PYZsq{}}\PY{p}{)}
\PY{n}{beta} \PY{o}{=} \PY{n}{beta}\PY{o}{.}\PY{n}{reset\PYZus{}index}\PY{p}{(}\PY{n}{drop} \PY{o}{=} \PY{k+kc}{True}\PY{p}{)}
\PY{n}{beta}\PY{o}{.}\PY{n}{head}\PY{p}{(}\PY{p}{)}
\end{Verbatim}
\end{tcolorbox}

            \begin{tcolorbox}[breakable, size=fbox, boxrule=.5pt, pad at break*=1mm, opacityfill=0]
\prompt{Out}{outcolor}{21}{\boxspacing}
\begin{Verbatim}[commandchars=\\\{\}]
  Stocks      Beta
0    MON  0.027794
1     PG  0.275025
2     SO  0.297969
3     CL  0.345618
4    PEP  0.370259
\end{Verbatim}
\end{tcolorbox}
        
    \begin{tcolorbox}[breakable, size=fbox, boxrule=1pt, pad at break*=1mm,colback=cellbackground, colframe=cellborder]
\prompt{In}{incolor}{22}{\boxspacing}
\begin{Verbatim}[commandchars=\\\{\}]
\PY{c+c1}{\PYZsh{}Assign points for each stock\PYZsq{}s beta value}
\PY{c+c1}{\PYZsh{}Loop through all indices in the condensed list and produce a point score for each stock}
\PY{c+c1}{\PYZsh{}A stock recieves more points for having a beta close to zero is less risky and offers low returns}
\PY{c+c1}{\PYZsh{}The goal of the portfolio is to be safe and have a return close to \PYZdl{}0 as possible so stocks with low betas will recieve more points}
\PY{n}{beta\PYZus{}points} \PY{o}{=} \PY{p}{[}\PY{p}{]}
\PY{k}{for} \PY{n}{i} \PY{o+ow}{in} \PY{n+nb}{range} \PY{p}{(}\PY{n+nb}{len}\PY{p}{(}\PY{n}{beta}\PY{p}{)}\PY{p}{)}\PY{p}{:}
    \PY{k}{if} \PY{n}{i} \PY{o}{\PYZlt{}} \PY{l+m+mi}{20}\PY{p}{:}
        \PY{n}{beta\PYZus{}points}\PY{o}{.}\PY{n}{append}\PY{p}{(}\PY{l+m+mi}{20}\PY{o}{\PYZhy{}}\PY{n}{i}\PY{p}{)}              
    \PY{k}{else}\PY{p}{:}
        \PY{n}{beta\PYZus{}points}\PY{o}{.}\PY{n}{append}\PY{p}{(}\PY{l+m+mi}{0}\PY{p}{)}
\PY{c+c1}{\PYZsh{}returns\PYZus{}points}
\end{Verbatim}
\end{tcolorbox}

    \begin{tcolorbox}[breakable, size=fbox, boxrule=1pt, pad at break*=1mm,colback=cellbackground, colframe=cellborder]
\prompt{In}{incolor}{23}{\boxspacing}
\begin{Verbatim}[commandchars=\\\{\}]
\PY{c+c1}{\PYZsh{} Initialize a points column for beta values}
\PY{n}{beta}\PY{p}{[}\PY{l+s+s1}{\PYZsq{}}\PY{l+s+s1}{PointsB}\PY{l+s+s1}{\PYZsq{}}\PY{p}{]} \PY{o}{=} \PY{n}{beta\PYZus{}points}
\PY{n}{beta}\PY{o}{.}\PY{n}{head}\PY{p}{(}\PY{p}{)}
\end{Verbatim}
\end{tcolorbox}

            \begin{tcolorbox}[breakable, size=fbox, boxrule=.5pt, pad at break*=1mm, opacityfill=0]
\prompt{Out}{outcolor}{23}{\boxspacing}
\begin{Verbatim}[commandchars=\\\{\}]
  Stocks      Beta  PointsB
0    MON  0.027794       20
1     PG  0.275025       19
2     SO  0.297969       18
3     CL  0.345618       17
4    PEP  0.370259       16
\end{Verbatim}
\end{tcolorbox}
        
    \hypertarget{beta-discussion}{%
\subsection{Beta Discussion:}\label{beta-discussion}}

Using the code above, we were able to find the beta of each individual
stock and award points to the stocks that have the lowest beta values.
Similar to our analysis for the standard deviation, we can compare the
returns between the stock with the lowest beta value and the stock with
the highest beta value as well as the returns of the market to showcase
how the stock with the lowest beta will also yield the lowest expected
returns and why we chose stocks with low beta values.

    \begin{tcolorbox}[breakable, size=fbox, boxrule=1pt, pad at break*=1mm,colback=cellbackground, colframe=cellborder]
\prompt{In}{incolor}{24}{\boxspacing}
\begin{Verbatim}[commandchars=\\\{\}]
\PY{c+c1}{\PYZsh{} Getting the two stocks:}
\PY{n}{lowest\PYZus{}beta\PYZus{}stock} \PY{o}{=} \PY{n}{yf}\PY{o}{.}\PY{n}{Ticker}\PY{p}{(}\PY{n+nb}{str}\PY{p}{(}\PY{n}{beta}\PY{p}{[}\PY{l+s+s2}{\PYZdq{}}\PY{l+s+s2}{Stocks}\PY{l+s+s2}{\PYZdq{}}\PY{p}{]}\PY{o}{.}\PY{n}{iloc}\PY{p}{[}\PY{l+m+mi}{0}\PY{p}{]}\PY{p}{)}\PY{p}{)}
\PY{n}{highest\PYZus{}beta\PYZus{}stock} \PY{o}{=} \PY{n}{yf}\PY{o}{.}\PY{n}{Ticker}\PY{p}{(}\PY{n+nb}{str}\PY{p}{(}\PY{n}{beta}\PY{p}{[}\PY{l+s+s2}{\PYZdq{}}\PY{l+s+s2}{Stocks}\PY{l+s+s2}{\PYZdq{}}\PY{p}{]}\PY{o}{.}\PY{n}{iloc}\PY{p}{[}\PY{o}{\PYZhy{}}\PY{l+m+mi}{1}\PY{p}{]}\PY{p}{)}\PY{p}{)}

\PY{c+c1}{\PYZsh{} Getting their returns from the historical data}
\PY{n}{lowest\PYZus{}beta\PYZus{}stock\PYZus{}df} \PY{o}{=} \PY{n}{pd}\PY{o}{.}\PY{n}{DataFrame}\PY{p}{(}\PY{l+m+mi}{100} \PY{o}{*} \PY{n}{lowest\PYZus{}beta\PYZus{}stock}\PY{o}{.}\PY{n}{history}\PY{p}{(}\PY{n}{start} \PY{o}{=} \PY{n}{start\PYZus{}date}\PY{p}{,} \PY{n}{end} \PY{o}{=} \PY{n}{end\PYZus{}date}\PY{p}{)}\PY{o}{.}\PY{n}{Close}\PY{o}{.}\PY{n}{pct\PYZus{}change}\PY{p}{(}\PY{p}{)}\PY{o}{.}\PY{n}{iloc}\PY{p}{[}\PY{l+m+mi}{1}\PY{p}{:}\PY{p}{]}\PY{p}{)}
\PY{n}{highest\PYZus{}beta\PYZus{}stock\PYZus{}df} \PY{o}{=} \PY{n}{pd}\PY{o}{.}\PY{n}{DataFrame}\PY{p}{(}\PY{l+m+mi}{100} \PY{o}{*} \PY{n}{highest\PYZus{}beta\PYZus{}stock}\PY{o}{.}\PY{n}{history}\PY{p}{(}\PY{n}{start} \PY{o}{=} \PY{n}{start\PYZus{}date}\PY{p}{,} \PY{n}{end} \PY{o}{=} \PY{n}{end\PYZus{}date}\PY{p}{)}\PY{o}{.}\PY{n}{Close}\PY{o}{.}\PY{n}{pct\PYZus{}change}\PY{p}{(}\PY{p}{)}\PY{o}{.}\PY{n}{iloc}\PY{p}{[}\PY{l+m+mi}{1}\PY{p}{:}\PY{p}{]}\PY{p}{)}
\PY{n}{highest\PYZus{}beta\PYZus{}stock\PYZus{}df}\PY{p}{[}\PY{l+s+s2}{\PYZdq{}}\PY{l+s+s2}{Highest Beta Returns}\PY{l+s+s2}{\PYZdq{}}\PY{p}{]} \PY{o}{=} \PY{n}{highest\PYZus{}beta\PYZus{}stock\PYZus{}df}\PY{p}{[}\PY{l+s+s2}{\PYZdq{}}\PY{l+s+s2}{Close}\PY{l+s+s2}{\PYZdq{}}\PY{p}{]}
\PY{n}{highest\PYZus{}beta\PYZus{}stock\PYZus{}df} \PY{o}{=} \PY{n}{highest\PYZus{}beta\PYZus{}stock\PYZus{}df}\PY{o}{.}\PY{n}{drop}\PY{p}{(}\PY{n}{columns} \PY{o}{=} \PY{p}{[}\PY{l+s+s1}{\PYZsq{}}\PY{l+s+s1}{Close}\PY{l+s+s1}{\PYZsq{}}\PY{p}{]}\PY{p}{)}
\PY{n}{combined\PYZus{}beta\PYZus{}df} \PY{o}{=} \PY{n}{pd}\PY{o}{.}\PY{n}{concat}\PY{p}{(}\PY{p}{[}\PY{n}{lowest\PYZus{}beta\PYZus{}stock\PYZus{}df}\PY{p}{,} \PY{n}{highest\PYZus{}beta\PYZus{}stock\PYZus{}df}\PY{p}{,} \PY{n}{market\PYZus{}index}\PY{p}{]}\PY{p}{,} \PY{n}{join} \PY{o}{=} \PY{l+s+s2}{\PYZdq{}}\PY{l+s+s2}{inner}\PY{l+s+s2}{\PYZdq{}}\PY{p}{,} \PY{n}{axis} \PY{o}{=} \PY{l+m+mi}{1}\PY{p}{)}

\PY{n}{plt}\PY{o}{.}\PY{n}{figure}\PY{p}{(}\PY{n}{figsize}\PY{o}{=}\PY{p}{(}\PY{l+m+mi}{15}\PY{p}{,}\PY{l+m+mi}{10}\PY{p}{)}\PY{p}{)}
\PY{c+c1}{\PYZsh{} Plot line for lowest\PYZus{}beta\PYZus{}stock}
\PY{n}{plt}\PY{o}{.}\PY{n}{plot}\PY{p}{(}\PY{n}{combined\PYZus{}beta\PYZus{}df}\PY{o}{.}\PY{n}{index}\PY{p}{,} \PY{n}{combined\PYZus{}beta\PYZus{}df}\PY{p}{[}\PY{l+s+s2}{\PYZdq{}}\PY{l+s+s2}{Close}\PY{l+s+s2}{\PYZdq{}}\PY{p}{]}\PY{p}{,} \PY{n}{label} \PY{o}{=} \PY{l+s+s1}{\PYZsq{}}\PY{l+s+s1}{Lowest Beta Stock}\PY{l+s+s1}{\PYZsq{}} \PY{p}{,} \PY{n}{marker}\PY{o}{=}\PY{l+s+s1}{\PYZsq{}}\PY{l+s+s1}{o}\PY{l+s+s1}{\PYZsq{}}\PY{p}{,} \PY{n}{ls}\PY{o}{=}\PY{l+s+s1}{\PYZsq{}}\PY{l+s+s1}{:}\PY{l+s+s1}{\PYZsq{}}\PY{p}{,} \PY{n}{color}\PY{o}{=}\PY{l+s+s1}{\PYZsq{}}\PY{l+s+s1}{b}\PY{l+s+s1}{\PYZsq{}}\PY{p}{)}
\PY{c+c1}{\PYZsh{} Plot line for highest\PYZus{}beta\PYZus{}stock}
\PY{n}{plt}\PY{o}{.}\PY{n}{plot}\PY{p}{(}\PY{n}{combined\PYZus{}beta\PYZus{}df}\PY{o}{.}\PY{n}{index}\PY{p}{,} \PY{n}{combined\PYZus{}beta\PYZus{}df}\PY{p}{[}\PY{l+s+s2}{\PYZdq{}}\PY{l+s+s2}{Highest Beta Returns}\PY{l+s+s2}{\PYZdq{}}\PY{p}{]} \PY{p}{,}\PY{n}{label} \PY{o}{=} \PY{l+s+s1}{\PYZsq{}}\PY{l+s+s1}{Highest Beta Stock}\PY{l+s+s1}{\PYZsq{}}\PY{p}{,} \PY{n}{marker} \PY{o}{=} \PY{l+s+s1}{\PYZsq{}}\PY{l+s+s1}{x}\PY{l+s+s1}{\PYZsq{}}\PY{p}{,}  \PY{n}{ls}\PY{o}{=}\PY{l+s+s1}{\PYZsq{}}\PY{l+s+s1}{:}\PY{l+s+s1}{\PYZsq{}}\PY{p}{,} \PY{n}{color}\PY{o}{=}\PY{l+s+s1}{\PYZsq{}}\PY{l+s+s1}{r}\PY{l+s+s1}{\PYZsq{}}\PY{p}{)}
\PY{c+c1}{\PYZsh{} Plot line for market}
\PY{n}{plt}\PY{o}{.}\PY{n}{plot}\PY{p}{(}\PY{n}{combined\PYZus{}beta\PYZus{}df}\PY{o}{.}\PY{n}{index}\PY{p}{,} \PY{n}{combined\PYZus{}beta\PYZus{}df}\PY{p}{[}\PY{l+s+s2}{\PYZdq{}}\PY{l+s+s2}{\PYZca{}GSPC}\PY{l+s+s2}{\PYZdq{}}\PY{p}{]} \PY{p}{,}\PY{n}{label} \PY{o}{=} \PY{l+s+s1}{\PYZsq{}}\PY{l+s+s1}{Market}\PY{l+s+s1}{\PYZsq{}}\PY{p}{,} \PY{n}{marker} \PY{o}{=} \PY{l+s+s1}{\PYZsq{}}\PY{l+s+s1}{.}\PY{l+s+s1}{\PYZsq{}}\PY{p}{,}  \PY{n}{ls}\PY{o}{=}\PY{l+s+s1}{\PYZsq{}}\PY{l+s+s1}{:}\PY{l+s+s1}{\PYZsq{}}\PY{p}{,} \PY{n}{color}\PY{o}{=}\PY{l+s+s1}{\PYZsq{}}\PY{l+s+s1}{g}\PY{l+s+s1}{\PYZsq{}}\PY{p}{)}


\PY{c+c1}{\PYZsh{} Title }
\PY{n}{plt}\PY{o}{.}\PY{n}{title}\PY{p}{(}\PY{l+s+s1}{\PYZsq{}}\PY{l+s+s1}{Returns of lowest beta, highest beta and market}\PY{l+s+s1}{\PYZsq{}}\PY{p}{)}
\PY{c+c1}{\PYZsh{} Add axes labels}
\PY{n}{plt}\PY{o}{.}\PY{n}{xlabel}\PY{p}{(}\PY{l+s+s1}{\PYZsq{}}\PY{l+s+s1}{Dates}\PY{l+s+s1}{\PYZsq{}}\PY{p}{)}
\PY{n}{plt}\PY{o}{.}\PY{n}{xticks}\PY{p}{(}\PY{n}{rotation}\PY{o}{=} \PY{l+m+mi}{60}\PY{p}{)}
\PY{n}{plt}\PY{o}{.}\PY{n}{ylabel}\PY{p}{(}\PY{l+s+s1}{\PYZsq{}}\PY{l+s+s1}{Daily Returns}\PY{l+s+s1}{\PYZsq{}}\PY{p}{)}
\PY{n}{plt}\PY{o}{.}\PY{n}{legend}\PY{p}{(}\PY{n}{loc} \PY{o}{=} \PY{l+s+s1}{\PYZsq{}}\PY{l+s+s1}{best}\PY{l+s+s1}{\PYZsq{}}\PY{p}{)}

\PY{n}{plt}\PY{o}{.}\PY{n}{show}\PY{p}{(}\PY{p}{)}
\end{Verbatim}
\end{tcolorbox}

    \begin{center}
    \adjustimage{max size={0.9\linewidth}{0.9\paperheight}}{output_35_0.png}
    \end{center}
    { \hspace*{\fill} \\}
    
    \hypertarget{beta-discussion-contd}{%
\subsection{Beta Discussion Cont'd:}\label{beta-discussion-contd}}

From the graph, we can see that, generally, the returns of the stock
with the highest beta had more volatile returns than that of the lower
beta stock. While this is not always the case, this is generally true
which is further backed by the beta calculations we did in the code.
Since the returns of lower beta stocks have, generally (not always as
there are instances when this is not true), lower volatility than their
higher beta stocks, then this indicates that lower beta stocks are a
better fit for our safe portfolio. While low beta stocks can still be
risky, they are simply less volatile which justifies our allocation of
points for this factor.

    \begin{tcolorbox}[breakable, size=fbox, boxrule=1pt, pad at break*=1mm,colback=cellbackground, colframe=cellborder]
\prompt{In}{incolor}{25}{\boxspacing}
\begin{Verbatim}[commandchars=\\\{\}]
\PY{o}{\PYZpc{}\PYZpc{}latex}
\PY{k}{\PYZbs{}newpage}
\end{Verbatim}
\end{tcolorbox}

    \newpage


    
    \hypertarget{sorting-the-stocks-by-their-points}{%
\subsection{Sorting the Stocks by their
Points:}\label{sorting-the-stocks-by-their-points}}

In this section, we will be compiling all of the stocks into one large
dataframe in order of their points. In each section, we will be sorting
the stocks in alphabetical order so that each of the separate dataframes
can then be merged together. Then, we shall combine all of the different
stocks and their associated point values for each of the factors listed
above and then all of the points of each factor for each stock will be
summed together to find the total points that each stock earned.

    \begin{tcolorbox}[breakable, size=fbox, boxrule=1pt, pad at break*=1mm,colback=cellbackground, colframe=cellborder]
\prompt{In}{incolor}{26}{\boxspacing}
\begin{Verbatim}[commandchars=\\\{\}]
\PY{n}{std} \PY{o}{=} \PY{n}{std}\PY{o}{.}\PY{n}{reset\PYZus{}index}\PY{p}{(}\PY{n}{drop} \PY{o}{=} \PY{k+kc}{True}\PY{p}{)}
\PY{c+c1}{\PYZsh{}Sort stocks in order alphabetical order in preparations to add the points together}
\PY{n}{std} \PY{o}{=} \PY{n}{std}\PY{o}{.}\PY{n}{reset\PYZus{}index}\PY{p}{(}\PY{n}{drop} \PY{o}{=} \PY{k+kc}{True}\PY{p}{)}
\PY{n}{std} \PY{o}{=} \PY{n}{std}\PY{o}{.}\PY{n}{sort\PYZus{}values}\PY{p}{(}\PY{n}{by} \PY{o}{=} \PY{l+s+s1}{\PYZsq{}}\PY{l+s+s1}{Stocks}\PY{l+s+s1}{\PYZsq{}}\PY{p}{)}
\PY{n}{std} \PY{o}{=} \PY{n}{std}\PY{o}{.}\PY{n}{reset\PYZus{}index}\PY{p}{(}\PY{n}{drop} \PY{o}{=} \PY{k+kc}{True}\PY{p}{)}
\PY{n}{std}\PY{o}{.}\PY{n}{head}\PY{p}{(}\PY{p}{)}
\end{Verbatim}
\end{tcolorbox}

            \begin{tcolorbox}[breakable, size=fbox, boxrule=.5pt, pad at break*=1mm, opacityfill=0]
\prompt{Out}{outcolor}{26}{\boxspacing}
\begin{Verbatim}[commandchars=\\\{\}]
  Stocks  Standard Deviation  PointsS
0   AAPL            1.052325        0
1   ABBV            1.036335        0
2    ABT            0.817759       11
3    ACN            0.828390       10
4    AIG            1.525788        0
\end{Verbatim}
\end{tcolorbox}
        
    \begin{tcolorbox}[breakable, size=fbox, boxrule=1pt, pad at break*=1mm,colback=cellbackground, colframe=cellborder]
\prompt{In}{incolor}{27}{\boxspacing}
\begin{Verbatim}[commandchars=\\\{\}]
\PY{n}{expected\PYZus{}returns} \PY{o}{=} \PY{n}{expected\PYZus{}returns}\PY{o}{.}\PY{n}{reset\PYZus{}index}\PY{p}{(}\PY{n}{drop} \PY{o}{=} \PY{k+kc}{True}\PY{p}{)}
\PY{c+c1}{\PYZsh{}Sort stocks in order alphabetical order in preparations to add the points together}
\PY{n}{expected\PYZus{}returns} \PY{o}{=} \PY{n}{expected\PYZus{}returns}\PY{o}{.}\PY{n}{sort\PYZus{}values}\PY{p}{(}\PY{n}{by} \PY{o}{=} \PY{l+s+s1}{\PYZsq{}}\PY{l+s+s1}{Stocks}\PY{l+s+s1}{\PYZsq{}}\PY{p}{)}
\PY{n}{expected\PYZus{}returns} \PY{o}{=} \PY{n}{expected\PYZus{}returns}\PY{o}{.}\PY{n}{reset\PYZus{}index}\PY{p}{(}\PY{n}{drop} \PY{o}{=} \PY{k+kc}{True}\PY{p}{)}
\PY{n}{expected\PYZus{}returns}\PY{o}{.}\PY{n}{head}\PY{p}{(}\PY{p}{)}
\end{Verbatim}
\end{tcolorbox}

            \begin{tcolorbox}[breakable, size=fbox, boxrule=.5pt, pad at break*=1mm, opacityfill=0]
\prompt{Out}{outcolor}{27}{\boxspacing}
\begin{Verbatim}[commandchars=\\\{\}]
  Stocks  Expected Returns  PointsR
0   AAPL          6.614777        0
1   ABBV          2.529781       10
2    ABT          6.219497        0
3    ACN         12.569653        0
4    AIG         20.586843        0
\end{Verbatim}
\end{tcolorbox}
        
    \begin{tcolorbox}[breakable, size=fbox, boxrule=1pt, pad at break*=1mm,colback=cellbackground, colframe=cellborder]
\prompt{In}{incolor}{28}{\boxspacing}
\begin{Verbatim}[commandchars=\\\{\}]
\PY{n}{beta} \PY{o}{=} \PY{n}{beta}\PY{o}{.}\PY{n}{reset\PYZus{}index}\PY{p}{(}\PY{n}{drop} \PY{o}{=} \PY{k+kc}{True}\PY{p}{)}
\PY{c+c1}{\PYZsh{}Sort stocks in order alphabetical order in preparations to add the points together}
\PY{n}{beta} \PY{o}{=} \PY{n}{beta}\PY{o}{.}\PY{n}{sort\PYZus{}values}\PY{p}{(}\PY{n}{by} \PY{o}{=} \PY{l+s+s1}{\PYZsq{}}\PY{l+s+s1}{Stocks}\PY{l+s+s1}{\PYZsq{}}\PY{p}{)}
\PY{n}{beta} \PY{o}{=} \PY{n}{beta}\PY{o}{.}\PY{n}{reset\PYZus{}index}\PY{p}{(}\PY{n}{drop} \PY{o}{=} \PY{k+kc}{True}\PY{p}{)}
\PY{n}{beta}\PY{o}{.}\PY{n}{head}\PY{p}{(}\PY{p}{)}
\end{Verbatim}
\end{tcolorbox}

            \begin{tcolorbox}[breakable, size=fbox, boxrule=.5pt, pad at break*=1mm, opacityfill=0]
\prompt{Out}{outcolor}{28}{\boxspacing}
\begin{Verbatim}[commandchars=\\\{\}]
  Stocks      Beta  PointsB
0   AAPL  1.327518        0
1   ABBV  0.679155        4
2    ABT  0.468943       12
3    ACN  1.031590        0
4    AIG  1.460985        0
\end{Verbatim}
\end{tcolorbox}
        
    \begin{tcolorbox}[breakable, size=fbox, boxrule=1pt, pad at break*=1mm,colback=cellbackground, colframe=cellborder]
\prompt{In}{incolor}{29}{\boxspacing}
\begin{Verbatim}[commandchars=\\\{\}]
\PY{c+c1}{\PYZsh{}Initialize dataframe with all point scores}
\PY{n}{pointsystem} \PY{o}{=} \PY{n}{pd}\PY{o}{.}\PY{n}{DataFrame}\PY{p}{(}\PY{p}{)}
\PY{c+c1}{\PYZsh{} We will be adding columns to the dataframe with all of the points from each of the different sections. }
\PY{n}{pointsystem}\PY{p}{[}\PY{l+s+s1}{\PYZsq{}}\PY{l+s+s1}{Stocks}\PY{l+s+s1}{\PYZsq{}}\PY{p}{]} \PY{o}{=} \PY{n}{beta}\PY{p}{[}\PY{l+s+s1}{\PYZsq{}}\PY{l+s+s1}{Stocks}\PY{l+s+s1}{\PYZsq{}}\PY{p}{]}
\PY{n}{pointsystem}\PY{p}{[}\PY{l+s+s1}{\PYZsq{}}\PY{l+s+s1}{PointsS}\PY{l+s+s1}{\PYZsq{}}\PY{p}{]} \PY{o}{=} \PY{n}{std}\PY{p}{[}\PY{l+s+s1}{\PYZsq{}}\PY{l+s+s1}{PointsS}\PY{l+s+s1}{\PYZsq{}}\PY{p}{]}
\PY{n}{pointsystem}\PY{p}{[}\PY{l+s+s1}{\PYZsq{}}\PY{l+s+s1}{PointsR}\PY{l+s+s1}{\PYZsq{}}\PY{p}{]} \PY{o}{=} \PY{n}{expected\PYZus{}returns}\PY{p}{[}\PY{l+s+s1}{\PYZsq{}}\PY{l+s+s1}{PointsR}\PY{l+s+s1}{\PYZsq{}}\PY{p}{]}
\PY{n}{pointsystem}\PY{p}{[}\PY{l+s+s1}{\PYZsq{}}\PY{l+s+s1}{PointsB}\PY{l+s+s1}{\PYZsq{}}\PY{p}{]} \PY{o}{=} \PY{n}{beta}\PY{p}{[}\PY{l+s+s1}{\PYZsq{}}\PY{l+s+s1}{PointsB}\PY{l+s+s1}{\PYZsq{}}\PY{p}{]}
\PY{c+c1}{\PYZsh{} Summing the points to get the total points.}
\PY{n}{pointsystem}\PY{p}{[}\PY{l+s+s1}{\PYZsq{}}\PY{l+s+s1}{TotalPoints}\PY{l+s+s1}{\PYZsq{}}\PY{p}{]} \PY{o}{=} \PY{n}{pointsystem}\PY{p}{[}\PY{l+s+s1}{\PYZsq{}}\PY{l+s+s1}{PointsS}\PY{l+s+s1}{\PYZsq{}}\PY{p}{]} \PY{o}{+} \PY{n}{pointsystem}\PY{p}{[}\PY{l+s+s1}{\PYZsq{}}\PY{l+s+s1}{PointsR}\PY{l+s+s1}{\PYZsq{}}\PY{p}{]} \PY{o}{+} \PY{n}{pointsystem}\PY{p}{[}\PY{l+s+s1}{\PYZsq{}}\PY{l+s+s1}{PointsB}\PY{l+s+s1}{\PYZsq{}}\PY{p}{]}
\PY{n}{pointsystem}\PY{o}{.}\PY{n}{head}\PY{p}{(}\PY{p}{)}
\end{Verbatim}
\end{tcolorbox}

            \begin{tcolorbox}[breakable, size=fbox, boxrule=.5pt, pad at break*=1mm, opacityfill=0]
\prompt{Out}{outcolor}{29}{\boxspacing}
\begin{Verbatim}[commandchars=\\\{\}]
  Stocks  PointsS  PointsR  PointsB  TotalPoints
0   AAPL        0        0        0            0
1   ABBV        0       10        4           14
2    ABT       11        0       12           23
3    ACN       10        0        0           10
4    AIG        0        0        0            0
\end{Verbatim}
\end{tcolorbox}
        
    \begin{tcolorbox}[breakable, size=fbox, boxrule=1pt, pad at break*=1mm,colback=cellbackground, colframe=cellborder]
\prompt{In}{incolor}{30}{\boxspacing}
\begin{Verbatim}[commandchars=\\\{\}]
\PY{n}{pointsystem} \PY{o}{=} \PY{n}{pointsystem}\PY{o}{.}\PY{n}{reset\PYZus{}index}\PY{p}{(}\PY{n}{drop} \PY{o}{=} \PY{k+kc}{True}\PY{p}{)}
\PY{c+c1}{\PYZsh{}Sort all point columns in order from greatest to least value}
\PY{n}{pointsystem} \PY{o}{=} \PY{n}{pointsystem}\PY{o}{.}\PY{n}{sort\PYZus{}values}\PY{p}{(}\PY{n}{by} \PY{o}{=} \PY{l+s+s1}{\PYZsq{}}\PY{l+s+s1}{TotalPoints}\PY{l+s+s1}{\PYZsq{}}\PY{p}{,} \PY{n}{ascending} \PY{o}{=} \PY{k+kc}{False}\PY{p}{)}
\PY{n}{pointsystem} \PY{o}{=} \PY{n}{pointsystem}\PY{o}{.}\PY{n}{reset\PYZus{}index}\PY{p}{(}\PY{n}{drop} \PY{o}{=} \PY{k+kc}{True}\PY{p}{)}
\PY{n}{pointsystem}
\end{Verbatim}
\end{tcolorbox}

            \begin{tcolorbox}[breakable, size=fbox, boxrule=.5pt, pad at break*=1mm, opacityfill=0]
\prompt{Out}{outcolor}{30}{\boxspacing}
\begin{Verbatim}[commandchars=\\\{\}]
   Stocks  PointsS  PointsR  PointsB  TotalPoints
0     MON       20       18       20           58
1      KO       18       16       10           44
2      PG       19        6       19           44
3     LMT       14       17        9           40
4      SO       16        0       18           34
5     PEP       17        0       16           33
6      CL       13        0       17           30
7    SBUX        4       20        0           24
8     ABT       11        0       12           23
9      PM        2       13        6           21
10    NEE        6        0       14           20
11      T       15        0        5           20
12     GM        0       19        0           19
13    BMY       12        0        7           19
14    MRK        0        2       13           15
15   BIIB        0        0       15           15
16   AMZN        0       15        0           15
17      C        0       14        0           14
18   ABBV        0       10        4           14
19    NKE        0       12        0           12
20    LLY        0        4        8           12
21   COST        9        0        3           12
22    PFE        0        0       11           11
23    BLK        0       11        0           11
24    ACN       10        0        0           10
25    KMI        0        9        0            9
26     MO        0        8        0            8
27    TGT        1        7        0            8
28   CSCO        8        0        0            8
29   MSFT        7        0        0            7
30    CVS        5        0        0            5
31    UPS        0        3        2            5
32    UNP        0        5        0            5
33   GOOG        3        0        0            3
34   ORCL        0        0        1            1
35    OXY        0        1        0            1
36   PYPL        0        0        0            0
37    UNH        0        0        0            0
38   QCOM        0        0        0            0
39    SPG        0        0        0            0
40    SLB        0        0        0            0
41    TXN        0        0        0            0
42   AAPL        0        0        0            0
43     MS        0        0        0            0
44    JPM        0        0        0            0
45    COP        0        0        0            0
46    COF        0        0        0            0
47  CMCSA        0        0        0            0
48    CAT        0        0        0            0
49     BK        0        0        0            0
50    BAC        0        0        0            0
51     BA        0        0        0            0
52    AXP        0        0        0            0
53    AIG        0        0        0            0
54    USB        0        0        0            0
\end{Verbatim}
\end{tcolorbox}
        
    \hypertarget{choosing-the-stocks-for-the-portfolio}{%
\subsection{Choosing the stocks for the
Portfolio:}\label{choosing-the-stocks-for-the-portfolio}}

After finding the total points of each of the stocks in our list of
stocks, we must then use the point system to choose the maximum number
of stocks, up to a maximum number of 20, that we wish to make up our
portfolio and then also use these points to determine the weight of each
stock in our portfolio. Thus, we will choose the top 20 (or if there are
less than 20 valid stocks then the number of valid of stocks) stocks
that have the highest point totals and these will be the stocks that
will make up our portfolio.

    \begin{tcolorbox}[breakable, size=fbox, boxrule=1pt, pad at break*=1mm,colback=cellbackground, colframe=cellborder]
\prompt{In}{incolor}{31}{\boxspacing}
\begin{Verbatim}[commandchars=\\\{\}]
\PY{c+c1}{\PYZsh{}Only take the top 20 valid stocks to make a portfolio}
\PY{n}{pointsystem} \PY{o}{=} \PY{n}{pointsystem}\PY{p}{[}\PY{p}{:}\PY{l+m+mi}{20}\PY{p}{]} \PY{c+c1}{\PYZsh{} same as df.head(20)}
\PY{c+c1}{\PYZsh{}Return the portfolio of 20 stocks}
\PY{n}{pointsystem}
\end{Verbatim}
\end{tcolorbox}

            \begin{tcolorbox}[breakable, size=fbox, boxrule=.5pt, pad at break*=1mm, opacityfill=0]
\prompt{Out}{outcolor}{31}{\boxspacing}
\begin{Verbatim}[commandchars=\\\{\}]
   Stocks  PointsS  PointsR  PointsB  TotalPoints
0     MON       20       18       20           58
1      KO       18       16       10           44
2      PG       19        6       19           44
3     LMT       14       17        9           40
4      SO       16        0       18           34
5     PEP       17        0       16           33
6      CL       13        0       17           30
7    SBUX        4       20        0           24
8     ABT       11        0       12           23
9      PM        2       13        6           21
10    NEE        6        0       14           20
11      T       15        0        5           20
12     GM        0       19        0           19
13    BMY       12        0        7           19
14    MRK        0        2       13           15
15   BIIB        0        0       15           15
16   AMZN        0       15        0           15
17      C        0       14        0           14
18   ABBV        0       10        4           14
19    NKE        0       12        0           12
\end{Verbatim}
\end{tcolorbox}
        
    \begin{tcolorbox}[breakable, size=fbox, boxrule=1pt, pad at break*=1mm,colback=cellbackground, colframe=cellborder]
\prompt{In}{incolor}{32}{\boxspacing}
\begin{Verbatim}[commandchars=\\\{\}]
\PY{c+c1}{\PYZsh{}Make list of stocks that will be in our portfolio.}
\PY{n}{portfoliostocks} \PY{o}{=} \PY{n}{pointsystem}\PY{p}{[}\PY{l+s+s2}{\PYZdq{}}\PY{l+s+s2}{Stocks}\PY{l+s+s2}{\PYZdq{}}\PY{p}{]}\PY{o}{.}\PY{n}{values}\PY{o}{.}\PY{n}{tolist}\PY{p}{(}\PY{p}{)}
\PY{n+nb}{print}\PY{p}{(}\PY{n}{portfoliostocks}\PY{p}{)}
\end{Verbatim}
\end{tcolorbox}

    \begin{Verbatim}[commandchars=\\\{\}]
['MON', 'KO', 'PG', 'LMT', 'SO', 'PEP', 'CL', 'SBUX', 'ABT', 'PM', 'NEE', 'T',
'GM', 'BMY', 'MRK', 'BIIB', 'AMZN', 'C', 'ABBV', 'NKE']
    \end{Verbatim}

    \hypertarget{correlation}{%
\subsection{Correlation:}\label{correlation}}

One final factor to be considered is the correlation of the stocks with
the rest of the stocks in our portfolio. Correaltion is a measure of how
securities move in relation to one another. Since we want our portfolio
to be safe and have lower returns, one method to minimize the expected
returns of the portfolio would be to place greater weighting on stocks
that have a negative correlation with the rest of the portfolio. For
instance, if a stock exhibits a negative correlation with a portfolio
that contains the rest of the stocks, then this indicates that when the
overall returns of the portfolio goes up, then the returns of that
portfolio goes down. When the correlations between a stock and the
portfolio of stocks is relatively low, then placing greater weighting on
these stocks will minimize the risk of the overall portfolio. By
increasing the weighting on stocks (by allocating points) that have a
relatively low correlation with the portfolio of 20 stocks that we have
chosen, then when the portfolio increases or decreases in value, then
the low or negatively correlated stock will either move in the opposite
direction of the portfolio or will only move slightly in the same
direction with the rest of the portfolio.

In this section, we will be calculating the correlation of each of the
stocks with the portfolio that contains the stocks that we have chosen
using the point system before. The reasoning behind why we are doing the
correlation after choosing the stocks in our portfolio is because we are
only interested in weighing the stocks that have a low correlation with
the rest of the stocks we have chosen. Generally, in statistics, stocks
that have a correlation value between 0.3-0.5 is considered to be a low
correlation
(https://www.researchgate.net/post/What-are-the-correlation-values-with-respect-to-low-moderate-high-correlation-specially-in-medical-research\#:\textasciitilde:text=Correlation\%20coefficients\%20whose\%20magnitude\%20are\%20between\%200.3\%20and\%200.5\%20indicate,if\%20any\%20(linear)\%20correlation.)
. Consequently, for each stock in our portfolio that has a correlation
value less than 0.5 with our portfolio will be assigned a flat number of
10 points.

    \begin{tcolorbox}[breakable, size=fbox, boxrule=1pt, pad at break*=1mm,colback=cellbackground, colframe=cellborder]
\prompt{In}{incolor}{33}{\boxspacing}
\begin{Verbatim}[commandchars=\\\{\}]
\PY{c+c1}{\PYZsh{} This is the dataframe with all of the closing prices of our list of stocks}
\PY{n}{closing\PYZus{}portfolio} \PY{o}{=} \PY{n}{get\PYZus{}closing\PYZus{}prices}\PY{p}{(}\PY{n}{portfoliostocks}\PY{p}{,} \PY{n}{start\PYZus{}date}\PY{p}{,} \PY{n}{end\PYZus{}date}\PY{p}{)}
\PY{n}{comparison\PYZus{}portfolio} \PY{o}{=} \PY{n}{get\PYZus{}data}\PY{p}{(}\PY{n}{portfoliostocks}\PY{p}{,} \PY{l+m+mi}{100000}\PY{p}{,} \PY{n}{closing\PYZus{}portfolio}\PY{p}{)}\PY{o}{.}\PY{n}{iloc}\PY{p}{[}\PY{l+m+mi}{1}\PY{p}{:}\PY{p}{]}
\PY{n}{comparison\PYZus{}portfolio} \PY{o}{=} \PY{n}{comparison\PYZus{}portfolio}\PY{p}{[}\PY{l+s+s2}{\PYZdq{}}\PY{l+s+s2}{Returns}\PY{l+s+s2}{\PYZdq{}}\PY{p}{]}
\PY{c+c1}{\PYZsh{} Initializes the dataframe with the columns we want. }
\PY{n}{correlations\PYZus{}df} \PY{o}{=} \PY{n}{pd}\PY{o}{.}\PY{n}{DataFrame}\PY{p}{(}\PY{p}{\PYZob{}}\PY{l+s+s1}{\PYZsq{}}\PY{l+s+s1}{Stock}\PY{l+s+s1}{\PYZsq{}}\PY{p}{:} \PY{p}{[}\PY{p}{]}\PY{p}{,} \PY{l+s+s2}{\PYZdq{}}\PY{l+s+s2}{Correlation}\PY{l+s+s2}{\PYZdq{}} \PY{p}{:} \PY{p}{[}\PY{p}{]}\PY{p}{,} \PY{l+s+s2}{\PYZdq{}}\PY{l+s+s2}{PointsC}\PY{l+s+s2}{\PYZdq{}}\PY{p}{:} \PY{p}{[}\PY{p}{]}\PY{p}{\PYZcb{}}\PY{p}{)}

\PY{c+c1}{\PYZsh{} This for loop will calculate the correlation of each stock with the portfolio that contains all of the stocks. If the correlation value of the stock is less than 0.5, then it will append it to the dataframe known as correlations\PYZus{}df. }
\PY{k}{for} \PY{n}{i} \PY{o+ow}{in} \PY{n+nb}{range}\PY{p}{(}\PY{n+nb}{len}\PY{p}{(}\PY{n}{portfoliostocks}\PY{p}{)}\PY{p}{)}\PY{p}{:}
    \PY{n}{stocks\PYZus{}close\PYZus{}df} \PY{o}{=} \PY{n}{pd}\PY{o}{.}\PY{n}{DataFrame}\PY{p}{(}\PY{p}{)}
    \PY{n}{stock} \PY{o}{=} \PY{n}{yf}\PY{o}{.}\PY{n}{Ticker}\PY{p}{(}\PY{n}{portfoliostocks}\PY{p}{[}\PY{n}{i}\PY{p}{]}\PY{p}{)}
    \PY{n}{stock\PYZus{}historical} \PY{o}{=} \PY{n}{stock}\PY{o}{.}\PY{n}{history}\PY{p}{(}\PY{n}{start}\PY{o}{=} \PY{n}{start\PYZus{}date} \PY{p}{,} \PY{n}{end} \PY{o}{=} \PY{n}{end\PYZus{}date} \PY{p}{,} \PY{n}{threads}\PY{o}{=}\PY{k+kc}{True}\PY{p}{)}
    \PY{n}{stocks\PYZus{}close\PYZus{}df}\PY{p}{[}\PY{n}{portfoliostocks}\PY{p}{[}\PY{n}{i}\PY{p}{]}\PY{p}{]} \PY{o}{=} \PY{n}{stock\PYZus{}historical}\PY{p}{[}\PY{l+s+s2}{\PYZdq{}}\PY{l+s+s2}{Close}\PY{l+s+s2}{\PYZdq{}}\PY{p}{]}
    \PY{n}{stocks\PYZus{}close\PYZus{}returns} \PY{o}{=} \PY{n}{stocks\PYZus{}close\PYZus{}df}\PY{o}{.}\PY{n}{resample}\PY{p}{(}\PY{l+s+s2}{\PYZdq{}}\PY{l+s+s2}{D}\PY{l+s+s2}{\PYZdq{}}\PY{p}{)}\PY{o}{.}\PY{n}{first}\PY{p}{(}\PY{p}{)}\PY{o}{.}\PY{n}{pct\PYZus{}change}\PY{p}{(}\PY{p}{)}\PY{o}{.}\PY{n}{iloc}\PY{p}{[}\PY{l+m+mi}{1}\PY{p}{:}\PY{p}{]}
    \PY{n}{portfolio} \PY{o}{=} \PY{n}{pd}\PY{o}{.}\PY{n}{concat}\PY{p}{(}\PY{p}{[}\PY{n}{comparison\PYZus{}portfolio}\PY{p}{,} \PY{n}{stocks\PYZus{}close\PYZus{}returns}\PY{p}{]}\PY{p}{,} \PY{n}{join} \PY{o}{=} \PY{l+s+s2}{\PYZdq{}}\PY{l+s+s2}{outer}\PY{l+s+s2}{\PYZdq{}}\PY{p}{,} \PY{n}{axis} \PY{o}{=} \PY{l+m+mi}{1}\PY{p}{)}
    \PY{n}{correlation} \PY{o}{=} \PY{n}{portfolio}\PY{o}{.}\PY{n}{corr}\PY{p}{(}\PY{p}{)}\PY{o}{.}\PY{n}{iat}\PY{p}{[}\PY{l+m+mi}{0}\PY{p}{,}\PY{l+m+mi}{1}\PY{p}{]}
    \PY{n}{stock\PYZus{}name} \PY{o}{=} \PY{n}{portfoliostocks}\PY{p}{[}\PY{n}{i}\PY{p}{]}
    \PY{k}{if} \PY{n}{correlation} \PY{o}{\PYZlt{}} \PY{l+m+mf}{0.5}\PY{p}{:}
        
        \PY{n}{stock\PYZus{}data} \PY{o}{=} \PY{n}{pd}\PY{o}{.}\PY{n}{DataFrame}\PY{p}{(}\PY{p}{\PYZob{}}\PY{l+s+s1}{\PYZsq{}}\PY{l+s+s1}{Stock}\PY{l+s+s1}{\PYZsq{}}\PY{p}{:} \PY{p}{[}\PY{n}{stock\PYZus{}name}\PY{p}{]}\PY{p}{,}
                                   \PY{l+s+s1}{\PYZsq{}}\PY{l+s+s1}{Correlation}\PY{l+s+s1}{\PYZsq{}}\PY{p}{:} \PY{p}{[}\PY{n}{correlation}\PY{p}{]}\PY{p}{,}
                                   \PY{l+s+s1}{\PYZsq{}}\PY{l+s+s1}{PointsC}\PY{l+s+s1}{\PYZsq{}}\PY{p}{:} \PY{p}{[}\PY{l+m+mi}{10}\PY{p}{]}\PY{p}{\PYZcb{}}\PY{p}{)}
        \PY{n}{correlations\PYZus{}df} \PY{o}{=} \PY{n}{correlations\PYZus{}df}\PY{o}{.}\PY{n}{append}\PY{p}{(}\PY{n}{stock\PYZus{}data}\PY{p}{)}
    \PY{k}{else}\PY{p}{:}
        \PY{n}{stock\PYZus{}data} \PY{o}{=} \PY{n}{pd}\PY{o}{.}\PY{n}{DataFrame}\PY{p}{(}\PY{p}{\PYZob{}}\PY{l+s+s1}{\PYZsq{}}\PY{l+s+s1}{Stock}\PY{l+s+s1}{\PYZsq{}}\PY{p}{:} \PY{p}{[}\PY{n}{stock\PYZus{}name}\PY{p}{]}\PY{p}{,}
                                   \PY{l+s+s1}{\PYZsq{}}\PY{l+s+s1}{Correlation}\PY{l+s+s1}{\PYZsq{}}\PY{p}{:} \PY{p}{[}\PY{n}{correlation}\PY{p}{]}\PY{p}{,}
                                   \PY{l+s+s1}{\PYZsq{}}\PY{l+s+s1}{PointsC}\PY{l+s+s1}{\PYZsq{}}\PY{p}{:} \PY{p}{[}\PY{l+m+mi}{0}\PY{p}{]}\PY{p}{\PYZcb{}}\PY{p}{)}
        \PY{n}{correlations\PYZus{}df} \PY{o}{=} \PY{n}{correlations\PYZus{}df}\PY{o}{.}\PY{n}{append}\PY{p}{(}\PY{n}{stock\PYZus{}data}\PY{p}{)}
\PY{c+c1}{\PYZsh{} Resetting the index after appending}
\PY{n}{correlations\PYZus{}df} \PY{o}{=} \PY{n}{correlations\PYZus{}df}\PY{o}{.}\PY{n}{reset\PYZus{}index}\PY{p}{(}\PY{p}{)}
\PY{c+c1}{\PYZsh{} Creating a new dataframe that simply drops an invalid column}
\PY{n}{correlations\PYZus{}final} \PY{o}{=} \PY{n}{correlations\PYZus{}df}\PY{o}{.}\PY{n}{drop}\PY{p}{(}\PY{n}{columns} \PY{o}{=} \PY{p}{[}\PY{l+s+s1}{\PYZsq{}}\PY{l+s+s1}{index}\PY{l+s+s1}{\PYZsq{}}\PY{p}{]}\PY{p}{)}
\PY{n}{correlations\PYZus{}final} \PY{o}{=} \PY{n}{correlations\PYZus{}final}\PY{o}{.}\PY{n}{sort\PYZus{}values}\PY{p}{(}\PY{n}{by} \PY{o}{=} \PY{p}{[}\PY{l+s+s2}{\PYZdq{}}\PY{l+s+s2}{Stock}\PY{l+s+s2}{\PYZdq{}}\PY{p}{]}\PY{p}{,} \PY{n}{ascending} \PY{o}{=} \PY{k+kc}{True}\PY{p}{)}
\PY{n}{correlations\PYZus{}final} \PY{o}{=} \PY{n}{correlations\PYZus{}final}\PY{o}{.}\PY{n}{reset\PYZus{}index}\PY{p}{(}\PY{n}{drop} \PY{o}{=} \PY{k+kc}{True}\PY{p}{)}
\PY{n}{correlations\PYZus{}final}
\end{Verbatim}
\end{tcolorbox}

            \begin{tcolorbox}[breakable, size=fbox, boxrule=.5pt, pad at break*=1mm, opacityfill=0]
\prompt{Out}{outcolor}{33}{\boxspacing}
\begin{Verbatim}[commandchars=\\\{\}]
   Stock  Correlation  PointsC
0   ABBV     0.557138      0.0
1    ABT     0.514683      0.0
2   AMZN     0.395677     10.0
3   BIIB     0.333288     10.0
4    BMY     0.633669      0.0
5      C     0.476108     10.0
6     CL     0.596817      0.0
7     GM     0.395425     10.0
8     KO     0.724464      0.0
9    LMT     0.483262     10.0
10   MON     0.048611     10.0
11   MRK     0.469656     10.0
12   NEE     0.470693     10.0
13   NKE     0.434801     10.0
14   PEP     0.583583      0.0
15    PG     0.472865     10.0
16    PM     0.558614      0.0
17  SBUX     0.516925      0.0
18    SO     0.556674      0.0
19     T     0.648968      0.0
\end{Verbatim}
\end{tcolorbox}
        
    \hypertarget{correlation-discussion}{%
\subsection{Correlation Discussion:}\label{correlation-discussion}}

Using the code above, we were able to find all of the stocks in our
portfolio that have a correlation value of less than 0.5 with our
portfolio. We then awarded points to all of these stocks that have these
lower correlation values so that when we use these points to weigh the
stocks, the stocks that have a lower correlation value will have a
greater weighting.

    \hypertarget{recalculating-the-points-with-the-correlation-points-now-added}{%
\subsection{Recalculating the Points with the Correlation Points now
added}\label{recalculating-the-points-with-the-correlation-points-now-added}}

    We will now be readding all of the points together.

    \begin{tcolorbox}[breakable, size=fbox, boxrule=1pt, pad at break*=1mm,colback=cellbackground, colframe=cellborder]
\prompt{In}{incolor}{34}{\boxspacing}
\begin{Verbatim}[commandchars=\\\{\}]
\PY{c+c1}{\PYZsh{} We can resort and and now add the correlation points by dropping the total points column, adding in the correlation points column and then readding in the total points columns after summing all of the points.}
\PY{n}{pointsystem} \PY{o}{=} \PY{n}{pointsystem}\PY{o}{.}\PY{n}{sort\PYZus{}values}\PY{p}{(}\PY{n}{by} \PY{o}{=} 
                            \PY{p}{[}\PY{l+s+s2}{\PYZdq{}}\PY{l+s+s2}{Stocks}\PY{l+s+s2}{\PYZdq{}}\PY{p}{]}\PY{p}{,}
                            \PY{n}{ascending} \PY{o}{=} \PY{k+kc}{True}\PY{p}{)}
\PY{n}{pointsystem} \PY{o}{=} \PY{n}{pointsystem}\PY{o}{.}\PY{n}{reset\PYZus{}index}\PY{p}{(}\PY{n}{drop} \PY{o}{=} \PY{k+kc}{True}\PY{p}{)}
\PY{n}{pointsystem} \PY{o}{=} \PY{n}{pointsystem}\PY{o}{.}\PY{n}{drop}\PY{p}{(}\PY{n}{columns} \PY{o}{=} \PY{p}{[}\PY{l+s+s1}{\PYZsq{}}\PY{l+s+s1}{TotalPoints}\PY{l+s+s1}{\PYZsq{}}\PY{p}{]}\PY{p}{)}
\PY{c+c1}{\PYZsh{} Adding in the points}
\PY{n}{pointsystem}\PY{p}{[}\PY{l+s+s2}{\PYZdq{}}\PY{l+s+s2}{PointsC}\PY{l+s+s2}{\PYZdq{}}\PY{p}{]} \PY{o}{=} \PY{n}{correlations\PYZus{}final}\PY{p}{[}\PY{l+s+s2}{\PYZdq{}}\PY{l+s+s2}{PointsC}\PY{l+s+s2}{\PYZdq{}}\PY{p}{]}
\PY{n}{pointsystem}\PY{p}{[}\PY{l+s+s1}{\PYZsq{}}\PY{l+s+s1}{TotalPoints}\PY{l+s+s1}{\PYZsq{}}\PY{p}{]} \PY{o}{=} \PY{n}{pointsystem}\PY{p}{[}\PY{l+s+s1}{\PYZsq{}}\PY{l+s+s1}{PointsS}\PY{l+s+s1}{\PYZsq{}}\PY{p}{]} \PY{o}{+} \PY{n}{pointsystem}\PY{p}{[}\PY{l+s+s1}{\PYZsq{}}\PY{l+s+s1}{PointsR}\PY{l+s+s1}{\PYZsq{}}\PY{p}{]} \PY{o}{+} \PY{n}{pointsystem}\PY{p}{[}\PY{l+s+s1}{\PYZsq{}}\PY{l+s+s1}{PointsB}\PY{l+s+s1}{\PYZsq{}}\PY{p}{]} \PY{o}{+} \PY{n}{pointsystem}\PY{p}{[}\PY{l+s+s2}{\PYZdq{}}\PY{l+s+s2}{PointsC}\PY{l+s+s2}{\PYZdq{}}\PY{p}{]}
\PY{n}{pointsystem} \PY{o}{=} \PY{n}{pointsystem}\PY{o}{.}\PY{n}{sort\PYZus{}values}\PY{p}{(}\PY{n}{by} \PY{o}{=} \PY{p}{[}\PY{l+s+s2}{\PYZdq{}}\PY{l+s+s2}{TotalPoints}\PY{l+s+s2}{\PYZdq{}}\PY{p}{]}\PY{p}{,}\PY{n}{ascending} \PY{o}{=} \PY{k+kc}{False}\PY{p}{)}
\PY{n}{pointsystem} \PY{o}{=} \PY{n}{pointsystem}\PY{o}{.}\PY{n}{reset\PYZus{}index}\PY{p}{(}\PY{n}{drop} \PY{o}{=} \PY{k+kc}{True}\PY{p}{)}
\PY{n}{pointsystem}
\end{Verbatim}
\end{tcolorbox}

            \begin{tcolorbox}[breakable, size=fbox, boxrule=.5pt, pad at break*=1mm, opacityfill=0]
\prompt{Out}{outcolor}{34}{\boxspacing}
\begin{Verbatim}[commandchars=\\\{\}]
   Stocks  PointsS  PointsR  PointsB  PointsC  TotalPoints
0     MON       20       18       20     10.0         68.0
1      PG       19        6       19     10.0         54.0
2     LMT       14       17        9     10.0         50.0
3      KO       18       16       10      0.0         44.0
4      SO       16        0       18      0.0         34.0
5     PEP       17        0       16      0.0         33.0
6      CL       13        0       17      0.0         30.0
7     NEE        6        0       14     10.0         30.0
8      GM        0       19        0     10.0         29.0
9    BIIB        0        0       15     10.0         25.0
10    MRK        0        2       13     10.0         25.0
11   AMZN        0       15        0     10.0         25.0
12      C        0       14        0     10.0         24.0
13   SBUX        4       20        0      0.0         24.0
14    ABT       11        0       12      0.0         23.0
15    NKE        0       12        0     10.0         22.0
16     PM        2       13        6      0.0         21.0
17      T       15        0        5      0.0         20.0
18    BMY       12        0        7      0.0         19.0
19   ABBV        0       10        4      0.0         14.0
\end{Verbatim}
\end{tcolorbox}
        
    \hypertarget{calculating-the-weightings}{%
\subsection{Calculating the
Weightings}\label{calculating-the-weightings}}

In order to calculate the weightings, we will first start by assigning
the miniumum base weighting of each stock, which is 100/(2n), where n is
the number of stocks in our portfolio. After assigning the minimum base
weightings, we will then find the total sum of the total points. We will
the find the \% of the points of each of the stocks and relative to the
total sum of the total points. This will be our additional weightings
and will be added onto the base weightings of each of the stocks. These
will be the weighting of our stocks within our portfolio.

    \begin{tcolorbox}[breakable, size=fbox, boxrule=1pt, pad at break*=1mm,colback=cellbackground, colframe=cellborder]
\prompt{In}{incolor}{35}{\boxspacing}
\begin{Verbatim}[commandchars=\\\{\}]
\PY{c+c1}{\PYZsh{}Calculate total points for weighting metrics}
\PY{n}{total\PYZus{}points} \PY{o}{=} \PY{n}{pointsystem}\PY{p}{[}\PY{l+s+s2}{\PYZdq{}}\PY{l+s+s2}{TotalPoints}\PY{l+s+s2}{\PYZdq{}}\PY{p}{]}\PY{o}{.}\PY{n}{sum}\PY{p}{(}\PY{n}{axis}\PY{o}{=}\PY{l+m+mi}{0}\PY{p}{)}
\PY{n}{total\PYZus{}points}
\end{Verbatim}
\end{tcolorbox}

            \begin{tcolorbox}[breakable, size=fbox, boxrule=.5pt, pad at break*=1mm, opacityfill=0]
\prompt{Out}{outcolor}{35}{\boxspacing}
\begin{Verbatim}[commandchars=\\\{\}]
614.0
\end{Verbatim}
\end{tcolorbox}
        
    \begin{tcolorbox}[breakable, size=fbox, boxrule=1pt, pad at break*=1mm,colback=cellbackground, colframe=cellborder]
\prompt{In}{incolor}{36}{\boxspacing}
\begin{Verbatim}[commandchars=\\\{\}]
\PY{c+c1}{\PYZsh{}Initialize a dataframe for calculating weights}
\PY{n}{base\PYZus{}weighting} \PY{o}{=} \PY{n}{pd}\PY{o}{.}\PY{n}{DataFrame}\PY{p}{(}\PY{p}{)}

\PY{c+c1}{\PYZsh{}Initialize all instances with the same weighting value (minimum 2.5\PYZpc{})}
\PY{k}{for} \PY{n}{i} \PY{o+ow}{in} \PY{n+nb}{range}\PY{p}{(}\PY{n+nb}{len}\PY{p}{(}\PY{n}{portfoliostocks}\PY{p}{)}\PY{p}{)}\PY{p}{:}
    \PY{n}{base\PYZus{}weighting}\PY{o}{.}\PY{n}{loc}\PY{p}{[}\PY{l+m+mi}{0}\PY{p}{,} \PY{n}{portfoliostocks}\PY{p}{[}\PY{n}{i}\PY{p}{]}\PY{p}{]} \PY{o}{=} \PY{l+m+mi}{100}\PY{o}{/}\PY{p}{(}\PY{l+m+mi}{2} \PY{o}{*} \PY{p}{(}\PY{n+nb}{len}\PY{p}{(}\PY{n}{portfoliostocks}\PY{p}{)}\PY{p}{)}\PY{p}{)}

\PY{n}{base\PYZus{}weighting} \PY{o}{=} \PY{n}{base\PYZus{}weighting}\PY{o}{.}\PY{n}{T}
\PY{n}{base\PYZus{}weighting} \PY{o}{=} \PY{n}{base\PYZus{}weighting}\PY{o}{.}\PY{n}{rename}\PY{p}{(}\PY{n}{columns}\PY{o}{=}\PY{p}{\PYZob{}}\PY{l+s+s2}{\PYZdq{}}\PY{l+s+s2}{index}\PY{l+s+s2}{\PYZdq{}}\PY{p}{:} \PY{l+s+s2}{\PYZdq{}}\PY{l+s+s2}{stocks}\PY{l+s+s2}{\PYZdq{}}\PY{p}{,} \PY{l+m+mi}{0}\PY{p}{:} \PY{l+s+s2}{\PYZdq{}}\PY{l+s+s2}{Weighting}\PY{l+s+s2}{\PYZdq{}}\PY{p}{\PYZcb{}}\PY{p}{)}
\PY{n}{base\PYZus{}weighting}
\end{Verbatim}
\end{tcolorbox}

            \begin{tcolorbox}[breakable, size=fbox, boxrule=.5pt, pad at break*=1mm, opacityfill=0]
\prompt{Out}{outcolor}{36}{\boxspacing}
\begin{Verbatim}[commandchars=\\\{\}]
      Weighting
MON         2.5
KO          2.5
PG          2.5
LMT         2.5
SO          2.5
PEP         2.5
CL          2.5
SBUX        2.5
ABT         2.5
PM          2.5
NEE         2.5
T           2.5
GM          2.5
BMY         2.5
MRK         2.5
BIIB        2.5
AMZN        2.5
C           2.5
ABBV        2.5
NKE         2.5
\end{Verbatim}
\end{tcolorbox}
        
    \begin{tcolorbox}[breakable, size=fbox, boxrule=1pt, pad at break*=1mm,colback=cellbackground, colframe=cellborder]
\prompt{In}{incolor}{37}{\boxspacing}
\begin{Verbatim}[commandchars=\\\{\}]
\PY{c+c1}{\PYZsh{}Target weight value vs. weight acheived }
\PY{n}{total\PYZus{}weight} \PY{o}{=} \PY{l+m+mi}{100}
\PY{n}{total\PYZus{}already\PYZus{}weighted} \PY{o}{=} \PY{n}{base\PYZus{}weighting}\PY{p}{[}\PY{l+s+s2}{\PYZdq{}}\PY{l+s+s2}{Weighting}\PY{l+s+s2}{\PYZdq{}}\PY{p}{]}\PY{o}{.}\PY{n}{sum}\PY{p}{(}\PY{n}{axis}\PY{o}{=}\PY{l+m+mi}{0}\PY{p}{)}
\PY{n}{total\PYZus{}already\PYZus{}weighted}
\end{Verbatim}
\end{tcolorbox}

            \begin{tcolorbox}[breakable, size=fbox, boxrule=.5pt, pad at break*=1mm, opacityfill=0]
\prompt{Out}{outcolor}{37}{\boxspacing}
\begin{Verbatim}[commandchars=\\\{\}]
50.0
\end{Verbatim}
\end{tcolorbox}
        
    \begin{tcolorbox}[breakable, size=fbox, boxrule=1pt, pad at break*=1mm,colback=cellbackground, colframe=cellborder]
\prompt{In}{incolor}{38}{\boxspacing}
\begin{Verbatim}[commandchars=\\\{\}]
\PY{c+c1}{\PYZsh{}Weight to be manipulated to achieve 100\PYZpc{} weighting }
\PY{n}{manipulate\PYZus{}weight} \PY{o}{=} \PY{n}{total\PYZus{}weight} \PY{o}{\PYZhy{}} \PY{n}{total\PYZus{}already\PYZus{}weighted}
\PY{n}{manipulate\PYZus{}weight}
\end{Verbatim}
\end{tcolorbox}

            \begin{tcolorbox}[breakable, size=fbox, boxrule=.5pt, pad at break*=1mm, opacityfill=0]
\prompt{Out}{outcolor}{38}{\boxspacing}
\begin{Verbatim}[commandchars=\\\{\}]
50.0
\end{Verbatim}
\end{tcolorbox}
        
    \begin{tcolorbox}[breakable, size=fbox, boxrule=1pt, pad at break*=1mm,colback=cellbackground, colframe=cellborder]
\prompt{In}{incolor}{39}{\boxspacing}
\begin{Verbatim}[commandchars=\\\{\}]
\PY{c+c1}{\PYZsh{}Initialize list for remaining 50\PYZpc{} weightage}
\PY{n}{further\PYZus{}weights} \PY{o}{=} \PY{p}{[}\PY{p}{]}
\PY{k}{for} \PY{n}{i} \PY{o+ow}{in} \PY{n+nb}{range} \PY{p}{(}\PY{n+nb}{len}\PY{p}{(}\PY{n}{pointsystem}\PY{p}{)}\PY{p}{)}\PY{p}{:}
    \PY{c+c1}{\PYZsh{}Points for the stock if it were out of 100}
    \PY{n}{out\PYZus{}of\PYZus{}100} \PY{o}{=} \PY{n}{pointsystem}\PY{o}{.}\PY{n}{iloc}\PY{p}{[}\PY{n}{i}\PY{p}{,} \PY{l+m+mi}{5}\PY{p}{]}\PY{o}{/}\PY{n}{total\PYZus{}points} \PY{o}{*} \PY{l+m+mi}{100}
    \PY{c+c1}{\PYZsh{} Stock weighting to be added per stock}
    \PY{n}{point\PYZus{}total} \PY{o}{=} \PY{n}{out\PYZus{}of\PYZus{}100} \PY{o}{*} \PY{n}{manipulate\PYZus{}weight}\PY{o}{/}\PY{n}{total\PYZus{}weight}
    \PY{n}{further\PYZus{}weights}\PY{o}{.}\PY{n}{append}\PY{p}{(}\PY{n}{point\PYZus{}total}\PY{p}{)}
\PY{n+nb}{print}\PY{p}{(}\PY{l+s+sa}{f}\PY{l+s+s2}{\PYZdq{}}\PY{l+s+s2}{Remaining weight distribution:}\PY{l+s+se}{\PYZbs{}n}\PY{l+s+si}{\PYZob{}}\PY{n}{further\PYZus{}weights}\PY{l+s+si}{\PYZcb{}}\PY{l+s+s2}{\PYZdq{}}\PY{p}{)}
\end{Verbatim}
\end{tcolorbox}

    \begin{Verbatim}[commandchars=\\\{\}]
Remaining weight distribution:
[5.537459283387622, 4.397394136807818, 4.071661237785016, 3.5830618892508146,
2.768729641693811, 2.687296416938111, 2.44299674267101, 2.44299674267101,
2.3615635179153096, 2.035830618892508, 2.035830618892508, 2.035830618892508,
1.9543973941368078, 1.9543973941368078, 1.8729641693811074, 1.7915309446254073,
1.710097719869707, 1.6286644951140063, 1.5472312703583062, 1.1400651465798046]
    \end{Verbatim}

    \begin{tcolorbox}[breakable, size=fbox, boxrule=1pt, pad at break*=1mm,colback=cellbackground, colframe=cellborder]
\prompt{In}{incolor}{40}{\boxspacing}
\begin{Verbatim}[commandchars=\\\{\}]
\PY{c+c1}{\PYZsh{}Initialize dataframe column with the remaining 50\PYZpc{} for the portfolio}
\PY{n}{base\PYZus{}weighting}\PY{p}{[}\PY{l+s+s2}{\PYZdq{}}\PY{l+s+s2}{MoreWeight}\PY{l+s+s2}{\PYZdq{}}\PY{p}{]} \PY{o}{=} \PY{n}{further\PYZus{}weights}
\PY{n}{base\PYZus{}weighting}
\end{Verbatim}
\end{tcolorbox}

            \begin{tcolorbox}[breakable, size=fbox, boxrule=.5pt, pad at break*=1mm, opacityfill=0]
\prompt{Out}{outcolor}{40}{\boxspacing}
\begin{Verbatim}[commandchars=\\\{\}]
      Weighting  MoreWeight
MON         2.5    5.537459
KO          2.5    4.397394
PG          2.5    4.071661
LMT         2.5    3.583062
SO          2.5    2.768730
PEP         2.5    2.687296
CL          2.5    2.442997
SBUX        2.5    2.442997
ABT         2.5    2.361564
PM          2.5    2.035831
NEE         2.5    2.035831
T           2.5    2.035831
GM          2.5    1.954397
BMY         2.5    1.954397
MRK         2.5    1.872964
BIIB        2.5    1.791531
AMZN        2.5    1.710098
C           2.5    1.628664
ABBV        2.5    1.547231
NKE         2.5    1.140065
\end{Verbatim}
\end{tcolorbox}
        
    \begin{tcolorbox}[breakable, size=fbox, boxrule=1pt, pad at break*=1mm,colback=cellbackground, colframe=cellborder]
\prompt{In}{incolor}{41}{\boxspacing}
\begin{Verbatim}[commandchars=\\\{\}]
\PY{c+c1}{\PYZsh{}Initialize dataframe column with total weightage}
\PY{n}{base\PYZus{}weighting} \PY{o}{=} \PY{n}{base\PYZus{}weighting}\PY{o}{.}\PY{n}{reset\PYZus{}index}\PY{p}{(}\PY{p}{)}
\PY{n}{base\PYZus{}weighting}\PY{p}{[}\PY{l+s+s2}{\PYZdq{}}\PY{l+s+s2}{TotalWeight}\PY{l+s+s2}{\PYZdq{}}\PY{p}{]} \PY{o}{=} \PY{n}{base\PYZus{}weighting}\PY{p}{[}\PY{l+s+s2}{\PYZdq{}}\PY{l+s+s2}{Weighting}\PY{l+s+s2}{\PYZdq{}}\PY{p}{]} \PY{o}{+} \PY{n}{base\PYZus{}weighting}\PY{p}{[}\PY{l+s+s2}{\PYZdq{}}\PY{l+s+s2}{MoreWeight}\PY{l+s+s2}{\PYZdq{}}\PY{p}{]}
\PY{n}{base\PYZus{}weighting}\PY{o}{.}\PY{n}{head}\PY{p}{(}\PY{p}{)}
\end{Verbatim}
\end{tcolorbox}

            \begin{tcolorbox}[breakable, size=fbox, boxrule=.5pt, pad at break*=1mm, opacityfill=0]
\prompt{Out}{outcolor}{41}{\boxspacing}
\begin{Verbatim}[commandchars=\\\{\}]
  index  Weighting  MoreWeight  TotalWeight
0   MON        2.5    5.537459     8.037459
1    KO        2.5    4.397394     6.897394
2    PG        2.5    4.071661     6.571661
3   LMT        2.5    3.583062     6.083062
4    SO        2.5    2.768730     5.268730
\end{Verbatim}
\end{tcolorbox}
        
    \begin{tcolorbox}[breakable, size=fbox, boxrule=1pt, pad at break*=1mm,colback=cellbackground, colframe=cellborder]
\prompt{In}{incolor}{42}{\boxspacing}
\begin{Verbatim}[commandchars=\\\{\}]
\PY{c+c1}{\PYZsh{}Check if the portfolio weight is correct}
\PY{n}{portfolio\PYZus{}weight\PYZus{}check} \PY{o}{=} \PY{n}{base\PYZus{}weighting}\PY{p}{[}\PY{l+s+s2}{\PYZdq{}}\PY{l+s+s2}{TotalWeight}\PY{l+s+s2}{\PYZdq{}}\PY{p}{]}\PY{o}{.}\PY{n}{sum}\PY{p}{(}\PY{n}{axis}\PY{o}{=}\PY{l+m+mi}{0}\PY{p}{)}
\PY{n+nb}{print}\PY{p}{(}\PY{l+s+s2}{\PYZdq{}}\PY{l+s+s2}{The total weighting of our portfolio is }\PY{l+s+s2}{\PYZdq{}} \PY{o}{+} \PY{n+nb}{str}\PY{p}{(}\PY{n}{portfolio\PYZus{}weight\PYZus{}check}\PY{p}{)}\PY{p}{)}
\end{Verbatim}
\end{tcolorbox}

    \begin{Verbatim}[commandchars=\\\{\}]
The total weighting of our portfolio is 100.0
    \end{Verbatim}

    \begin{tcolorbox}[breakable, size=fbox, boxrule=1pt, pad at break*=1mm,colback=cellbackground, colframe=cellborder]
\prompt{In}{incolor}{43}{\boxspacing}
\begin{Verbatim}[commandchars=\\\{\}]
\PY{c+c1}{\PYZsh{}Use the datetime feature to get the current date}
\PY{k+kn}{from} \PY{n+nn}{datetime} \PY{k+kn}{import} \PY{n}{date}
\PY{k+kn}{import} \PY{n+nn}{datetime}

\PY{n}{today} \PY{o}{=} \PY{n}{date}\PY{o}{.}\PY{n}{today}\PY{p}{(}\PY{p}{)}
\PY{c+c1}{\PYZsh{}Convert the date into a usable string for getting tickers}
\PY{n}{now} \PY{o}{=} \PY{n}{today}\PY{o}{.}\PY{n}{strftime}\PY{p}{(}\PY{l+s+s2}{\PYZdq{}}\PY{l+s+s2}{20}\PY{l+s+s2}{\PYZpc{}}\PY{l+s+s2}{y\PYZhy{}}\PY{l+s+s2}{\PYZpc{}}\PY{l+s+s2}{m\PYZhy{}}\PY{l+s+si}{\PYZpc{}d}\PY{l+s+s2}{\PYZdq{}}\PY{p}{)}
\PY{n}{now}
\end{Verbatim}
\end{tcolorbox}

            \begin{tcolorbox}[breakable, size=fbox, boxrule=.5pt, pad at break*=1mm, opacityfill=0]
\prompt{Out}{outcolor}{43}{\boxspacing}
\begin{Verbatim}[commandchars=\\\{\}]
'2021-11-26'
\end{Verbatim}
\end{tcolorbox}
        
    \begin{tcolorbox}[breakable, size=fbox, boxrule=1pt, pad at break*=1mm,colback=cellbackground, colframe=cellborder]
\prompt{In}{incolor}{44}{\boxspacing}
\begin{Verbatim}[commandchars=\\\{\}]
\PY{n}{todaym} \PY{o}{=} \PY{n}{datetime}\PY{o}{.}\PY{n}{date}\PY{o}{.}\PY{n}{today}\PY{p}{(}\PY{p}{)}
\PY{n}{week\PYZus{}ago} \PY{o}{=} \PY{n}{today} \PY{o}{\PYZhy{}} \PY{n}{datetime}\PY{o}{.}\PY{n}{timedelta}\PY{p}{(}\PY{n}{days} \PY{o}{=} \PY{l+m+mi}{7}\PY{p}{)}
\PY{c+c1}{\PYZsh{}Get the date from a week ago}
\PY{n}{previous} \PY{o}{=} \PY{n}{week\PYZus{}ago}\PY{o}{.}\PY{n}{strftime}\PY{p}{(}\PY{l+s+s2}{\PYZdq{}}\PY{l+s+s2}{20}\PY{l+s+s2}{\PYZpc{}}\PY{l+s+s2}{y\PYZhy{}}\PY{l+s+s2}{\PYZpc{}}\PY{l+s+s2}{m\PYZhy{}}\PY{l+s+si}{\PYZpc{}d}\PY{l+s+s2}{\PYZdq{}}\PY{p}{)}
\PY{n}{previous}
\end{Verbatim}
\end{tcolorbox}

            \begin{tcolorbox}[breakable, size=fbox, boxrule=.5pt, pad at break*=1mm, opacityfill=0]
\prompt{Out}{outcolor}{44}{\boxspacing}
\begin{Verbatim}[commandchars=\\\{\}]
'2021-11-19'
\end{Verbatim}
\end{tcolorbox}
        
    \begin{tcolorbox}[breakable, size=fbox, boxrule=1pt, pad at break*=1mm,colback=cellbackground, colframe=cellborder]
\prompt{In}{incolor}{45}{\boxspacing}
\begin{Verbatim}[commandchars=\\\{\}]
\PY{c+c1}{\PYZsh{}Generate a portfolio with the closing prices from a week before today to today, the reason this was done is so that if there is invalid data for}
\PY{c+c1}{\PYZsh{}today or something does not display properly, the code will automatically take the last row of the dataframe}
\PY{c+c1}{\PYZsh{}The last row of the dataframe will contain the most up to date closing prices}
\PY{n}{portfolioclose} \PY{o}{=} \PY{n}{pd}\PY{o}{.}\PY{n}{DataFrame}\PY{p}{(}\PY{p}{)}

\PY{k}{for} \PY{n}{ticker} \PY{o+ow}{in} \PY{n}{portfoliostocks}\PY{p}{:}
    \PY{n}{x} \PY{o}{=} \PY{n}{yf}\PY{o}{.}\PY{n}{Ticker}\PY{p}{(}\PY{n}{ticker}\PY{p}{)}
    
    \PY{n}{y} \PY{o}{=} \PY{n}{x}\PY{o}{.}\PY{n}{history}\PY{p}{(}\PY{n}{start} \PY{o}{=} \PY{n}{week\PYZus{}ago}\PY{p}{,} \PY{n}{end} \PY{o}{=} \PY{n}{now}\PY{p}{)}
    
    \PY{n}{portfolioclose}\PY{p}{[}\PY{n}{ticker}\PY{p}{]} \PY{o}{=} \PY{n}{y}\PY{p}{[}\PY{l+s+s1}{\PYZsq{}}\PY{l+s+s1}{Close}\PY{l+s+s1}{\PYZsq{}}\PY{p}{]}

\PY{n}{portfolioclose}\PY{o}{.}\PY{n}{tail}\PY{p}{(}\PY{p}{)}
\end{Verbatim}
\end{tcolorbox}

            \begin{tcolorbox}[breakable, size=fbox, boxrule=.5pt, pad at break*=1mm, opacityfill=0]
\prompt{Out}{outcolor}{45}{\boxspacing}
\begin{Verbatim}[commandchars=\\\{\}]
             MON         KO          PG         LMT         SO         PEP  \textbackslash{}
Date
2021-11-19  9.76  55.130001  146.820007  341.440002  62.669998  163.809998
2021-11-22  9.75  55.470001  147.800003  344.980011  63.119999  164.149994
2021-11-23  9.81  55.880001  149.440002  345.179993  63.009998  165.250000
2021-11-24  9.81  55.430000  148.660004  343.579987  63.099998  163.740005

                   CL        SBUX         ABT         PM        NEE  \textbackslash{}
Date
2021-11-19  77.199997  110.779999  126.839996  90.400002  87.920593
2021-11-22  77.559998  111.449997  125.260002  89.779999  86.974785
2021-11-23  77.930000  113.580002  124.480003  89.650002  86.925003
2021-11-24  77.760002  113.970001  125.070000  90.029999  87.209999

                    T         GM        BMY        MRK        BIIB  \textbackslash{}
Date
2021-11-19  24.129999  61.799999  57.830002  80.699997  257.190002
2021-11-22  24.700001  64.059998  57.040001  81.639999  252.210007
2021-11-23  24.760000  63.049999  57.450001  82.800003  254.149994
2021-11-24  24.469999  62.189999  56.810001  82.279999  250.130005

                   AMZN          C        ABBV         NKE
Date
2021-11-19  3676.570068  66.339996  116.239998  174.880005
2021-11-22  3572.570068  67.040001  115.650002  174.240005
2021-11-23  3580.040039  68.070000  118.879997  172.149994
2021-11-24  3580.409912  67.279999  118.660004  172.029999
\end{Verbatim}
\end{tcolorbox}
        
    \begin{tcolorbox}[breakable, size=fbox, boxrule=1pt, pad at break*=1mm,colback=cellbackground, colframe=cellborder]
\prompt{In}{incolor}{46}{\boxspacing}
\begin{Verbatim}[commandchars=\\\{\}]
\PY{c+c1}{\PYZsh{}Get the last row of the dataframe}
\PY{n}{buildclose} \PY{o}{=} \PY{n}{portfolioclose}\PY{o}{.}\PY{n}{iloc}\PY{p}{[}\PY{p}{[}\PY{o}{\PYZhy{}}\PY{l+m+mi}{1}\PY{p}{]}\PY{p}{]}

\PY{n}{buildclose} \PY{o}{=} \PY{n}{buildclose}\PY{o}{.}\PY{n}{reset\PYZus{}index}\PY{p}{(}\PY{n}{drop} \PY{o}{=} \PY{k+kc}{True}\PY{p}{)}
\PY{n}{buildclose} \PY{o}{=} \PY{n}{buildclose}\PY{o}{.}\PY{n}{transpose}\PY{p}{(}\PY{p}{)}
\PY{n}{buildclose} \PY{o}{=} \PY{n}{buildclose}\PY{o}{.}\PY{n}{reset\PYZus{}index}\PY{p}{(}\PY{p}{)}
\PY{n}{buildclose}\PY{o}{.}\PY{n}{head}\PY{p}{(}\PY{p}{)}
\end{Verbatim}
\end{tcolorbox}

            \begin{tcolorbox}[breakable, size=fbox, boxrule=.5pt, pad at break*=1mm, opacityfill=0]
\prompt{Out}{outcolor}{46}{\boxspacing}
\begin{Verbatim}[commandchars=\\\{\}]
  index           0
0   MON    9.810000
1    KO   55.430000
2    PG  148.660004
3   LMT  343.579987
4    SO   63.099998
\end{Verbatim}
\end{tcolorbox}
        
    \begin{tcolorbox}[breakable, size=fbox, boxrule=1pt, pad at break*=1mm,colback=cellbackground, colframe=cellborder]
\prompt{In}{incolor}{47}{\boxspacing}
\begin{Verbatim}[commandchars=\\\{\}]
\PY{c+c1}{\PYZsh{}Merge the total weight with the closing price, sorry about the closing price column having a column name of 0}
\PY{n}{portfolioc} \PY{o}{=} \PY{n}{base\PYZus{}weighting}\PY{o}{.}\PY{n}{merge}\PY{p}{(}\PY{n}{buildclose}\PY{p}{)}
\PY{n}{portfolioc} \PY{o}{=} \PY{n}{portfolioc}\PY{o}{.}\PY{n}{rename}\PY{p}{(}\PY{n}{columns}\PY{o}{=}\PY{p}{\PYZob{}}\PY{l+s+s1}{\PYZsq{}}\PY{l+s+s1}{index}\PY{l+s+s1}{\PYZsq{}}\PY{p}{:} \PY{l+s+s1}{\PYZsq{}}\PY{l+s+s1}{Ticker}\PY{l+s+s1}{\PYZsq{}}\PY{p}{,} \PY{l+s+s1}{\PYZsq{}}\PY{l+s+s1}{Weighting}\PY{l+s+s1}{\PYZsq{}}\PY{p}{:} \PY{l+s+s1}{\PYZsq{}}\PY{l+s+s1}{Weighting}\PY{l+s+s1}{\PYZsq{}}\PY{p}{,} \PY{l+s+s2}{\PYZdq{}}\PY{l+s+s2}{MoreWeight}\PY{l+s+s2}{\PYZdq{}} \PY{p}{:} \PY{l+s+s2}{\PYZdq{}}\PY{l+s+s2}{MoreWeight}\PY{l+s+s2}{\PYZdq{}}\PY{p}{,} \PY{l+s+s2}{\PYZdq{}}\PY{l+s+s2}{TotalWeight}\PY{l+s+s2}{\PYZdq{}} \PY{p}{:} \PY{l+s+s2}{\PYZdq{}}\PY{l+s+s2}{TotalWeight}\PY{l+s+s2}{\PYZdq{}}\PY{p}{,} \PY{l+m+mi}{0} \PY{p}{:} \PY{l+s+s2}{\PYZdq{}}\PY{l+s+s2}{ClosingPrice}\PY{l+s+s2}{\PYZdq{}}\PY{p}{\PYZcb{}}\PY{p}{)}
\PY{n}{portfolioc}
\end{Verbatim}
\end{tcolorbox}

            \begin{tcolorbox}[breakable, size=fbox, boxrule=.5pt, pad at break*=1mm, opacityfill=0]
\prompt{Out}{outcolor}{47}{\boxspacing}
\begin{Verbatim}[commandchars=\\\{\}]
   Ticker  Weighting  MoreWeight  TotalWeight  ClosingPrice
0     MON        2.5    5.537459     8.037459      9.810000
1      KO        2.5    4.397394     6.897394     55.430000
2      PG        2.5    4.071661     6.571661    148.660004
3     LMT        2.5    3.583062     6.083062    343.579987
4      SO        2.5    2.768730     5.268730     63.099998
5     PEP        2.5    2.687296     5.187296    163.740005
6      CL        2.5    2.442997     4.942997     77.760002
7    SBUX        2.5    2.442997     4.942997    113.970001
8     ABT        2.5    2.361564     4.861564    125.070000
9      PM        2.5    2.035831     4.535831     90.029999
10    NEE        2.5    2.035831     4.535831     87.209999
11      T        2.5    2.035831     4.535831     24.469999
12     GM        2.5    1.954397     4.454397     62.189999
13    BMY        2.5    1.954397     4.454397     56.810001
14    MRK        2.5    1.872964     4.372964     82.279999
15   BIIB        2.5    1.791531     4.291531    250.130005
16   AMZN        2.5    1.710098     4.210098   3580.409912
17      C        2.5    1.628664     4.128664     67.279999
18   ABBV        2.5    1.547231     4.047231    118.660004
19    NKE        2.5    1.140065     3.640065    172.029999
\end{Verbatim}
\end{tcolorbox}
        
    \begin{tcolorbox}[breakable, size=fbox, boxrule=1pt, pad at break*=1mm,colback=cellbackground, colframe=cellborder]
\prompt{In}{incolor}{48}{\boxspacing}
\begin{Verbatim}[commandchars=\\\{\}]
\PY{c+c1}{\PYZsh{}Calculate how much money will be invested into each stock based on the weight}
\PY{n}{portfoliomoney} \PY{o}{=} \PY{l+m+mi}{100000}
\PY{n}{portfolioc}\PY{p}{[}\PY{l+s+s1}{\PYZsq{}}\PY{l+s+s1}{Invest}\PY{l+s+s1}{\PYZsq{}}\PY{p}{]} \PY{o}{=} \PY{n}{portfoliomoney} \PY{o}{*} \PY{n}{portfolioc}\PY{p}{[}\PY{l+s+s1}{\PYZsq{}}\PY{l+s+s1}{TotalWeight}\PY{l+s+s1}{\PYZsq{}}\PY{p}{]}\PY{o}{/}\PY{l+m+mi}{100}
\PY{n}{portfolioc}
\end{Verbatim}
\end{tcolorbox}

            \begin{tcolorbox}[breakable, size=fbox, boxrule=.5pt, pad at break*=1mm, opacityfill=0]
\prompt{Out}{outcolor}{48}{\boxspacing}
\begin{Verbatim}[commandchars=\\\{\}]
   Ticker  Weighting  MoreWeight  TotalWeight  ClosingPrice       Invest
0     MON        2.5    5.537459     8.037459      9.810000  8037.459283
1      KO        2.5    4.397394     6.897394     55.430000  6897.394137
2      PG        2.5    4.071661     6.571661    148.660004  6571.661238
3     LMT        2.5    3.583062     6.083062    343.579987  6083.061889
4      SO        2.5    2.768730     5.268730     63.099998  5268.729642
5     PEP        2.5    2.687296     5.187296    163.740005  5187.296417
6      CL        2.5    2.442997     4.942997     77.760002  4942.996743
7    SBUX        2.5    2.442997     4.942997    113.970001  4942.996743
8     ABT        2.5    2.361564     4.861564    125.070000  4861.563518
9      PM        2.5    2.035831     4.535831     90.029999  4535.830619
10    NEE        2.5    2.035831     4.535831     87.209999  4535.830619
11      T        2.5    2.035831     4.535831     24.469999  4535.830619
12     GM        2.5    1.954397     4.454397     62.189999  4454.397394
13    BMY        2.5    1.954397     4.454397     56.810001  4454.397394
14    MRK        2.5    1.872964     4.372964     82.279999  4372.964169
15   BIIB        2.5    1.791531     4.291531    250.130005  4291.530945
16   AMZN        2.5    1.710098     4.210098   3580.409912  4210.097720
17      C        2.5    1.628664     4.128664     67.279999  4128.664495
18   ABBV        2.5    1.547231     4.047231    118.660004  4047.231270
19    NKE        2.5    1.140065     3.640065    172.029999  3640.065147
\end{Verbatim}
\end{tcolorbox}
        
    \begin{tcolorbox}[breakable, size=fbox, boxrule=1pt, pad at break*=1mm,colback=cellbackground, colframe=cellborder]
\prompt{In}{incolor}{49}{\boxspacing}
\begin{Verbatim}[commandchars=\\\{\}]
\PY{c+c1}{\PYZsh{}Calculate how many shares can be bought based on the amount invested in each stock and the price of the stock}
\PY{n}{portfolioc}\PY{p}{[}\PY{l+s+s1}{\PYZsq{}}\PY{l+s+s1}{Shares}\PY{l+s+s1}{\PYZsq{}}\PY{p}{]} \PY{o}{=} \PY{n}{portfolioc}\PY{p}{[}\PY{l+s+s2}{\PYZdq{}}\PY{l+s+s2}{Invest}\PY{l+s+s2}{\PYZdq{}}\PY{p}{]}\PY{o}{/}\PY{n}{portfolioc}\PY{p}{[}\PY{l+s+s2}{\PYZdq{}}\PY{l+s+s2}{ClosingPrice}\PY{l+s+s2}{\PYZdq{}}\PY{p}{]}
\PY{n}{portfolioc}
\end{Verbatim}
\end{tcolorbox}

            \begin{tcolorbox}[breakable, size=fbox, boxrule=.5pt, pad at break*=1mm, opacityfill=0]
\prompt{Out}{outcolor}{49}{\boxspacing}
\begin{Verbatim}[commandchars=\\\{\}]
   Ticker  Weighting  MoreWeight  TotalWeight  ClosingPrice       Invest  \textbackslash{}
0     MON        2.5    5.537459     8.037459      9.810000  8037.459283
1      KO        2.5    4.397394     6.897394     55.430000  6897.394137
2      PG        2.5    4.071661     6.571661    148.660004  6571.661238
3     LMT        2.5    3.583062     6.083062    343.579987  6083.061889
4      SO        2.5    2.768730     5.268730     63.099998  5268.729642
5     PEP        2.5    2.687296     5.187296    163.740005  5187.296417
6      CL        2.5    2.442997     4.942997     77.760002  4942.996743
7    SBUX        2.5    2.442997     4.942997    113.970001  4942.996743
8     ABT        2.5    2.361564     4.861564    125.070000  4861.563518
9      PM        2.5    2.035831     4.535831     90.029999  4535.830619
10    NEE        2.5    2.035831     4.535831     87.209999  4535.830619
11      T        2.5    2.035831     4.535831     24.469999  4535.830619
12     GM        2.5    1.954397     4.454397     62.189999  4454.397394
13    BMY        2.5    1.954397     4.454397     56.810001  4454.397394
14    MRK        2.5    1.872964     4.372964     82.279999  4372.964169
15   BIIB        2.5    1.791531     4.291531    250.130005  4291.530945
16   AMZN        2.5    1.710098     4.210098   3580.409912  4210.097720
17      C        2.5    1.628664     4.128664     67.279999  4128.664495
18   ABBV        2.5    1.547231     4.047231    118.660004  4047.231270
19    NKE        2.5    1.140065     3.640065    172.029999  3640.065147

        Shares
0   819.312838
1   124.434315
2    44.205981
3    17.704937
4    83.498095
5    31.680080
6    63.567343
7    43.371034
8    38.870741
9    50.381325
10   52.010442
11  185.362924
12   71.625623
13   78.408683
14   53.147353
15   17.157202
16    1.175870
17   61.365407
18   34.107797
19   21.159479
\end{Verbatim}
\end{tcolorbox}
        
    \begin{tcolorbox}[breakable, size=fbox, boxrule=1pt, pad at break*=1mm,colback=cellbackground, colframe=cellborder]
\prompt{In}{incolor}{50}{\boxspacing}
\begin{Verbatim}[commandchars=\\\{\}]
\PY{c+c1}{\PYZsh{}Create a final portfolio that begins with an index of 1 and it outputs all of the Tickers and Shares that will be bought}
\PY{n}{FinalPortfolio} \PY{o}{=} \PY{n}{pd}\PY{o}{.}\PY{n}{DataFrame}\PY{p}{(}\PY{p}{)}
\PY{n}{FinalPortfolio}\PY{p}{[}\PY{l+s+s2}{\PYZdq{}}\PY{l+s+s2}{Ticker}\PY{l+s+s2}{\PYZdq{}}\PY{p}{]} \PY{o}{=} \PY{n}{portfolioc}\PY{p}{[}\PY{l+s+s2}{\PYZdq{}}\PY{l+s+s2}{Ticker}\PY{l+s+s2}{\PYZdq{}}\PY{p}{]}

\PY{c+c1}{\PYZsh{}finalportfolio}
\PY{n}{FinalPortfolio}\PY{p}{[}\PY{l+s+s2}{\PYZdq{}}\PY{l+s+s2}{Price}\PY{l+s+s2}{\PYZdq{}}\PY{p}{]} \PY{o}{=} \PY{n}{portfolioc}\PY{p}{[}\PY{l+s+s2}{\PYZdq{}}\PY{l+s+s2}{ClosingPrice}\PY{l+s+s2}{\PYZdq{}}\PY{p}{]}
\PY{n}{FinalPortfolio}\PY{p}{[}\PY{l+s+s2}{\PYZdq{}}\PY{l+s+s2}{Shares}\PY{l+s+s2}{\PYZdq{}}\PY{p}{]} \PY{o}{=} \PY{n}{portfolioc}\PY{p}{[}\PY{l+s+s2}{\PYZdq{}}\PY{l+s+s2}{Shares}\PY{l+s+s2}{\PYZdq{}}\PY{p}{]}
\PY{n}{FinalPortfolio}\PY{p}{[}\PY{l+s+s2}{\PYZdq{}}\PY{l+s+s2}{Value}\PY{l+s+s2}{\PYZdq{}}\PY{p}{]} \PY{o}{=} \PY{n}{FinalPortfolio}\PY{p}{[}\PY{l+s+s2}{\PYZdq{}}\PY{l+s+s2}{Shares}\PY{l+s+s2}{\PYZdq{}}\PY{p}{]} \PY{o}{*} \PY{n}{FinalPortfolio}\PY{p}{[}\PY{l+s+s2}{\PYZdq{}}\PY{l+s+s2}{Price}\PY{l+s+s2}{\PYZdq{}}\PY{p}{]}
\PY{n}{FinalPortfolio}\PY{p}{[}\PY{l+s+s2}{\PYZdq{}}\PY{l+s+s2}{Weight}\PY{l+s+s2}{\PYZdq{}}\PY{p}{]} \PY{o}{=} \PY{n}{portfolioc}\PY{p}{[}\PY{l+s+s2}{\PYZdq{}}\PY{l+s+s2}{TotalWeight}\PY{l+s+s2}{\PYZdq{}}\PY{p}{]}

\PY{c+c1}{\PYZsh{}finalportfolio}
\PY{n}{FinalPortfolio}\PY{o}{.}\PY{n}{index} \PY{o}{=} \PY{n}{np}\PY{o}{.}\PY{n}{arange}\PY{p}{(}\PY{l+m+mi}{1}\PY{p}{,} \PY{n+nb}{len}\PY{p}{(}\PY{n}{portfolioc}\PY{p}{)} \PY{o}{+} \PY{l+m+mi}{1}\PY{p}{)}
\PY{n}{FinalPortfolio}
\end{Verbatim}
\end{tcolorbox}

            \begin{tcolorbox}[breakable, size=fbox, boxrule=.5pt, pad at break*=1mm, opacityfill=0]
\prompt{Out}{outcolor}{50}{\boxspacing}
\begin{Verbatim}[commandchars=\\\{\}]
   Ticker        Price      Shares        Value    Weight
1     MON     9.810000  819.312838  8037.459283  8.037459
2      KO    55.430000  124.434315  6897.394137  6.897394
3      PG   148.660004   44.205981  6571.661238  6.571661
4     LMT   343.579987   17.704937  6083.061889  6.083062
5      SO    63.099998   83.498095  5268.729642  5.268730
6     PEP   163.740005   31.680080  5187.296417  5.187296
7      CL    77.760002   63.567343  4942.996743  4.942997
8    SBUX   113.970001   43.371034  4942.996743  4.942997
9     ABT   125.070000   38.870741  4861.563518  4.861564
10     PM    90.029999   50.381325  4535.830619  4.535831
11    NEE    87.209999   52.010442  4535.830619  4.535831
12      T    24.469999  185.362924  4535.830619  4.535831
13     GM    62.189999   71.625623  4454.397394  4.454397
14    BMY    56.810001   78.408683  4454.397394  4.454397
15    MRK    82.279999   53.147353  4372.964169  4.372964
16   BIIB   250.130005   17.157202  4291.530945  4.291531
17   AMZN  3580.409912    1.175870  4210.097720  4.210098
18      C    67.279999   61.365407  4128.664495  4.128664
19   ABBV   118.660004   34.107797  4047.231270  4.047231
20    NKE   172.029999   21.159479  3640.065147  3.640065
\end{Verbatim}
\end{tcolorbox}
        
    \begin{tcolorbox}[breakable, size=fbox, boxrule=1pt, pad at break*=1mm,colback=cellbackground, colframe=cellborder]
\prompt{In}{incolor}{51}{\boxspacing}
\begin{Verbatim}[commandchars=\\\{\}]
\PY{n}{total\PYZus{}percent} \PY{o}{=} \PY{n}{FinalPortfolio}\PY{p}{[}\PY{l+s+s2}{\PYZdq{}}\PY{l+s+s2}{Weight}\PY{l+s+s2}{\PYZdq{}}\PY{p}{]}\PY{o}{.}\PY{n}{sum}\PY{p}{(}\PY{n}{axis}\PY{o}{=}\PY{l+m+mi}{0}\PY{p}{)}
\PY{n+nb}{print}\PY{p}{(} \PY{l+s+s2}{\PYZdq{}}\PY{l+s+s2}{The total weight of our portfolio is }\PY{l+s+s2}{\PYZpc{}}\PY{l+s+s2}{\PYZdq{}} \PY{o}{+} \PY{n+nb}{str}\PY{p}{(}\PY{n+nb}{round}\PY{p}{(}\PY{n}{total\PYZus{}weight}\PY{p}{,} \PY{l+m+mi}{2}\PY{p}{)}\PY{p}{)}\PY{p}{)}
\end{Verbatim}
\end{tcolorbox}

    \begin{Verbatim}[commandchars=\\\{\}]
The total weight of our portfolio is \%100
    \end{Verbatim}

    \begin{tcolorbox}[breakable, size=fbox, boxrule=1pt, pad at break*=1mm,colback=cellbackground, colframe=cellborder]
\prompt{In}{incolor}{52}{\boxspacing}
\begin{Verbatim}[commandchars=\\\{\}]
\PY{n}{total\PYZus{}value} \PY{o}{=} \PY{n}{FinalPortfolio}\PY{p}{[}\PY{l+s+s2}{\PYZdq{}}\PY{l+s+s2}{Value}\PY{l+s+s2}{\PYZdq{}}\PY{p}{]}\PY{o}{.}\PY{n}{sum}\PY{p}{(}\PY{n}{axis}\PY{o}{=}\PY{l+m+mi}{0}\PY{p}{)}
\PY{n+nb}{print}\PY{p}{(} \PY{l+s+s2}{\PYZdq{}}\PY{l+s+s2}{The total value of our portfolio is \PYZdl{}}\PY{l+s+s2}{\PYZdq{}} \PY{o}{+} \PY{n+nb}{str}\PY{p}{(}\PY{n+nb}{round}\PY{p}{(}\PY{n}{total\PYZus{}value}\PY{p}{,} \PY{l+m+mi}{2}\PY{p}{)}\PY{p}{)}\PY{p}{)}
\end{Verbatim}
\end{tcolorbox}

    \begin{Verbatim}[commandchars=\\\{\}]
The total value of our portfolio is \$100000.0
    \end{Verbatim}

    \begin{tcolorbox}[breakable, size=fbox, boxrule=1pt, pad at break*=1mm,colback=cellbackground, colframe=cellborder]
\prompt{In}{incolor}{53}{\boxspacing}
\begin{Verbatim}[commandchars=\\\{\}]
\PY{n}{Stocks} \PY{o}{=} \PY{n}{pd}\PY{o}{.}\PY{n}{DataFrame}\PY{p}{(}\PY{p}{)}
\PY{n}{Stocks}\PY{p}{[}\PY{l+s+s2}{\PYZdq{}}\PY{l+s+s2}{Ticker}\PY{l+s+s2}{\PYZdq{}}\PY{p}{]} \PY{o}{=} \PY{n}{portfolioc}\PY{p}{[}\PY{l+s+s2}{\PYZdq{}}\PY{l+s+s2}{Ticker}\PY{l+s+s2}{\PYZdq{}}\PY{p}{]}
\PY{n}{Stocks}\PY{p}{[}\PY{l+s+s2}{\PYZdq{}}\PY{l+s+s2}{Shares}\PY{l+s+s2}{\PYZdq{}}\PY{p}{]} \PY{o}{=} \PY{n}{portfolioc}\PY{p}{[}\PY{l+s+s2}{\PYZdq{}}\PY{l+s+s2}{Shares}\PY{l+s+s2}{\PYZdq{}}\PY{p}{]}
\PY{n}{Stocks}\PY{o}{.}\PY{n}{index} \PY{o}{=} \PY{n}{np}\PY{o}{.}\PY{n}{arange}\PY{p}{(}\PY{l+m+mi}{1}\PY{p}{,} \PY{n+nb}{len}\PY{p}{(}\PY{n}{portfolioc}\PY{p}{)} \PY{o}{+} \PY{l+m+mi}{1}\PY{p}{)}
\PY{n}{Stocks}
\end{Verbatim}
\end{tcolorbox}

            \begin{tcolorbox}[breakable, size=fbox, boxrule=.5pt, pad at break*=1mm, opacityfill=0]
\prompt{Out}{outcolor}{53}{\boxspacing}
\begin{Verbatim}[commandchars=\\\{\}]
   Ticker      Shares
1     MON  819.312838
2      KO  124.434315
3      PG   44.205981
4     LMT   17.704937
5      SO   83.498095
6     PEP   31.680080
7      CL   63.567343
8    SBUX   43.371034
9     ABT   38.870741
10     PM   50.381325
11    NEE   52.010442
12      T  185.362924
13     GM   71.625623
14    BMY   78.408683
15    MRK   53.147353
16   BIIB   17.157202
17   AMZN    1.175870
18      C   61.365407
19   ABBV   34.107797
20    NKE   21.159479
\end{Verbatim}
\end{tcolorbox}
        
    \begin{tcolorbox}[breakable, size=fbox, boxrule=1pt, pad at break*=1mm,colback=cellbackground, colframe=cellborder]
\prompt{In}{incolor}{54}{\boxspacing}
\begin{Verbatim}[commandchars=\\\{\}]
\PY{c+c1}{\PYZsh{}Output a csv}
\PY{n}{Stocks}\PY{o}{.}\PY{n}{to\PYZus{}csv}\PY{p}{(}\PY{l+s+s2}{\PYZdq{}}\PY{l+s+s2}{Stocks\PYZus{}Group\PYZus{}18.csv}\PY{l+s+s2}{\PYZdq{}}\PY{p}{)}
\end{Verbatim}
\end{tcolorbox}

    \hypertarget{contribution-declaration}{%
\subsection{Contribution Declaration}\label{contribution-declaration}}

The following team members made a meaningful contribution to this
assignment:

Daniel Kim, Kitty Cai, Andre Slavescu


    % Add a bibliography block to the postdoc
    
    
    
\end{document}
